\documentclass[12pt,a4paper]{article}
\usepackage[utf8]{inputenc}
\usepackage[french]{babel}
\usepackage{amsmath,amssymb}
\usepackage{enumitem} % Pour personnaliser les listes

% Définition des opérateurs pour la cohérence avec la notation française
\DeclareMathOperator{\divergence}{div}
\DeclareMathOperator{\rotation}{rot}

\newcommand{\vect}[1]{\vec{#1}} % Pour les vecteurs

\title{Correction de Colle de Physique - Magnétostatique}
\author{Élève de Prépa Scientifique}
\date{Période de colle : du 24 au 29 novembre}

\begin{document}
\maketitle

\section{Réponses aux questions de cours}

\subsection*{Question 1 : Que représente l’ARQS ? Définir le courant de déplacement. Dans quelles conditions est-il possible de négliger le courant de déplacement ?}

\paragraph*{Définition de l'ARQS (Approximation des Régimes Quasi-Stationnaires) magnétique :}
L'ARQS magnétique est une approximation valide en régime variable lorsque le temps caractéristique de variation des sources $T$ est beaucoup plus grand que le temps de propagation de l'onde électromagnétique sur la dimension caractéristique $L$ du système. Mathématiquement, cela se traduit par $L/c \ll T$, ou de manière équivalente, $L \ll \lambda$, où $c$ est la célérité de la lumière dans le vide et $\lambda = cT$ est la longueur d'onde associée aux variations.
Dans le cadre de l'ARQS magnétique, l'information électromagnétique est considérée comme se propageant quasi-instantanément à travers le système. La principale simplification qu'elle introduit est la négligence du courant de déplacement dans l'équation de Maxwell-Ampère.

\paragraph*{Définition du courant de déplacement :}
Le courant de déplacement, noté $\vect{j}_D$, est le terme $\varepsilon_0 \frac{\partial \vect{E}}{\partial t}$ (densité de courant de déplacement) qui apparaît dans l'équation de Maxwell-Ampère généralisée.
L'équation de Maxwell-Ampère s'écrit localement :
$$ \rotation \vect{B} = \mu_0 \left( \vect{j} + \varepsilon_0 \frac{\partial \vect{E}}{\partial t} \right) = \mu_0 (\vect{j} + \vect{j}_D) $$
Où $\vect{j}$ est la densité de courant de conduction (ou de convection). Le courant de déplacement n'est pas un courant de charges en mouvement, mais une entité physique due à la variation temporelle du champ électrique. Il a été introduit par Maxwell pour garantir la cohérence des équations et la conservation de la charge, et il est crucial pour la description des ondes électromagnétiques.

\paragraph*{Conditions pour négliger le courant de déplacement :}
Le courant de déplacement $\varepsilon_0 \frac{\partial \vect{E}}{\partial t}$ peut être négligé devant le courant de conduction $\vect{j}$ lorsque :
\begin{enumerate}[label=(\alph*)]
    \item Les variations temporelles du champ électrique sont suffisamment lentes, c'est-à-dire pour des fréquences $f$ faibles. Cela correspond à la condition $L \ll \lambda$ de l'ARQS.
    \item La conductivité $\sigma$ du milieu est élevée. Si l'on considère un conducteur ohmique ($\vect{j} = \sigma \vect{E}$) et un champ électrique variant sinusoïdalement ($\vect{E} \propto e^{i\omega t}$), la densité de courant de déplacement est $\vect{j}_D \propto i\omega\varepsilon_0 \vect{E}$. Le rapport des amplitudes est alors $|\vect{j}_D|/|\vect{j}| \approx \frac{\omega\varepsilon_0}{\sigma}$. Le courant de déplacement est négligeable si $\omega\varepsilon_0 \ll \sigma$.
\end{enumerate}
En régime stationnaire (continu), $\frac{\partial \vect{E}}{\partial t} = \vect{0}$, et le courant de déplacement est rigoureusement nul.

\subsection*{Question 2 : Montrer que le champ magnétique est à flux conservatif à partir d’une équation de Maxwell.}

Un champ vectoriel $\vect{F}$ est à flux conservatif si sa divergence est nulle, c'est-à-dire $\divergence \vect{F} = 0$.
Pour le champ magnétique $\vect{B}$, cette propriété est directement exprimée par l'équation de Maxwell-Thomson, qui est l'une des quatre équations fondamentales de l'électromagnétisme :
$$ \divergence \vect{B} = 0 $$
Cette équation signifie qu'il n'existe pas de monopôles magnétiques isolés. Les lignes de champ magnétique sont toujours fermées sur elles-mêmes. En conséquence, le flux magnétique à travers toute surface fermée est toujours nul.

\subsection*{Question 3 : Etablir la loi de Faraday à partir d’une équation de Maxwell.}

La loi de Faraday, qui décrit le phénomène d'induction électromagnétique, s'établit à partir de l'équation de Maxwell-Faraday sous sa forme locale :
$$ \rotation \vect{E} = - \frac{\partial \vect{B}}{\partial t} \quad \text{(Équation de Maxwell-Faraday)} $$
Pour obtenir la loi de Faraday sous sa forme intégrale, considérons un contour fermé $\Gamma$ délimitant une surface ouverte $S$.
\begin{enumerate}[label=(\roman*)]
    \item Calculons la circulation du champ électrique $\vect{E}$ le long du contour $\Gamma$. En utilisant le théorème de Stokes, cette circulation est égale au flux du rotationnel de $\vect{E}$ à travers la surface $S$ :
    $$ \oint_{\Gamma} \vect{E} \cdot \mathrm{d}\vect{l} = \iint_S (\rotation \vect{E}) \cdot \mathrm{d}\vect{S} $$
    \item Substituons l'expression de $\rotation \vect{E}$ donnée par l'équation de Maxwell-Faraday :
    $$ \oint_{\Gamma} \vect{E} \cdot \mathrm{d}\vect{l} = \iint_S \left( - \frac{\partial \vect{B}}{\partial t} \right) \cdot \mathrm{d}\vect{S} $$
    \item Si la surface $S$ est fixe dans le temps, l'opérateur de dérivation temporelle peut être sorti de l'intégrale :
    $$ \oint_{\Gamma} \vect{E} \cdot \mathrm{d}\vect{l} = - \frac{\mathrm{d}}{\mathrm{d}t} \left( \iint_S \vect{B} \cdot \mathrm{d}\vect{S} \right) $$
\end{enumerate}
Le terme $\oint_{\Gamma} \vect{E} \cdot \mathrm{d}\vect{l}$ représente la force électromotrice induite $\mathcal{E}$ dans le contour $\Gamma$.
Le terme $\Phi_B = \iint_S \vect{B} \cdot \mathrm{d}\vect{S}$ représente le flux du champ magnétique à travers la surface $S$.
On obtient ainsi la loi de Faraday :
$$ \mathcal{E} = - \frac{\mathrm{d}\Phi_B}{\mathrm{d}t} $$
Cette loi indique qu'une force électromotrice est induite dans un circuit (ou le long d'un contour) chaque fois que le flux magnétique à travers ce circuit varie au cours du temps. Le signe négatif est la loi de Lenz, qui exprime que le sens du courant induit est tel qu'il s'oppose à la cause qui lui donne naissance.

\subsection*{Question 4 : Etablir le théorème d’Ampère généralisé (en régime variable hors ARQS) à partir d’une équation de Maxwell.}

Le théorème d'Ampère généralisé est la forme intégrale de l'équation de Maxwell-Ampère généralisée (incluant le courant de déplacement), qui est fournie dans le texte :
$$ \rotation \vect{B} = \mu_0 \left( \vect{j} + \varepsilon_0 \frac{\partial \vect{E}}{\partial t} \right) \quad \text{(Équation de Maxwell-Ampère généralisée)} $$
Pour obtenir sa forme intégrale, nous intégrons cette équation sur une surface ouverte $S$ s'appuyant sur un contour fermé $\Gamma$.
\begin{enumerate}[label=(\roman*)]
    \item Intégrons les deux membres de l'équation sur la surface $S$ :
    $$ \iint_S (\rotation \vect{B}) \cdot \mathrm{d}\vect{S} = \iint_S \mu_0 \left( \vect{j} + \varepsilon_0 \frac{\partial \vect{E}}{\partial t} \right) \cdot \mathrm{d}\vect{S} $$
    \item Appliquons le théorème de Stokes au membre de gauche de l'équation :
    $$ \iint_S (\rotation \vect{B}) \cdot \mathrm{d}\vect{S} = \oint_{\Gamma} \vect{B} \cdot \mathrm{d}\vect{l} $$
    \item Développons le membre de droite :
    $$ \mu_0 \iint_S \vect{j} \cdot \mathrm{d}\vect{S} + \mu_0 \varepsilon_0 \iint_S \frac{\partial \vect{E}}{\partial t} \cdot \mathrm{d}\vect{S} $$
    Le premier terme, $\iint_S \vect{j} \cdot \mathrm{d}\vect{S}$, représente le courant de conduction $I_{conduction}$ traversant la surface $S$ et enlacé par le contour $\Gamma$.
    Le second terme, si la surface $S$ est fixe, peut s'écrire $\mu_0 \varepsilon_0 \frac{\mathrm{d}}{\mathrm{d}t} \left( \iint_S \vect{E} \cdot \mathrm{d}\vect{S} \right)$. Le terme $\Phi_E = \iint_S \vect{E} \cdot \mathrm{d}\vect{S}$ est le flux du champ électrique à travers la surface $S$. Le second terme représente donc $\mu_0 \varepsilon_0 \frac{\mathrm{d}\Phi_E}{\mathrm{d}t}$.
\end{enumerate}
En combinant ces résultats, on obtient le théorème d'Ampère généralisé sous sa forme intégrale :
$$ \oint_{\Gamma} \vect{B} \cdot \mathrm{d}\vect{l} = \mu_0 \left( I_{conduction} + \varepsilon_0 \frac{\mathrm{d}\Phi_E}{\mathrm{d}t} \right) $$
Ce théorème établit que la circulation du champ magnétique le long d'un contour fermé est due à la somme du courant de conduction et du "courant de déplacement" total traversant la surface délimitée par ce contour. Il est fondamental pour l'étude des champs électromagnétiques variables.

\subsection*{Question 5 : Etablir à partir des équations de Maxwell, l’équation de conservation de la charge. Que devient-elle dans le cadre de l’ARQS ?}

\paragraph*{Établissement de l'équation de conservation de la charge :}
L'équation de conservation de la charge, qui exprime la permanence de la charge électrique, s'énonce localement sous la forme :
$$ \divergence \vect{j} + \frac{\partial \rho}{\partial t} = 0 $$
où $\vect{j}$ est la densité de courant et $\rho$ est la densité volumique de charge.

Pour la dériver des équations de Maxwell, nous utilisons l'équation de Maxwell-Ampère généralisée et l'équation de Maxwell-Gauss.
\begin{enumerate}[label=(\roman*)]
    \item Partons de l'équation de Maxwell-Ampère généralisée :
    $$ \rotation \vect{B} = \mu_0 \vect{j} + \mu_0 \varepsilon_0 \frac{\partial \vect{E}}{\partial t} $$
    \item Appliquons l'opérateur divergence aux deux membres de cette équation :
    $$ \divergence (\rotation \vect{B}) = \divergence (\mu_0 \vect{j}) + \divergence \left( \mu_0 \varepsilon_0 \frac{\partial \vect{E}}{\partial t} \right) $$
    \item Nous savons que la divergence d'un rotationnel est toujours nulle pour tout champ vectoriel : $\divergence (\rotation \vect{B}) = 0$.
    \item Les constantes $\mu_0$ et $\varepsilon_0$ peuvent être sorties de l'opérateur divergence, et les opérateurs divergence et dérivation temporelle peuvent être intervertis (pour des fonctions suffisamment lisses) :
    $$ 0 = \mu_0 \divergence \vect{j} + \mu_0 \varepsilon_0 \frac{\partial}{\partial t} (\divergence \vect{E}) $$
    \item Utilisons l'équation de Maxwell-Gauss (qui décrit la relation entre le champ électrique et sa source, la charge électrique) :
    $$ \divergence \vect{E} = \frac{\rho}{\varepsilon_0} $$
    \item Substituons cette expression de $\divergence \vect{E}$ dans l'équation précédente :
    $$ 0 = \mu_0 \divergence \vect{j} + \mu_0 \varepsilon_0 \frac{\partial}{\partial t} \left( \frac{\rho}{\varepsilon_0} \right) $$
    $$ 0 = \mu_0 \divergence \vect{j} + \mu_0 \frac{\partial \rho}{\partial t} $$
    \item En divisant par $\mu_0$ (qui est une constante non nulle), nous obtenons l'équation de conservation de la charge :
    $$ \divergence \vect{j} + \frac{\partial \rho}{\partial t} = 0 $$
\end{enumerate}

\paragraph*{Que devient-elle dans le cadre de l'ARQS ?}
L'équation de conservation de la charge est une loi fondamentale qui reste intrinsèquement valable même dans le cadre de l'ARQS. Cependant, l'ARQS magnétique introduit une simplification dans l'équation de Maxwell-Ampère en négligeant le courant de déplacement :
$$ \rotation \vect{B} \approx \mu_0 \vect{j} \quad \text{(Maxwell-Ampère en ARQS magnétique)} $$
Si l'on prend la divergence de cette forme simplifiée :
$$ \divergence (\rotation \vect{B}) = \divergence (\mu_0 \vect{j}) $$
Puisque $\divergence (\rotation \vect{B}) = 0$, on en déduit :
$$ 0 = \mu_0 \divergence \vect{j} \implies \divergence \vect{j} = 0 $$
Si $\divergence \vect{j} = 0$ est la conséquence de l'ARQS magnétique (en négligeant le courant de déplacement), alors en reportant cette condition dans l'équation de conservation de la charge $\divergence \vect{j} + \frac{\partial \rho}{\partial t} = 0$, on obtient :
$$ 0 + \frac{\partial \rho}{\partial t} = 0 \implies \frac{\partial \rho}{\partial t} = 0 $$
Ainsi, dans le cadre de l'ARQS magnétique, la densité de charge $\rho$ est localement constante dans le temps. Il n'y a pas d'accumulation ou de diminution de charge en un point donné du système. C'est une conséquence directe de la négligence du courant de déplacement et de la cohérence des équations de Maxwell.

\section{Notions clés à retenir}

\begin{itemize}[label=$\blacktriangleright$]
    \item \textbf{Les quatre équations de Maxwell} (sous leurs formes locales et intégrales) sont le fondement de l'électromagnétisme :
    \begin{itemize}[label=$\circ$]
        \item Maxwell-Gauss : $\divergence \vect{E} = \frac{\rho}{\varepsilon_0}$ (sources du champ électrique)
        \item Maxwell-Thomson : $\divergence \vect{B} = 0$ (pas de monopôles magnétiques)
        \item Maxwell-Faraday : $\rotation \vect{E} = - \frac{\partial \vect{B}}{\partial t}$ (induction électromagnétique)
        \item Maxwell-Ampère généralisée : $\rotation \vect{B} = \mu_0 \left( \vect{j} + \varepsilon_0 \frac{\partial \vect{E}}{\partial t} \right)$ (sources du champ magnétique, incluant le courant de déplacement)
    \end{itemize}
    \item \textbf{L'Approximation des Régimes Quasi-Stationnaires (ARQS) magnétique} est valide quand $L \ll \lambda$. Elle permet de négliger le courant de déplacement, simplifiant Maxwell-Ampère en $\rotation \vect{B} \approx \mu_0 \vect{j}$.
    \item Le \textbf{courant de déplacement} ($\vect{j}_D = \varepsilon_0 \frac{\partial \vect{E}}{\partial t}$) n'est pas un courant de charges. Il est crucial pour la cohérence des équations de Maxwell et l'existence des ondes électromagnétiques.
    \item L'\textbf{équation de conservation de la charge} ($\divergence \vect{j} + \frac{\partial \rho}{\partial t} = 0$) est une conséquence directe des équations de Maxwell et exprime la permanence de la charge. Dans l'ARQS magnétique, elle implique $\frac{\partial \rho}{\partial t} = 0$.
    \item Les \textbf{théorèmes de Stokes et de Gauss-Ostrogradski} sont essentiels pour passer des formes locales aux formes intégrales des équations de Maxwell.
\end{itemize}

\section{Erreurs fréquentes}

\begin{itemize}[label=$\times$]
    \item \textbf{Confusion entre le théorème d'Ampère statique et généralisé :} Le théorème d'Ampère simple ($\oint_{\Gamma} \vect{B} \cdot \mathrm{d}\vect{l} = \mu_0 I_{enlacé}$) n'est valide qu'en régime stationnaire. Il est crucial d'inclure le terme de courant de déplacement ($\mu_0 \varepsilon_0 \frac{\mathrm{d}\Phi_E}{\mathrm{d}t}$) en régime variable.
    \item \textbf{Interprétation erronée du courant de déplacement :} Le courant de déplacement est souvent mal compris comme un mouvement réel de charges. Il s'agit plutôt d'une variation temporelle du champ électrique ayant les mêmes effets magnétiques qu'un courant de conduction.
    \item \textbf{Oubli des conditions de validité de l'ARQS :} L'ARQS n'est pas une vérité universelle. Ses conditions ($L \ll \lambda$) doivent toujours être mentionnées ou implicites. Son utilisation en dehors de ces conditions mène à des résultats incorrects.
    \item \textbf{Erreurs dans l'application des opérateurs différentiels :} Confondre $\divergence$, $\rotation$ ou gradient, ou oublier les identités vectorielles importantes comme $\divergence (\rotation \vect{A}) = 0$ ou $\rotation (\nabla f) = \vect{0}$.
    \item \textbf{Signes et conventions :} Attention aux signes dans les équations de Maxwell, en particulier dans Maxwell-Faraday, où le signe moins est associé à la loi de Lenz.
    \item \textbf{Dérivation incorrecte de l'équation de conservation de la charge :} Une erreur courante est de tenter de la dériver uniquement à partir de Maxwell-Ampère statique, ignorant ainsi le rôle de Maxwell-Gauss.
\end{itemize}

\end{document}