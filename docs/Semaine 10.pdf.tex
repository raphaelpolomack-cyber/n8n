\documentclass[12pt,a4paper]{article}
\usepackage[utf8]{inputenc}
\usepackage[french]{babel}
\usepackage{amsmath,amssymb}
\usepackage{mathtools}
\usepackage{siunitx}
\usepackage{esint}
\usepackage{enumitem}
\usepackage{geometry}
\geometry{a4paper, left=2.5cm, right=2.5cm, top=2.5cm, bottom=2.5cm}

% Definitions mathematiques
\DeclareMathOperator{\rot}{rot}
\DeclareMathOperator{\diver}{div}
\newcommand{\dd}[1]{\mathrm{d}#1}
\newcommand{\norm}[1]{\left\|#1\right\|}
\newcommand{\vect}[1]{\vec{#1}}

\title{Correction de colle}
\author{Correction type}
\date{\today}

\begin{document}
\maketitle

\section{Réponses aux questions de cours}

\subsection*{Questions de cours autour des équations de Maxwell et de l'ARQS magnétique}

\begin{enumerate}[label=\textbf{\arabic*)}]
    \item \textbf{Que représente l'ARQS ? Définir le courant de déplacement. Dans quelles conditions est-il possible de négliger le courant de déplacement ?}
    \begin{itemize}
        \item L'\textbf{ARQS} signifie \textbf{Approximation des Régimes Quasi Stationnaires}. Elle est applicable lorsque les champs électromagnétiques varient suffisamment lentement pour que les effets de propagation (déphasages spatiaux) soient négligeables à l'échelle caractéristique $L$ du système étudié. Cela revient à considérer que le temps de propagation d'un signal électromagnétique à travers le système ($\tau_p = L/c$) est très inférieur au temps caractéristique de variation des grandeurs du système ($\tau$). Autrement dit, $\tau_p \ll \tau$, ou $\frac{L}{c\tau} \ll 1$. Sous l'ARQS, les équations de Maxwell sont simplifiées, notamment le terme de courant de déplacement.
        \item Le \textbf{courant de déplacement} est le terme $\epsilon_0 \frac{\partial \vect{E}}{\partial t}$ dans l'équation de Maxwell-Ampère généralisée. Ce terme a été introduit par Maxwell pour assurer la conservation de la charge et pour que l'équation de Maxwell-Ampère soit compatible avec le théorème de Stokes (divergence du rotationnel nulle). Il représente la variation temporelle du flux du champ électrique, et bien qu'il ne s'agisse pas d'un mouvement de charges, il est une source de champ magnétique au même titre qu'un courant de conduction. L'intensité de courant de déplacement à travers une surface $S$ est $I_{\text{deplacement}} = \iint_S \epsilon_0 \frac{\partial \vect{E}}{\partial t} \cdot \dd{\vect{S}}$.
        \item Le courant de déplacement peut être négligé (on parle alors d'\textbf{ARQS magnétique}) lorsque sa contribution est très faible par rapport à celle du courant de conduction $\vect{j}$. Mathématiquement, on néglige le terme $\epsilon_0 \frac{\partial \vect{E}}{\partial t}$ devant $\vect{j}$ dans l'équation de Maxwell-Ampère.
        Les conditions sont :
        \begin{itemize}
            \item Pour un milieu conducteur de conductivité $\sigma$ : le temps de relaxation des charges $\tau_R = \frac{\epsilon}{\sigma}$ doit être très faible devant le temps caractéristique de variation $\tau$ du champ électrique. Si $\tau_R \ll \tau$, les charges s'adaptent très rapidement aux variations de champ, ce qui rend le courant de déplacement négligeable.
            \item Plus généralement, si le rapport $ \frac{\norm{\epsilon_0 \frac{\partial \vect{E}}{\partial t}}}{\norm{\vect{j}}} \ll 1 $.
            \item Si la fréquence $\omega$ des champs est faible telle que $\omega \ll \frac{\sigma}{\epsilon}$ ou $\omega \ll \frac{c}{L}$ (où $L$ est la dimension caractéristique du système).
        \end{itemize}
    \end{itemize}

    \item \textbf{Montrer que le champ magnétique est à flux conservatif à partir d'une équation de Maxwell.}
    L'équation de Maxwell-Thomson, ou équation de Maxwell sans divergence, s'écrit :
    \[ \diver \vect{B} = 0 \]
    Cette équation stipule que la divergence du champ magnétique est toujours nulle en tout point de l'espace. En appliquant le théorème d'Ostrogradski (ou théorème de flux-divergence) à un volume fermé $V$ délimité par une surface fermée $S$, on obtient :
    \[ \iiint_V \diver \vect{B} \, \dd{V} = \oiint_S \vect{B} \cdot \dd{\vect{S}} \]
    Puisque $\diver \vect{B} = 0$ partout, l'intégrale volumique est nulle, ce qui implique :
    \[ \oiint_S \vect{B} \cdot \dd{\vect{S}} = 0 \]
    Cette relation signifie que le flux du champ magnétique à travers toute surface fermée est nul. Il n'existe pas de "sources" ou de "puits" de champ magnétique (pas de monopôles magnétiques isolés). Les lignes de champ magnétique sont donc toujours fermées. C'est la définition d'un champ à flux conservatif.

    \item \textbf{Établir la loi de Faraday à partir d'une équation de Maxwell.}
    La loi de Faraday, sous sa forme locale, est donnée par l'équation de Maxwell-Faraday :
    \[ \rot \vect{E} = - \frac{\partial \vect{B}}{\partial t} \]
    Pour établir la loi de Faraday sous sa forme intégrale, on intègre cette équation sur une surface $S$ quelconque, bordée par un contour fermé $\Gamma$.
    \[ \iint_S \rot \vect{E} \cdot \dd{\vect{S}} = \iint_S \left( - \frac{\partial \vect{B}}{\partial t} \right) \cdot \dd{\vect{S}} \]
    En utilisant le théorème de Stokes pour le membre de gauche ($\iint_S \rot \vect{E} \cdot \dd{\vect{S}} = \oint_\Gamma \vect{E} \cdot \dd{\vect{l}}$) et en supposant que la surface $S$ est fixe (le contour $\Gamma$ ne se déforme pas), nous pouvons intervertir l'opérateur de dérivation temporelle et l'intégration spatiale pour le membre de droite :
    \[ \oint_\Gamma \vect{E} \cdot \dd{\vect{l}} = - \frac{\dd{}}{\dd{t}} \left( \iint_S \vect{B} \cdot \dd{\vect{S}} \right) \]
    Le membre de gauche, $\mathcal{E} = \oint_\Gamma \vect{E} \cdot \dd{\vect{l}}$, est la force électromotrice (f.e.m.) induite le long du contour $\Gamma$. Le terme $\Phi_B = \iint_S \vect{B} \cdot \dd{\vect{S}}$ est le flux du champ magnétique à travers la surface $S$.
    On obtient donc la loi de Faraday sous sa forme intégrale :
    \[ \mathcal{E} = - \frac{\dd{\Phi_B}}{\dd{t}} \]
    Cette loi indique qu'une variation du flux magnétique à travers une surface induit une force électromotrice le long du contour qui borde cette surface.

    \item \textbf{Établir le théorème d'Ampère généralisé (en régime variable hors ARQS) à partir d'une équation de Maxwell.}
    L'équation de Maxwell-Ampère généralisée (c'est-à-dire en régime variable, incluant le courant de déplacement) s'écrit :
    \[ \rot \vect{B} = \mu_0 \vect{j} + \mu_0 \epsilon_0 \frac{\partial \vect{E}}{\partial t} \]
    Pour établir le théorème d'Ampère généralisé sous sa forme intégrale, on intègre cette équation sur une surface $S$ quelconque, bordée par un contour fermé $\Gamma$.
    \[ \iint_S \rot \vect{B} \cdot \dd{\vect{S}} = \iint_S \left( \mu_0 \vect{j} + \mu_0 \epsilon_0 \frac{\partial \vect{E}}{\partial t} \right) \cdot \dd{\vect{S}} \]
    En utilisant le théorème de Stokes pour le membre de gauche ($\iint_S \rot \vect{B} \cdot \dd{\vect{S}} = \oint_\Gamma \vect{B} \cdot \dd{\vect{l}}$) :
    \[ \oint_\Gamma \vect{B} \cdot \dd{\vect{l}} = \mu_0 \iint_S \vect{j} \cdot \dd{\vect{S}} + \mu_0 \epsilon_0 \iint_S \frac{\partial \vect{E}}{\partial t} \cdot \dd{\vect{S}} \]
    Le terme $\iint_S \vect{j} \cdot \dd{\vect{S}}$ représente le courant de conduction $I_{\text{conduction}}$ traversant la surface $S$.
    Le terme $\epsilon_0 \iint_S \frac{\partial \vect{E}}{\partial t} \cdot \dd{\vect{S}} = \epsilon_0 \frac{\dd{}}{\dd{t}} \left( \iint_S \vect{E} \cdot \dd{\vect{S}} \right) = \epsilon_0 \frac{\dd{\Phi_E}}{\dd{t}}$ représente le courant de déplacement $I_{\text{deplacement}}$ à travers la surface $S$, où $\Phi_E$ est le flux électrique.
    On obtient ainsi le théorème d'Ampère généralisé :
    \[ \oint_\Gamma \vect{B} \cdot \dd{\vect{l}} = \mu_0 \left( I_{\text{conduction}} + I_{\text{deplacement}} \right) \]
    où $I_{\text{conduction}}$ est le courant de conduction enlacé par le contour $\Gamma$ et $I_{\text{deplacement}}$ est le courant de déplacement enlacé par le même contour.

    \item \textbf{Établir à partir des équations de Maxwell, l'équation de conservation de la charge. Que devient-elle dans le cadre de l'ARQS ?}
    On part de l'équation de Maxwell-Ampère généralisée et de l'équation de Maxwell-Gauss.
    \begin{align*} \label{eq:maxwell_ampere_gen}
        \rot \vect{B} &= \mu_0 \vect{j} + \mu_0 \epsilon_0 \frac{\partial \vect{E}}{\partial t} \\
        \diver \vect{E} &= \frac{\rho}{\epsilon_0}
    \end{align*}
    Appliquons l'opérateur divergence à l'équation de Maxwell-Ampère généralisée :
    \[ \diver (\rot \vect{B}) = \diver \left( \mu_0 \vect{j} + \mu_0 \epsilon_0 \frac{\partial \vect{E}}{\partial t} \right) \]
    On sait que la divergence d'un rotationnel est toujours nulle ($\diver (\rot \vect{X}) = 0$ pour tout champ vectoriel $\vect{X}$). Donc, le membre de gauche est nul :
    \[ 0 = \mu_0 \diver \vect{j} + \mu_0 \epsilon_0 \diver \left( \frac{\partial \vect{E}}{\partial t} \right) \]
    On peut intervertir les opérateurs de dérivation spatiale et temporelle :
    \[ 0 = \mu_0 \diver \vect{j} + \mu_0 \epsilon_0 \frac{\partial}{\partial t} (\diver \vect{E}) \]
    Substituons l'expression de $\diver \vect{E}$ à partir de l'équation de Maxwell-Gauss :
    \[ 0 = \mu_0 \diver \vect{j} + \mu_0 \epsilon_0 \frac{\partial}{\partial t} \left( \frac{\rho}{\epsilon_0} \right) \]
    Simplifions :
    \[ 0 = \mu_0 \diver \vect{j} + \mu_0 \frac{\partial \rho}{\partial t} \]
    En divisant par $\mu_0$ (qui est non nul), on obtient l'équation locale de conservation de la charge :
    \[ \diver \vect{j} + \frac{\partial \rho}{\partial t} = 0 \]
    Cette équation stipule que la variation temporelle de la densité de charge en un point est compensée par un flux de courant sortant (ou entrant) de ce point.

    \textbf{Dans le cadre de l'ARQS} :
    L'équation de conservation de la charge est une loi fondamentale qui reste toujours valide, même sous l'ARQS. Cependant, dans de nombreux cas où l'ARQS est appliquée, les régimes considérés sont suffisamment lents pour que les accumulations ou déplétions de charges ne soient pas significatives sur les échelles de temps étudiées, ou que le système soit en régime stationnaire.
    Si l'on considère un régime où la densité de charge $\rho$ est supposée constante dans le temps (régime stationnaire, $\frac{\partial \rho}{\partial t} = 0$), alors l'équation de conservation de la charge se simplifie en :
    \[ \diver \vect{j} = 0 \]
    Ceci est souvent le cas pour les courants de conduction en régime continu. L'ARQS magnétique implique la négligence du courant de déplacement, mais pas directement l'annulation de $\frac{\partial \rho}{\partial t}$. Néanmoins, pour des matériaux conducteurs, la relaxation des charges est rapide, et l'ARQS implique souvent des variations de $\rho$ suffisamment lentes pour que l'approximation $\frac{\partial \rho}{\partial t} \approx 0$ soit valable.
\end{enumerate}

\section{Notions clés à retenir}
\begin{itemize}
    \item \textbf{Équations de Maxwell sous forme locale} : Maîtriser les quatre équations (Maxwell-Gauss, Maxwell-Thomson, Maxwell-Faraday, Maxwell-Ampère généralisée) et leur signification physique.
    \item \textbf{Théorèmes intégraux} : Savoir appliquer les théorèmes d'Ostrogradski et de Stokes pour passer des équations locales aux équations intégrales (Théorème d'Ampère, Loi de Faraday).
    \item \textbf{ARQS magnétique} : Comprendre ses conditions d'application et les simplifications qu'elle apporte aux équations de Maxwell (notamment la négligence du courant de déplacement).
    \item \textbf{Courant de déplacement} : Connaître sa définition et son rôle essentiel dans la cohérence des équations de Maxwell et la conservation de la charge.
    \item \textbf{Conservation de la charge} : Démontrer l'équation locale de conservation de la charge à partir des équations de Maxwell et comprendre sa signification.
    \item \textbf{Calculs de champs magnétiques} : Être capable d'utiliser le théorème d'Ampère (stationnaire) pour calculer le champ magnétique de distributions de courants à forte symétrie (fil, cylindre, solénoïde, bobine torique), en respectant la règle du tire-bouchon.
    \item \textbf{Équilibres chimiques} : Maîtrise des grandeurs thermodynamiques ($\Delta_r H$, $\Delta_r S$, $\Delta_r G$, $K^0$) et de leurs relations. Application du critère d'évolution et de la loi de Van't Hoff pour l'influence de la température.
    \item \textbf{Optimisation des procédés chimiques} : Comprendre les différentes stratégies (modification de $K^0$, du quotient réactionnel, introduction de constituants) pour déplacer un équilibre.
\end{itemize}

\section{Erreurs fréquentes}
\begin{itemize}
    \item \textbf{Accents dans les indices mathématiques} : Oubli de supprimer les accents (ex: utiliser des indices sans accents en mode mathematique).
    \item \textbf{Syntaxe \textbf{}} : Oubli de l'accolade ouvrante ou fermante après `\textbf` (ex: toujours mettre les accolades correctement apres textbf).
    \item \textbf{Confusion entre $\Delta_r G$ et $\Delta_r G^0$} : Il est crucial de bien distinguer l'enthalpie libre de réaction standard $\Delta_r G^0$ (dépend uniquement de T) de l'enthalpie libre de réaction $\Delta_r G$ (dépend de T et des pressions/concentrations partielles), et leur rôle respectif dans le critère d'évolution et la constante d'équilibre.
    \item \textbf{Oubli du $\mu_0$ dans le théorème d'Ampère} : Ne pas oublier la perméabilité magnétique du vide $\mu_0$ dans la formule $\oint \vect{B} \cdot \dd{\vect{l}} = \mu_0 I_{\text{enlace}}$.
    \item \textbf{Orientation du contour et de la surface} : Ne pas appliquer correctement la règle du tire-bouchon pour l'orientation du contour $\Gamma$ et de la surface $S$ dans les théorèmes de Stokes et d'Ampère.
    \item \textbf{Négligence du courant de déplacement en régime variable} : Appliquer la forme simplifiée du théorème d'Ampère ($\rot \vect{B} = \mu_0 \vect{j}$) dans des situations où les champs varient rapidement, conduisant à des résultats incorrects.
    \item \textbf{Confusion ARQS magnétique et ARQS électrique} : L'ARQS magnétique néglige le courant de déplacement, tandis que l'ARQS électrique néglige la variation temporelle du champ magnétique ($\rot \vect{E} = \vect{0}$). Ces approximations ne sont pas équivalentes et ont des domaines d'application différents.
    \item \textbf{Confondre $\Delta_r G < 0$ et $\Delta_r G^0 < 0$} : Le critère d'évolution spontanée est $\Delta_r G < 0$, non $\Delta_r G^0 < 0$. Un $\Delta_r G^0 < 0$ indique seulement que la réaction est spontanée dans les conditions standards, mais pas nécessairement dans d'autres conditions.
\end{itemize}

\end{document}