\documentclass[12pt,a4paper]{article}
\usepackage[utf8]{inputenc}
\usepackage[french]{babel}
\usepackage{amsmath,amssymb}
\usepackage{mathtools}
\usepackage{siunitx}
\usepackage{esint}
\usepackage{enumitem}
\usepackage{geometry}
\geometry{a4paper, left=2.5cm, right=2.5cm, top=2.5cm, bottom=2.5cm}

% Definitions mathematiques
\DeclareMathOperator{\rot}{rot}
\DeclareMathOperator{\diver}{div}
\newcommand{\dd}[1]{\mathrm{d}#1}
\newcommand{\norm}[1]{\left\|#1\right\|}
\newcommand{\vect}[1]{\vec{#1}}

\title{Correction de colle}
\author{Correction type}
\date{\today}

\begin{document}
\maketitle

\section{Réponses aux questions de cours}

\subsection*{1) Que représente l'ARQS ? Définir le courant de déplacement. Dans quelles conditions est-il possible de négliger le courant de déplacement ?}

\begin{itemize}[label={}]
    \item \textbf{L'Approximation des Régimes Quasi-Stationnaires (ARQS)} :
    L'ARQS est une approximation valide lorsque le temps caractéristique de variation des grandeurs électromagnétiques (champs, courants) est très grand devant le temps de propagation d'un signal électromagnétique à travers la dimension caractéristique $L$ du système étudié. Autrement dit, si $\tau_{\text{variation}} \gg \tau_{\text{propagation}} = L/c$, où $c$ est la célérité de la lumière dans le milieu.
    Cela signifie que les informations se propagent de manière quasi-instantanée dans le système, et on peut souvent considérer les champs électriques et magnétiques comme découplés (électrostatique et magnétostatique) ou partiellement découplés.

    \item \textbf{Définition du courant de déplacement} :
    Le courant de déplacement est le terme $\epsilon_0 \frac{\partial \vect{E}}{\partial t}$ (ou $\frac{\partial \vect{D}}{\partial t}$ dans un milieu diélectrique) qui apparaît dans l'équation de Maxwell-Ampère généralisée. Il a les dimensions d'une densité de courant ($\text{A} \cdot \text{m}^{-2}$) et, comme le courant de conduction $\vect{j}$, il est une source de champ magnétique. Contrairement au courant de conduction qui est dû au mouvement de charges réelles, le courant de déplacement est lié à la variation temporelle du champ électrique ou du flux électrique.

    \item \textbf{Conditions pour négliger le courant de déplacement} :
    Le courant de déplacement est négligeable devant le courant de conduction $\vect{j}$ lorsque $\norm{\epsilon_0 \frac{\partial \vect{E}}{\partial t}} \ll \norm{\vect{j}}$.
    Cela se produit typiquement dans les conditions suivantes :
    \begin{itemize}
        \item Lorsque les fréquences de variation des champs sont faibles (basse fréquence).
        \item Dans les matériaux bons conducteurs où le courant de conduction est très important.
        \item Lorsque la dimension caractéristique $L$ du système est très petite devant la longueur d'onde $\lambda = c/f$ associée à la fréquence des variations.
    \end{itemize}
    En négligeant le courant de déplacement, l'équation de Maxwell-Ampère se simplifie en $\rot \vect{B} = \mu_0 \vect{j}$, ce qui est l'équation de Maxwell-Ampère en régime stationnaire.
\end{itemize}

\subsection*{2) Montrer que le champ magnétique est à flux conservatif à partir d'une équation de Maxwell.}

L'équation de Maxwell-Thomson est donnée par :
$\diver \vect{B} = 0$

D'après le théorème de Green-Ostrogradsky (ou théorème de la divergence), le flux d'un champ vectoriel à travers une surface fermée $S$ est égal à l'intégrale de la divergence de ce champ sur le volume $V$ délimité par cette surface :
$\oiint_S \vect{B} \cdot \dd{\vect{S}} = \iiint_V (\diver \vect{B}) \dd{V}$

Puisque l'équation de Maxwell-Thomson établit que $\diver \vect{B} = 0$, on en déduit que :
$\oiint_S \vect{B} \cdot \dd{\vect{S}} = \iiint_V (0) \dd{V} = 0$

Cette relation, valable pour toute surface fermée $S$, signifie que le flux du champ magnétique à travers n'importe quelle surface fermée est toujours nul. On dit que le champ magnétique est à \textbf{flux conservatif} ou \textbf{solénoïdal}. Cela implique qu'il n'existe pas de "charges magnétiques" isolées (monopôles magnétiques) et que les lignes de champ magnétique sont toujours fermées.

\subsection*{3) Etablir la loi de Faraday à partir d'une équation de Maxwell.}

La loi de Faraday, qui décrit l'induction électromagnétique, peut être établie à partir de l'équation de Maxwell-Faraday (ou loi de Faraday locale) :
$\rot \vect{E} = - \frac{\partial \vect{B}}{\partial t}$

Appliquons le théorème de Stokes à cette équation. Le théorème de Stokes relie l'intégrale de surface du rotationnel d'un champ vectoriel à l'intégrale curviligne de ce champ le long du contour fermé $\Gamma$ qui délimite cette surface $S$ :
$\oint_\Gamma \vect{E} \cdot \dd{\vect{l}} = \iint_S (\rot \vect{E}) \cdot \dd{\vect{S}}$

Substituons l'expression du rotationnel de $\vect{E}$ donnée par l'équation de Maxwell-Faraday dans le membre de droite :
$\oint_\Gamma \vect{E} \cdot \dd{\vect{l}} = \iint_S \left( - \frac{\partial \vect{B}}{\partial t} \right) \cdot \dd{\vect{S}}$

Le terme $\oint_\Gamma \vect{E} \cdot \dd{\vect{l}}$ représente la force électromotrice (f.é.m.) induite $\mathcal{E}$ le long du contour $\Gamma$.
En supposant que la surface $S$ (et son contour $\Gamma$) est fixe dans le temps, nous pouvons intervertir la dérivation partielle par rapport au temps et l'intégration spatiale :
$\mathcal{E} = - \frac{\dd{}}{\dd{t}} \left( \iint_S \vect{B} \cdot \dd{\vect{S}} \right)$

Le terme $\Phi_B = \iint_S \vect{B} \cdot \dd{\vect{S}}$ est le flux magnétique à travers la surface $S$.
On obtient ainsi la \textbf{loi de Faraday} sous sa forme intégrale (pour un circuit fixe) :
$\mathcal{E} = - \frac{\dd{\Phi_B}}{\dd{t}}$
Cette loi stipule que toute variation du flux magnétique à travers un circuit fermé induit une force électromotrice dans ce circuit.

\subsection*{4) Etablir le théorème d'Ampère généralisé (en régime variable hors ARQS) à partir d'une équation de Maxwell.}

Le théorème d'Ampère généralisé est établi à partir de l'équation de Maxwell-Ampère, qui est valable en régime variable (et hors ARQS, c'est-à-dire incluant le courant de déplacement) :
$\rot \vect{B} = \mu_0 \left( \vect{j} + \epsilon_0 \frac{\partial \vect{E}}{\partial t} \right)$

Appliquons le théorème de Stokes à cette équation. Pour un contour fermé $\Gamma$ s'appuyant sur une surface ouverte $S$ :
$\oint_\Gamma \vect{B} \cdot \dd{\vect{l}} = \iint_S (\rot \vect{B}) \cdot \dd{\vect{S}}$

Substituons l'expression du rotationnel de $\vect{B}$ donnée par l'équation de Maxwell-Ampère :
$\oint_\Gamma \vect{B} \cdot \dd{\vect{l}} = \iint_S \left[ \mu_0 \left( \vect{j} + \epsilon_0 \frac{\partial \vect{E}}{\partial t} \right) \right] \cdot \dd{\vect{S}}$

En sortant la constante $\mu_0$ de l'intégrale et en séparant les termes :
$\oint_\Gamma \vect{B} \cdot \dd{\vect{l}} = \mu_0 \left( \iint_S \vect{j} \cdot \dd{\vect{S}} + \iint_S \epsilon_0 \frac{\partial \vect{E}}{\partial t} \cdot \dd{\vect{S}} \right)$

Le premier terme de l'accolade est le courant de conduction $I_{\text{conduction}}$ enlacé par le contour $\Gamma$ : $I_{\text{conduction}} = \iint_S \vect{j} \cdot \dd{\vect{S}}$.
Le second terme est le courant de déplacement $I_{\text{deplacement}}$ enlacé par le contour $\Gamma$ : $I_{\text{deplacement}} = \iint_S \epsilon_0 \frac{\partial \vect{E}}{\partial t} \cdot \dd{\vect{S}}$.

On obtient ainsi le \textbf{théorème d'Ampère généralisé} :
$\oint_\Gamma \vect{B} \cdot \dd{\vect{l}} = \mu_0 \left( I_{\text{conduction}} + I_{\text{deplacement}} \right)$
Ce théorème stipule que la circulation du champ magnétique le long d'un contour fermé est proportionnelle à la somme des courants de conduction et de déplacement enlacés par ce contour. Il est fondamental pour la cohérence de l'électromagnétisme en régime variable.

\subsection*{5) Etablir à partir des équations de Maxwell, l'équation de conservation de la charge. Que devient-elle dans le cadre de l'ARQS ?}

Pour établir l'équation de conservation de la charge, nous utilisons les équations de Maxwell-Gauss et Maxwell-Ampère.
1. \textbf{Équation de Maxwell-Gauss} : $\diver \vect{E} = \frac{\rho}{\epsilon_0}$ (1)
2. \textbf{Équation de Maxwell-Ampère} : $\rot \vect{B} = \mu_0 \vect{j} + \mu_0 \epsilon_0 \frac{\partial \vect{E}}{\partial t}$ (2)

Appliquons l'opérateur divergence à l'équation (2) :
$\diver(\rot \vect{B}) = \diver(\mu_0 \vect{j} + \mu_0 \epsilon_0 \frac{\partial \vect{E}}{\partial t})$

Nous savons que la divergence d'un rotationnel est toujours nulle ($\diver(\rot \vect{A}) = 0$ pour tout champ vectoriel $\vect{A}$). Donc le membre de gauche est nul :
$0 = \diver(\mu_0 \vect{j}) + \diver\left(\mu_0 \epsilon_0 \frac{\partial \vect{E}}{\partial t}\right)$

Les opérateurs divergence et dérivée temporelle peuvent être intervertis pour des fonctions suffisamment régulières, et $\mu_0, \epsilon_0$ sont des constantes :
$0 = \mu_0 \diver \vect{j} + \mu_0 \epsilon_0 \frac{\partial}{\partial t} (\diver \vect{E})$

Maintenant, substituons l'expression de $\diver \vect{E}$ de l'équation (1) dans cette dernière relation :
$0 = \mu_0 \diver \vect{j} + \mu_0 \epsilon_0 \frac{\partial}{\partial t} \left( \frac{\rho}{\epsilon_0} \right)$
$0 = \mu_0 \diver \vect{j} + \mu_0 \frac{\partial \rho}{\partial t}$

En divisant par $\mu_0$ (qui est non nul), on obtient l'\textbf{équation locale de conservation de la charge} :
$\diver \vect{j} + \frac{\partial \rho}{\partial t} = 0$
Cette équation signifie que la variation temporelle de la densité de charge $\rho$ en un point est compensée par le flux de courant $\vect{j}$ sortant de ce point. En d'autres termes, la charge électrique est conservée : elle ne peut être ni créée ni détruite, seulement déplacée.

\textbf{Que devient-elle dans le cadre de l'ARQS ?}
L'équation de conservation de la charge est une loi fondamentale de la physique et \textbf{reste toujours valide} dans le cadre de l'ARQS. L'ARQS est une simplification des équations de Maxwell qui porte principalement sur la négligibilité du courant de déplacement dans l'équation de Maxwell-Ampère et la non-prise en compte des effets de propagation, mais elle n'invalide pas la conservation de la charge.

Cependant, dans certaines applications de l'ARQS (notamment en magnétostatique ou pour des régimes de très basse fréquence), on peut considérer que les densités de charge varient si lentement que $\frac{\partial \rho}{\partial t} \approx 0$. Dans ce cas, l'équation de conservation de la charge se simplifie en :
$\diver \vect{j} \approx 0$
Cette approximation signifie que les lignes de courant sont alors des boucles fermées, sans début ni fin, et qu'il n'y a pas d'accumulation ou de diminution de charge en un point. C'est l'approximation des courants stationnaires, souvent utilisée en magnétostatique.

\section{Notions clés à retenir}
\begin{itemize}
    \item \textbf{Équations de Maxwell} : Connaître les quatre équations (Maxwell-Gauss, Maxwell-Thomson, Maxwell-Faraday, Maxwell-Ampère généralisée) sous leurs formes locales et intégrales. Comprendre la signification physique de chaque terme.
    \item \textbf{Théorème d'Ampère} : Maîtriser son application pour le calcul de champs magnétiques, en identifiant correctement les symétries et invariances du système, le contour d'intégration et l'orientation de la surface enlacée.
    \item \textbf{ARQS} : Comprendre l'Approximation des Régimes Quasi-Stationnaires, ses conditions de validité ($L \ll \lambda$) et ses implications (négligence du courant de déplacement dans l'équation de Maxwell-Ampère pour les effets magnétiques).
    \item \textbf{Équilibres chimiques} : Maîtriser les grandeurs thermodynamiques standards ($\Delta_r H^0, \Delta_r S^0, \Delta_r G^0$) et les relations entre $\Delta_r G$, $\Delta_r G^0$ et le quotient réactionnel $Q_r$.
    \item \textbf{Critère d'évolution} : Savoir utiliser l'enthalpie libre de réaction $\Delta_r G$ pour déterminer le sens spontané d'une réaction ($\Delta_r G < 0$ sens direct, $\Delta_r G > 0$ sens inverse, $\Delta_r G = 0$ équilibre).
    \item \textbf{Influence des facteurs} : Comprendre l'influence de la température (relation de Van't Hoff), de la pression et des concentrations (principe de Le Châtelier) sur la position de l'équilibre chimique et sur la valeur de la constante d'équilibre $K^0(T)$.
\end{itemize}

\section{Erreurs fréquentes}
\begin{itemize}
    \item \textbf{Confusion $\Delta_r G$ et $\Delta_r G^0$} : Utiliser $\Delta_r G^0$ comme critère d'évolution directe au lieu de $\Delta_r G$. $\Delta_r G^0$ indique l'évolution si la réaction part d'un état standard, tandis que $\Delta_r G$ est le critère général dans les conditions réelles du système.
    \item \textbf{Mauvaise application du théorème d'Ampère} : Ne pas choisir un contour fermé adapté aux symétries du problème, oublier la règle du tire-bouchon pour l'orientation du flux enlacé, ou confondre le courant de conduction avec le courant de déplacement.
    \item \textbf{Oubli du courant de déplacement} : Négliger le terme $\epsilon_0 \frac{\partial \vect{E}}{\partial t}$ dans l'équation de Maxwell-Ampère en régime variable, ce qui mène à des inconsistances physiques (ex: non-conservation de la charge). Il ne doit être négligé que dans le cadre de l'ARQS, si justifié.
    \item \textbf{Accents dans les indices mathématiques} : Écrire des indices avec des accents (ex: $I_{\text{enlace}}$, $I_{\text{deplacement}}$) directement en mode mathématique `$I_{\text{enlace}}$` au lieu de `$I_{\text{enlace}}$` ou `$I_{\text{deplacement}}$` ce qui provoque des erreurs de compilation LaTeX.
    \item \textbf{Méconnaissance des relations entre opérateurs vectoriels} : Ne pas se souvenir que $\diver(\rot \vect{A}) = 0$ ou $\rot(\vect{\nabla}f) = \vect{0}$, ce qui bloque les démonstrations fondamentales (comme l'équation de conservation de la charge).
    \item \textbf{Erreurs d'application du principe de Le Châtelier} : Mal prédire le sens de déplacement de l'équilibre sous l'influence de variations de température, pression ou concentration, surtout si la réaction est exothermique/endothermique, ou implique des variations de moles gazeuses.
\end{itemize}

\end{document}