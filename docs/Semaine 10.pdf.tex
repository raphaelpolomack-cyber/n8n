\documentclass[12pt,a4paper]{article}
\usepackage[utf8]{inputenc}
\usepackage[french]{babel}
\usepackage{amsmath,amssymb}
\usepackage{physics} % Pour \grad, \div, \rot, etc.
\usepackage{upgreek} % Pour avoir des lettres grecques droites comme mu
\usepackage{geometry} % Pour gérer les marges
\geometry{a4paper, left=2.5cm, right=2.5cm, top=2.5cm, bottom=2.5cm}

\title{Correction de Colle -- Électromagnétisme (Équations de Maxwell)}
\author{Préparation à l'agrégation ou concours}
\date{24 - 29 Novembre}

\begin{document}
\maketitle

\section{Réponses aux questions de cours}

\subsection*{1) Que représente l'ARQS ? Définir le courant de déplacement. Dans quelles conditions est-il possible de négliger le courant de déplacement ?}

\paragraph*{L'Approximation des Régimes Quasi Stationnaires (ARQS):}
L'ARQS est une simplification des équations de Maxwell utilisée lorsque les champs électriques et magnétiques varient suffisamment lentement dans le temps pour que les phénomènes de propagation des ondes électromagnétiques puissent être négligés. Concrètement, cela signifie que le temps caractéristique de variation des champs ($T$) est très grand devant le temps de propagation d'une onde sur une distance caractéristique du système ($L/c$), où $c$ est la vitesse de la lumière.
Dans ce régime, les interactions électromagnétiques sont considérées comme quasi instantanées à l'échelle du système.

\paragraph*{Le courant de déplacement ($\vec{j_d}$):}
Le courant de déplacement est le terme $\varepsilon_0 \frac{\partial \vec{E}}{\partial t}$ introduit par Maxwell dans l'équation d'Ampère. Il est défini par :
$$ \vec{j_d} = \varepsilon_0 \frac{\partial \vec{E}}{\partial t} $$
Il ne s'agit pas d'un courant de porteurs de charge, mais d'un courant "fictif" lié à la variation temporelle du champ électrique. Son rôle est fondamental pour :
\begin{itemize}
    \item Assurer la conservation de la charge dans un circuit où un courant n'est pas "bouclé" par des conducteurs (ex: dans un condensateur en régime variable).
    \item Prévoir l'existence des ondes électromagnétiques (il est le "manquant" qui rend les équations symétriques et ondulatoires).
\end{itemize}

\paragraph*{Conditions pour négliger le courant de déplacement:}
Le courant de déplacement $\vec{j_d}$ peut être négligé devant le courant de conduction $\vec{j}$ lorsque :
$$ \norm{\vec{j_d}} \ll \norm{\vec{j}} \quad \text{soit} \quad \norm{\varepsilon_0 \frac{\partial \vec{E}}{\partial t}} \ll \norm{\vec{j}} $$
Ces conditions sont généralement réunies dans les cas suivants :
\begin{itemize}
    \item \textbf{À basse fréquence :} Si les champs varient très lentement, $\frac{\partial \vec{E}}{\partial t}$ est faible, et donc $\vec{j_d}$ est faible.
    \item \textbf{Dans les bons conducteurs :} Dans un bon conducteur, la loi d'Ohm locale est $\vec{j} = \sigma \vec{E}$. Le terme de courant de conduction est dominant. Le critère de validité de l'ARQS est souvent exprimé en comparant le temps de relaxation diélectrique $\tau_R = \frac{\varepsilon_0}{\sigma}$ avec la période $T$ des variations du champ. L'ARQS est valable si $T \gg \tau_R$.
\end{itemize}
Si l'on néglige le courant de déplacement dans l'équation de Maxwell-Ampère, celle-ci devient $\rot \vec{B} = \mu_0 \vec{j}$, qui est l'équation de Maxwell-Ampère en régime stationnaire.

\subsection*{2) Montrer que le champ magnétique est à flux conservatif à partir d'une équation de Maxwell.}

L'équation de Maxwell-Thomson (ou équation de non-divergence du champ magnétique) est :
$$ \div \vec{B} = 0 $$
Cette équation signifie que le champ magnétique $\vec{B}$ est un champ à divergence nulle (solénoïdal).
Par le théorème d'Ostrogradsky (ou théorème de flux-divergence), le flux d'un champ vectoriel à travers une surface fermée $S$ est égal à l'intégrale de sa divergence sur le volume $V$ délimité par $S$:
$$ \oint_S \vec{B} \cdot \dd{\vec{S}} = \iiint_V (\div \vec{B}) \dd{V} $$
Puisque $\div \vec{B} = 0$, il en découle :
$$ \oint_S \vec{B} \cdot \dd{\vec{S}} = 0 $$
Cette relation signifie que le flux de champ magnétique à travers toute surface fermée est nul. Physiquement, cela traduit l'absence de monopôles magnétiques : les lignes de champ magnétique sont toujours fermées sur elles-mêmes (elles "entrent" autant qu'elles "sortent" d'un volume fermé). C'est pourquoi on dit que le champ magnétique est à flux conservatif.

\subsection*{3) Etablir la loi de Faraday à partir d'une équation de Maxwell.}

La loi de Faraday, sous sa forme locale, est l'équation de Maxwell-Faraday :
$$ \rot \vec{E} = - \frac{\partial \vec{B}}{\partial t} $$
Pour obtenir la loi de Faraday sous sa forme intégrale (qui donne la force électromotrice induite), nous appliquons le théorème de Stokes. Considérons un contour fermé $\Gamma$ et une surface $S$ s'appuyant sur ce contour, orientée par un vecteur normal $\dd{\vec{S}}$.
Le théorème de Stokes relie la circulation d'un champ vectoriel le long d'un contour à l'intégrale du rotationnel de ce champ sur la surface s'appuyant sur ce contour :
$$ \oint_\Gamma \vec{E} \cdot \dd{\vec{l}} = \iint_S (\rot \vec{E}) \cdot \dd{\vec{S}} $$
En remplaçant $\rot \vec{E}$ par son expression tirée de l'équation de Maxwell-Faraday :
$$ \oint_\Gamma \vec{E} \cdot \dd{\vec{l}} = \iint_S \left( - \frac{\partial \vec{B}}{\partial t} \right) \cdot \dd{\vec{S}} $$
Si le contour $\Gamma$ et la surface $S$ sont fixes dans le temps, on peut intervertir l'opérateur de dérivation temporelle et l'intégrale spatiale :
$$ \oint_\Gamma \vec{E} \cdot \dd{\vec{l}} = - \frac{d}{dt} \left( \iint_S \vec{B} \cdot \dd{\vec{S}} \right) $$
Le terme $\oint_\Gamma \vec{E} \cdot \dd{\vec{l}}$ représente la force électromotrice (f.e.m.) induite le long du contour $\Gamma$.
Le terme $\iint_S \vec{B} \cdot \dd{\vec{S}}$ représente le flux magnétique $\Phi_B$ à travers la surface $S$.
Ainsi, la loi de Faraday est établie sous sa forme intégrale :
$$ \text{f.e.m.} = - \frac{d\Phi_B}{dt} $$
Elle stipule que toute variation du flux magnétique à travers un circuit induit une force électromotrice dans ce circuit.

\subsection*{4) Etablir le théorème d'Ampère généralisé (en régime variable hors ARQS) à partir d'une équation de Maxwell.}

Le théorème d'Ampère généralisé (ou équation de Maxwell-Ampère-Maxwell) sous sa forme locale est :
$$ \rot \vec{B} = \mu_0 \left( \vec{j} + \varepsilon_0 \frac{\partial \vec{E}}{\partial t} \right) $$
Pour obtenir sa forme intégrale, nous appliquons le théorème de Stokes. Considérons un contour fermé $\Gamma$ et une surface $S$ s'appuyant sur ce contour, orientée par un vecteur normal $\dd{\vec{S}}$.
$$ \oint_\Gamma \vec{B} \cdot \dd{\vec{l}} = \iint_S (\rot \vec{B}) \cdot \dd{\vec{S}} $$
En remplaçant $\rot \vec{B}$ par son expression :
$$ \oint_\Gamma \vec{B} \cdot \dd{\vec{l}} = \iint_S \mu_0 \left( \vec{j} + \varepsilon_0 \frac{\partial \vec{E}}{\partial t} \right) \cdot \dd{\vec{S}} $$
En distribuant l'intégrale et la constante $\mu_0$:
$$ \oint_\Gamma \vec{B} \cdot \dd{\vec{l}} = \mu_0 \left( \iint_S \vec{j} \cdot \dd{\vec{S}} + \iint_S \varepsilon_0 \frac{\partial \vec{E}}{\partial t} \cdot \dd{\vec{S}} \right) $$
Le terme $\iint_S \vec{j} \cdot \dd{\vec{S}}$ représente le courant de conduction $I_{conduction}$ \text{enlacé} par le contour $\Gamma$.
Le terme $\iint_S \varepsilon_0 \frac{\partial \vec{E}}{\partial t} \cdot \dd{\vec{S}}$ représente le courant de déplacement $I_{déplacement}$ \text{enlacé} par le contour $\Gamma$.
Ainsi, le théorème d'Ampère généralisé est établi sous sa forme intégrale :
$$ \oint_\Gamma \vec{B} \cdot \dd{\vec{l}} = \mu_0 (I_{conduction} + I_{déplacement}) $$
Cette équation signifie que la circulation du champ magnétique le long d'un contour fermé est proportionnelle à la somme des courants de conduction et de déplacement qui traversent toute surface s'appuyant sur ce contour.

\subsection*{5) Etablir à partir des équations de Maxwell, l'équation de conservation de la charge. Que devient-elle dans le cadre de l'ARQS ?}

Pour établir l'équation de conservation de la charge, nous utilisons l'équation de Maxwell-Ampère-Maxwell et l'équation de Maxwell-Gauss.
1.  Prenons l'équation de Maxwell-Ampère-Maxwell sous sa forme locale :
    $$ \rot \vec{B} = \mu_0 \vec{j} + \mu_0 \varepsilon_0 \frac{\partial \vec{E}}{\partial t} \quad (1) $$
2.  Appliquons l'opérateur divergence à cette équation. Nous savons que la divergence d'un rotationnel est toujours nulle ($\div (\rot \vec{V}) = 0$ pour tout champ vectoriel $\vec{V}$).
    $$ \div (\rot \vec{B}) = \div \left( \mu_0 \vec{j} + \mu_0 \varepsilon_0 \frac{\partial \vec{E}}{\partial t} \right) $$
    $$ 0 = \mu_0 \div \vec{j} + \mu_0 \varepsilon_0 \div \left( \frac{\partial \vec{E}}{\partial t} \right) $$
    Nous pouvons intervertir l'opérateur divergence et la dérivation temporelle :
    $$ 0 = \mu_0 \div \vec{j} + \mu_0 \varepsilon_0 \frac{\partial}{\partial t} (\div \vec{E}) \quad (2) $$
3.  Utilisons l'équation de Maxwell-Gauss (qui relie la divergence du champ électrique à la densité volumique de charge $\rho$) :
    $$ \div \vec{E} = \frac{\rho}{\varepsilon_0} \quad (3) $$
4.  Substituons l'expression de $\div \vec{E}$ de l'équation (3) dans l'équation (2) :
    $$ 0 = \mu_0 \div \vec{j} + \mu_0 \varepsilon_0 \frac{\partial}{\partial t} \left( \frac{\rho}{\varepsilon_0} \right) $$
    $$ 0 = \mu_0 \div \vec{j} + \mu_0 \frac{\partial \rho}{\partial t} $$
5.  Comme $\mu_0 \neq 0$, nous pouvons diviser par $\mu_0$ pour obtenir l'équation locale de conservation de la charge :
    $$ \div \vec{j} + \frac{\partial \rho}{\partial t} = 0 $$
Cette équation exprime que la variation temporelle de la densité de charge dans un volume est compensée par un flux net de courant hors de ce volume. En d'autres termes, la charge ne peut pas apparaître ou disparaître localement ; elle se déplace.

\paragraph*{Dans le cadre de l'ARQS :}
Dans le cadre de l'ARQS, le terme de courant de déplacement est négligé dans l'équation de Maxwell-Ampère, qui devient :
$$ \rot \vec{B} = \mu_0 \vec{j} $$
Si nous reprenons l'étape 2 du raisonnement ci-dessus et appliquons la divergence à cette équation simplifiée :
$$ \div (\rot \vec{B}) = \div (\mu_0 \vec{j}) $$
$$ 0 = \mu_0 \div \vec{j} $$
Ceci implique que :
$$ \div \vec{j} = 0 $$
En substituant ce résultat dans l'équation de conservation de la charge exacte ($\div \vec{j} + \frac{\partial \rho}{\partial t} = 0$), nous obtenons :
$$ 0 + \frac{\partial \rho}{\partial t} = 0 \quad \Rightarrow \quad \frac{\partial \rho}{\partial t} = 0 $$
Ainsi, dans le cadre de l'ARQS, la densité de charge $\rho$ est considérée comme stationnaire (indépendante du temps) en tout point de l'espace. Cela signifie qu'il n'y a pas d'accumulation ni de disparition locale de charge. Les courants sont considérés comme permanents (non variables dans le temps ou à des fréquences très basses).

\section{Notions clés à retenir}

\begin{itemize}
    \item \textbf{Les quatre équations de Maxwell} (forme locale et intégrale) et leur signification physique (Maxwell-Gauss, Maxwell-Thomson, Maxwell-Faraday, Maxwell-Ampère-Maxwell).
    \item \textbfLe courant de déplacement ($\varepsilon_0 \partial \vec{E}/\partial t$)} : Son rôle fondamental dans la cohérence des équations et l'existence des ondes électromagnétiques. Ce n'est pas un courant de charges.
    \item \textbfL'ARQS (Approximation des Régimes Quasi Stationnaires)} : Définition, conditions de validité (basse fréquence, bon conducteur), et implications (négligence du courant de déplacement, densité de charge stationnaire).
    \item \textbfLe lien entre les équations de Maxwell et la conservation de la charge} : La cohérence des équations est intrinsèquement liée à la conservation de la charge.
    \item \textbfLes théorèmes fondamentaux de l'analyse vectorielle} :
    \begin{itemize}
        \item Théorème de Stokes (circulation $\leftrightarrow$ flux de rotationnel).
        \item Théorème d'Ostrogradsky (flux $\leftrightarrow$ intégrale de divergence).
    \end{itemize}
    Ces théorèmes sont essentiels pour passer des formes locales aux formes intégrales des équations de Maxwell et vice-versa.
\end{itemize}

\section{Erreurs fréquentes}

\begin{itemize}
    \item \textbf{Oubli du courant de déplacement} : La principale erreur est d'appliquer le théorème d'Ampère sans le terme de courant de déplacement en régime variable, ce qui conduit à des contradictions (e.g., non-conservation de la charge).
    \item \textbf{Confusion entre régime stationnaire et quasi-stationnaire} : Le régime stationnaire est un cas particulier de l'ARQS où les champs sont constants dans le temps ($\partial/\partial t = 0$). L'ARQS autorise des variations lentes.
    \item \textbf{Mauvaise interprétation de $\div \vec{j} = 0$ en ARQS} : Bien que $\div \vec{j} = 0$ implique $\partial \rho / \partial t = 0$ en ARQS, il faut comprendre que cela vient de la négligence du courant de déplacement, ce qui simplifie la conservation de la charge mais ne signifie pas que la conservation elle-même disparaît.
    \item \textbf{Intervertir $\frac{d}{dt}$ et l'intégrale spatiale} : En général, cette interversion n'est valide que si la surface d'intégration est fixe dans le temps. Pour une surface déformable, il faut utiliser la dérivée matérielle ou le théorème de Reynolds. Pour les questions posées, la surface est implicitement fixe.
    \item \textbf{Ne pas maîtriser les conventions d'orientation} : Pour les théorèmes de Stokes et Ampère, l'orientation du contour $\Gamma$ et de la surface $S$ (via le vecteur normal $\dd{\vec{S}}$) est cruciale et doit suivre la règle du tire-bouchon.
\end{itemize}

\end{document}