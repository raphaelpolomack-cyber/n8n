\documentclass{article}
\usepackage[utf8]{inputenc}
\usepackage[T1]{fontenc}
\usepackage[french]{babel}
\usepackage{amsmath, amssymb}
\usepackage{physics} % Provides \div, \curl, \grad, \pdv for convenience

% Custom commands for consistency
\newcommand{\vect}[1]{\vec{#1}} % For vectors
\newcommand{\ud}{\mathrm{d}} % For differentials

\begin{document}

\title{Correction de Colle -- Électromagnétisme et Thermodynamique Chimique}
\author{}
\date{24 au 29 novembre}
\maketitle
\thispagestyle{empty} % Supprime le numéro de page sur la première page

\section*{Réponses aux questions de cours}

\subsection*{1) Que représente l'ARQS ? Définir le courant de déplacement. Dans quelles conditions est-il possible de négliger le courant de déplacement ?}

\begin{itemize}
    \item \textbf{Approximation des Régimes Quasi-Stationnaires (ARQS) :}\\
    L'ARQS est une approximation qui suppose que les champs électriques et magnétiques varient suffisamment lentement dans le temps pour que les phénomènes de propagation des ondes électromagnétiques soient négligeables à l'échelle du système étudié. Plus précisément, le temps caractéristique de variation $\tau$ des grandeurs physiques est supposé très grand devant le temps de propagation $L/c$ des ondes électromagnétiques sur une distance $L$ caractéristique du système ($L \ll c\tau$).
    Cela permet de considérer les interactions électromagnétiques comme instantanées. Dans le cadre de l'équation de Maxwell-Ampère, cela se traduit souvent par la possibilité de négliger le terme de courant de déplacement.

    \item \textbf{Définition du courant de déplacement :}\\
    Le courant de déplacement, noté $\vect{J}_d$, est un terme introduit par Maxwell dans l'équation de Maxwell-Ampère. Il est défini par $\vect{J}_d = \epsilon_0 \frac{\partial \vect{E}}{\partial t}$, où $\epsilon_0$ est la permittivité diélectrique du vide et $\frac{\partial \vect{E}}{\partial t}$ est la dérivée temporelle du champ électrique. Ce n'est pas un courant de porteurs de charge en mouvement, mais un "courant" associé à la variation du champ électrique dans le temps, produisant un champ magnétique de la même manière qu'un courant de conduction.

    \item \textbf{Conditions pour négliger le courant de déplacement :}\\
    Le courant de déplacement est négligeable lorsque sa norme est très petite devant celle du courant de conduction $\vect{J}$. Les conditions principales sont :
    \begin{enumerate}
        \item \textbf{En régime stationnaire :} Par définition, en régime stationnaire, toutes les grandeurs physiques sont indépendantes du temps, donc $\frac{\partial \vect{E}}{\partial t} = \vect{0}$. Le courant de déplacement est alors nul. C'est ce que suggère la fiche de colle avec `rot B = μ₀ j` en régime stationnaire.
        \item \textbf{En régime quasi-stationnaire où les courants de conduction dominent :} Dans un conducteur (où le courant de conduction $\vect{J} = \sigma \vect{E}$ est présent), si la fréquence de variation des champs est faible ou si la conductivité $\sigma$ du milieu est élevée, le courant de conduction est prédominant par rapport au courant de déplacement. On peut le négliger si $\left\| \epsilon_0 \frac{\partial \vect{E}}{\partial t} \right\| \ll \left\| \vect{J} \right\|$. Typiquement, pour des fréquences $f \ll \frac{\sigma}{2\pi\epsilon_0}$.
    \end{enumerate}
\end{itemize}

\subsection*{2) Montrer que le champ magnétique est à flux conservatif à partir d'une équation de Maxwell.}

Le champ magnétique $\vect{B}$ est dit à flux conservatif si son divergence est nulle en tout point de l'espace, c'est-à-dire si $\div \vect{B} = 0$.
L'une des quatre équations de Maxwell, spécifiquement l'\textbf{équation de Maxwell-Thomson}, énonce directement cette propriété :
$$ \div \vect{B} = 0 $$
Cette équation signifie qu'il n'existe pas de "sources" ou de "puits" de champ magnétique isolés, ce qui traduit l'absence de monopôles magnétiques. Le flux du champ magnétique à travers toute surface fermée est toujours nul.

\subsection*{3) Etablir la loi de Faraday à partir d'une équation de Maxwell.}

La loi de Faraday, ou loi de l'induction électromagnétique, est directement issue de l'\textbf{équation de Maxwell-Faraday}. Bien que non explicitement écrite sous cette forme dans la fiche, elle est une équation de Maxwell fondamentale :
$$ \curl \vect{E} = - \frac{\partial \vect{B}}{\partial t} $$
Pour établir la loi de Faraday sous sa forme intégrale, on intègre cette équation sur une surface $S$ s'appuyant sur un contour fermé $\Gamma$ (orienté par la règle du tire-bouchon) et on applique le théorème de Stokes :
$$ \iint_S (\curl \vect{E}) \cdot \ud\vect{S} = \oint_\Gamma \vect{E} \cdot \ud\vect{l} $$
$$ \iint_S \left( - \frac{\partial \vect{B}}{\partial t} \right) \cdot \ud\vect{S} = - \frac{\ud}{\ud t} \iint_S \vect{B} \cdot \ud\vect{S} $$
En combinant ces deux expressions, on obtient la loi de Faraday sous sa forme intégrale :
$$ \boxed{\oint_\Gamma \vect{E} \cdot \ud\vect{l} = - \frac{\ud \Phi_B}{\ud t}} $$
où $\Phi_B = \iint_S \vect{B} \cdot \ud\vect{S}$ est le flux magnétique à travers la surface $S$. L'intégrale $\oint_\Gamma \vect{E} \cdot \ud\vect{l}$ représente la force électromotrice (f.e.m.) induite le long du contour $\Gamma$. Cette loi stipule qu'une variation du flux magnétique à travers une surface génère une force électromotrice le long du contour délimitant cette surface.

\subsection*{4) Etablir le théorème d'Ampère généralisé (en régime variable hors ARQS) à partir d'une équation de Maxwell.}

Le théorème d'Ampère généralisé est établi à partir de l'\textbf{équation de Maxwell-Ampère}, qui est donnée dans la fiche :
$$ \curl \vect{B} = \mu_0 \left( \vect{J} + \epsilon_0 \frac{\partial \vect{E}}{\partial t} \right) $$
Pour obtenir le théorème d'Ampère sous sa forme intégrale, on intègre cette équation sur une surface $S$ délimitée par un contour fermé $\Gamma$. On applique ensuite le théorème de Stokes, qui est rappelé dans la fiche de colle :
$$ \oint_\Gamma \vect{B} \cdot \ud\vect{l} = \iint_S (\curl \vect{B}) \cdot \ud\vect{S} $$
En substituant l'expression de $\curl \vect{B}$ par l'équation de Maxwell-Ampère :
$$ \oint_\Gamma \vect{B} \cdot \ud\vect{l} = \iint_S \mu_0 \left( \vect{J} + \epsilon_0 \frac{\partial \vect{E}}{\partial t} \right) \cdot \ud\vect{S} $$
En sortant la constante $\mu_0$ de l'intégrale et en distribuant l'intégrale :
$$ \boxed{\oint_\Gamma \vect{B} \cdot \ud\vect{l} = \mu_0 \left( \iint_S \vect{J} \cdot \ud\vect{S} + \iint_S \epsilon_0 \frac{\partial \vect{E}}{\partial t} \cdot \ud\vect{S} \right)} $$
$$ \boxed{\oint_\Gamma \vect{B} \cdot \ud\vect{l} = \mu_0 \left( I_{\text{conduction enlacée}} + I_{\text{déplacement enlacée}} \right)} $$
où $I_{\text{conduction enlacée}}$ est le courant de conduction traversant la surface $S$, et $I_{\text{déplacement enlacée}}$ est le "flux du courant de déplacement" à travers la même surface. C'est la forme généralisée du théorème d'Ampère, valable en régime variable, incluant le terme de courant de déplacement.

\subsection*{5) Etablir à partir des équations de Maxwell, l'équation de conservation de la charge. Que devient-elle dans le cadre de l'ARQS ?}

L'équation de conservation de la charge, qui exprime que la charge électrique ne peut être ni créée ni détruite, peut être établie à partir des équations de Maxwell. Elle s'écrit sous forme locale : $\div \vect{J} + \frac{\partial \rho}{\partial t} = 0$.

\textbf{Démonstration :}
\begin{enumerate}
    \item On part de l'équation de Maxwell-Ampère (généralisée) :
    $$ \curl \vect{B} = \mu_0 \vect{J} + \mu_0 \epsilon_0 \frac{\partial \vect{E}}{\partial t} $$
    \item On prend la divergence de chaque membre de cette équation :
    $$ \div (\curl \vect{B}) = \div \left( \mu_0 \vect{J} + \mu_0 \epsilon_0 \frac{\partial \vect{E}}{\partial t} \right) $$
    \item On sait que la divergence d'un rotationnel est toujours nulle ($\div (\curl \vect{X}) = 0$ pour tout champ vectoriel $\vect{X}$) :
    $$ 0 = \mu_0 \div \vect{J} + \mu_0 \epsilon_0 \div \left( \frac{\partial \vect{E}}{\partial t} \right) $$
    \item En simplifiant par $\mu_0$ (qui est non nul) et en intervertissant l'opérateur divergence et la dérivée temporelle (sous certaines conditions de régularité des champs) :
    $$ 0 = \div \vect{J} + \epsilon_0 \frac{\partial}{\partial t} (\div \vect{E}) $$
    \item On utilise maintenant l'équation de Maxwell-Gauss, qui relie la divergence du champ électrique à la densité de charge volumique $\rho$ :
    $$ \div \vect{E} = \frac{\rho}{\epsilon_0} $$
    \item En substituant cette expression dans l'équation précédente :
    $$ 0 = \div \vect{J} + \epsilon_0 \frac{\partial}{\partial t} \left( \frac{\rho}{\epsilon_0} \right) $$
    $$ 0 = \div \vect{J} + \frac{\partial \rho}{\partial t} $$
\end{enumerate}
Ainsi, on obtient l'\textbf{équation de conservation de la charge} :
$$ \boxed{\div \vect{J} + \frac{\partial \rho}{\partial t} = 0} $$

\textbf{Dans le cadre de l'ARQS :}
Dans le cadre de l'ARQS, si l'on néglige le courant de déplacement (c'est-à-dire si on utilise l'équation de Maxwell-Ampère sous sa forme de régime stationnaire, $\curl \vect{B} = \mu_0 \vect{J}$), alors en prenant la divergence des deux côtés :
$$ \div (\curl \vect{B}) = \div (\mu_0 \vect{J}) $$
$$ 0 = \mu_0 \div \vect{J} $$
Ce qui implique :
$$ \div \vect{J} = 0 $$
En reportant cette condition dans l'équation de conservation de la charge :
$$ 0 + \frac{\partial \rho}{\partial t} = 0 $$
Soit :
$$ \frac{\partial \rho}{\partial t} = 0 $$
Cela signifie que dans cette formulation simplifiée de l'ARQS (où le courant de déplacement est négligé), la densité de charge volumique $\rho$ est constante au cours du temps à tout point de l'espace. Les charges ne s'accumulent pas et ne disparaissent pas localement ; elles se déplacent en un flux conservatif.

\section*{Notions clés à retenir}

\subsection*{Thermodynamique Chimique}
\begin{itemize}
    \item \textbf{Grandeurs de réaction standard :} Enthalpie de réaction ($\Delta_r H^0$), entropie de réaction ($\Delta_r S^0$), enthalpie libre de réaction ($\Delta_r G^0$). Elles sont définies pour une avancement de 1 mol dans les conditions standards.
    \item \textbf{Enthalpie libre de réaction et quotient de réaction :} $\Delta_r G = \Delta_r G^0 + RT \ln Q_r$. À l'équilibre, $\Delta_r G = 0$, d'où $\Delta_r G^0 = -RT \ln K^0(T)$. On en déduit $\Delta_r G = RT \ln \left( \frac{Q_r}{K^0(T)} \right)$.
    \item \textbf{Critère d'évolution spontanée :} $\Delta_r G \cdot d\xi < 0$. Une réaction évolue spontanément dans le sens où $\Delta_r G$ et $d\xi$ sont de signes opposés (si $\Delta_r G < 0$, $\xi$ augmente ; si $\Delta_r G > 0$, $\xi$ diminue).
    \item \textbf{Loi de Van't Hoff :} Décrit l'influence de la température sur la constante d'équilibre : $\frac{d \ln (K^0)}{dT} = \frac{\Delta_r H^0}{RT^2}$.
    \item \textbf{Approximation d'Ellingham :} Hypothèse simplificatrice où $\Delta_r H^0$ et $\Delta_r S^0$ sont considérés indépendants de la température. Permet d'écrire $\Delta_r G^0(T) = \Delta_r H^0 - T \Delta_r S^0$.
    \item \textbf{Optimisation d'un procédé chimique :} Possible par modification de $K^0$ (température, loi de Van't Hoff) ou du quotient réactionnel $Q_r$ (pression, introduction de constituants actifs/inertes).
\end{itemize}

\subsection*{Magnétostatique et Électromagnétisme}
\begin{itemize}
    \item \textbf{Équations de Maxwell :}
    \begin{itemize}
        \item Maxwell-Thomson : $\div \vect{B} = 0$ (pas de monopôles magnétiques, champ à flux conservatif).
        \item Maxwell-Ampère (généralisée) : $\curl \vect{B} = \mu_0 \left( \vect{J} + \epsilon_0 \frac{\partial \vect{E}}{\partial t} \right)$. En régime stationnaire, $\curl \vect{B} = \mu_0 \vect{J}$.
        \item Maxwell-Faraday (loi de Faraday) : $\curl \vect{E} = - \frac{\partial \vect{B}}{\partial t}$ (création de champ électrique par variation de champ magnétique).
        \item Maxwell-Gauss : $\div \vect{E} = \frac{\rho}{\epsilon_0}$ (sources du champ électrique : les charges).
    \end{itemize}
    \item \textbf{Théorème d'Ampère :} $\oint_\Gamma \vect{B} \cdot \ud\vect{l} = \mu_0 I_{\text{enlacé}}$ (forme simplifiée en magnétostatique).
    \item \textbf{Théorème de Stokes-Ampère :} Relation entre la forme intégrale et différentielle du théorème d'Ampère : $\oint_\Gamma \vect{B} \cdot \ud\vect{l} = \iint_S (\curl \vect{B}) \cdot \ud\vect{S}$.
    \item \textbf{Densité d'énergie :} Magnétique $w_m = \frac{B^2}{2\mu_0}$ et électromagnétique $w_{em} = \frac{1}{2}\epsilon_0 E^2 + \frac{B^2}{2\mu_0}$.
    \item \textbf{ARQS (Approximation des Régimes Quasi-Stationnaires) :} Condition sous laquelle le terme de courant de déplacement dans Maxwell-Ampère peut être négligé, permettant une simplification des calculs.
    \item \textbf{Courant de déplacement :} $\vect{J}_d = \epsilon_0 \frac{\partial \vect{E}}{\partial t}$, un terme de "courant" lié à la variation du champ électrique.
    \item \textbf{Conservation de la charge :} $\div \vect{J} + \frac{\partial \rho}{\partial t} = 0$.
\end{itemize}

\section*{Erreurs fréquentes}

\subsection*{Thermodynamique Chimique}
\begin{itemize}
    \item \textbf{Confusion entre $\Delta_r G$ et $\Delta_r G^0$ :} $\Delta_r G^0$ est l'enthalpie libre de réaction dans les conditions standards, liée à $K^0$. $\Delta_r G$ est l'enthalpie libre de réaction dans des conditions quelconques, liée à $Q_r$ et déterminant le sens d'évolution.
    \item \textbf{Oubli du facteur $RT$ ou du logarithme :} Dans les relations $\Delta_r G = \Delta_r G^0 + RT \ln Q_r$ ou $\Delta_r G^0 = -RT \ln K^0$.
    \item \textbf{Erreur de signe dans la loi de Van't Hoff :} Ne pas confondre la dépendance de $K^0$ avec la température avec le signe de $\Delta_r H^0$.
    \item \textbf{Mauvaise application du critère d'évolution :} Le critère est $\Delta_r G \cdot d\xi < 0$, pas seulement $\Delta_r G < 0$.
\end{itemize}

\subsection*{Magnétostatique et Électromagnétisme}
\begin{itemize}
    \item \textbf{Oubli du courant de déplacement :} Ne pas le négliger par défaut en régime variable. Le théorème d'Ampère "classique" n'est valable qu'en magnétostatique ou ARQS si $\vect{J}_d$ est négligeable.
    \item \textbf{Confusion entre $\div \vect{B} = 0$ et $\div \vect{E} = 0$ :} Ces deux équations sont distinctes et ont des significations physiques différentes (pas de monopôles magnétiques vs. sources du champ électrique).
    \item \textbf{Erreur de signe dans la loi de Faraday :} L'équation est $\curl \vect{E} = - \frac{\partial \vect{B}}{\partial t}$, le signe moins est crucial (loi de Lenz).
    \item \textbf{Orientation du contour et de la surface :} Pour le théorème d'Ampère et de Stokes, l'orientation du contour $\Gamma$ et de la surface $S$ (via le vecteur $\ud\vect{S}$) doit être cohérente selon la règle du tire-bouchon. Une erreur d'orientation peut entraîner un signe erroné pour $I_{\text{enlacé}}$.
    \item \textbf{Utilisation incorrecte de l'ARQS :} Ne pas l'appliquer sans vérifier que les conditions de validité sont remplies (fréquence faible, conductivité élevée, etc.).
    \item \textbf{Mémorisation erronée des opérateurs différentiels :} Confondre $\div$, $\curl$ ou $\grad$.
\end{itemize}

\end{document}