\documentclass[12pt,a4paper]{article}
\usepackage[utf8]{inputenc}
\usepackage[french]{babel}
\usepackage{amsmath,amssymb}
\usepackage{mathtools}
\usepackage{siunitx}
\usepackage{esint}
\usepackage{enumitem}
\usepackage{geometry}
\geometry{a4paper, left=2.5cm, right=2.5cm, top=2.5cm, bottom=2.5cm}

% Definitions mathematiques
\DeclareMathOperator{\rot}{rot}
\DeclareMathOperator{\diver}{div}
\newcommand{\dd}[1]{\mathrm{d}#1}
\newcommand{\norm}[1]{\left\|#1\right\|}
\newcommand{\vect}[1]{\vec{#1}}

\title{Correction de colle}
\author{Correction type}
\date{\today}

\begin{document}
\maketitle

\section{Réponses aux questions de cours}

\subsection{Optimisation d'un procédé chimique}

\begin{itemize}[label={}]
    \item \textbf{Modification de la valeur de $\boldsymbol{K^{\circ}}$}
    La valeur de la constante d'équilibre $K^{\circ}$ dépend de la température. L'influence de la température est donnée par la \textbf{loi de Van't Hoff} :
    $\frac{\mathrm{d}(\ln K^{\circ})}{\mathrm{d}T} = \frac{\Delta_{\text{r}}H^{\circ}}{RT^2}$
    \begin{itemize}
        \item Si $\Delta_{\text{r}}H^{\circ} > 0$ (réaction endothermique), $K^{\circ}$ augmente avec la température.
        \item Si $\Delta_{\text{r}}H^{\circ} < 0$ (réaction exothermique), $K^{\circ}$ diminue avec la température.
    \end{itemize}
    Pour optimiser le rendement, on choisira la température qui favorise le sens direct (vers les produits).

    \item \textbf{Modification de la valeur du quotient réactionnel $\boldsymbol{Q_r}$}
    Le quotient réactionnel peut être modifié par :
    \begin{itemize}
        \item \textbf{Introduction d'un constituant actif} :
        \begin{itemize}
            \item L'ajout d'un réactif (ou l'élimination d'un produit) diminue $Q_r$. Le système évolue dans le sens direct pour atteindre l'équilibre, augmentant le rendement.
            \item L'ajout d'un produit (ou l'élimination d'un réactif) augmente $Q_r$. Le système évolue dans le sens indirect, diminuant le rendement.
        \end{itemize}
        \item \textbf{Influence de la pression (pour les systèmes gazeux)} :
        Selon la \textbf{loi de Le Châtelier}, une augmentation de la pression totale (en diminuant le volume) déplace l'équilibre dans le sens qui diminue le nombre total de moles gazeuses. Une diminution de la pression (en augmentant le volume) déplace l'équilibre dans le sens qui augmente le nombre total de moles gazeuses.
        \item \textbf{Introduction d'un constituant inerte} :
        \begin{itemize}
            \item À volume constant : L'ajout d'un inerte n'affecte pas les pressions partielles ou les concentrations des constituants actifs, donc $Q_r$ ne change pas et l'équilibre n'est pas déplacé.
            \item À pression constante : L'ajout d'un inerte augmente le volume total, diminuant les pressions partielles des constituants actifs. Cela revient à une diminution de pression, favorisant le sens qui augmente le nombre de moles gazeuses.
        \end{itemize}
    \end{itemize}

    \item \textbf{Critère d'évolution utilisé}
    Le critère d'évolution spontanée d'un système est donné par la deuxième loi de la thermodynamique (à $T, P$ constantes, pour des transformations réversibles ou irréversibles) :
    $\Delta_{\text{r}}G \cdot \dd{\xi} < 0$
    où $\Delta_{\text{r}}G$ est l'enthalpie libre de réaction et $\dd{\xi}$ est l'avancement infinitésimal de la réaction.
    On peut aussi utiliser le quotient réactionnel $Q_r$ par rapport à la constante d'équilibre $K^{\circ}$ :
    Si $Q_r < K^{\circ}$, le système évolue dans le sens direct.
    Si $Q_r > K^{\circ}$, le système évolue dans le sens indirect.
\end{itemize}

\subsection{Électromagnétisme V : Les régimes variables}

\subsubsection{Courants de déplacement et ARQS magnétique}
\begin{itemize}[label={}]
    \item \textbf{Compatibilité des équations de Maxwell avec la conservation de la charge}
    L'équation de Maxwell-Ampère est $\rot \vect{B} = \mu_0 \vect{j} + \mu_0 \epsilon_0 \frac{\partial \vect{E}}{\partial t}$.
    En prenant la divergence de cette équation :
    $\diver(\rot \vect{B}) = \mu_0 \diver \vect{j} + \mu_0 \epsilon_0 \diver \left( \frac{\partial \vect{E}}{\partial t} \right)$
    Puisque $\diver(\rot \vect{X}) = 0$ pour tout champ $\vect{X}$ :
    $0 = \mu_0 \diver \vect{j} + \mu_0 \epsilon_0 \frac{\partial}{\partial t} (\diver \vect{E})$
    D'après l'équation de Maxwell-Gauss, $\diver \vect{E} = \frac{\rho}{\epsilon_0}$. En substituant :
    $0 = \mu_0 \diver \vect{j} + \mu_0 \epsilon_0 \frac{\partial}{\partial t} \left( \frac{\rho}{\epsilon_0} \right)$
    $0 = \mu_0 \diver \vect{j} + \mu_0 \frac{\partial \rho}{\partial t}$
    On obtient donc $\diver \vect{j} + \frac{\partial \rho}{\partial t} = 0$, qui est l'\textbf{équation locale de conservation de la charge}. Les équations de Maxwell sont donc compatibles avec le principe de conservation de la charge.

    \item \textbf{Simplification des équations de Maxwell dans l'ARQS magnétique}
    Dans l'\textbf{Approximation des Régimes Quasi Stationnaires (ARQS) magnétique}, on admet que les courants de déplacement sont négligeables devant les courants de conduction ($\norm{\epsilon_0 \frac{\partial \vect{E}}{\partial t}} \ll \norm{\vect{j}}$).
    Les équations de Maxwell deviennent :
    \begin{enumerate}[label=(\alph*)]
        \item $\diver \vect{E} = \frac{\rho}{\epsilon_0}$ (Maxwell-Gauss)
        \item $\diver \vect{B} = 0$ (Maxwell-flux)
        \item $\rot \vect{E} = - \frac{\partial \vect{B}}{\partial t}$ (Maxwell-Faraday)
        \item $\rot \vect{B} = \mu_0 \vect{j}$ (Maxwell-Ampère sans courant de deplacement)
    \end{enumerate}
    L'équation de conservation de la charge devient (en prenant la divergence de (d)) :
    $\diver (\rot \vect{B}) = \mu_0 \diver \vect{j} \implies 0 = \mu_0 \diver \vect{j}$.
    Ainsi, $\diver \vect{j} = 0$. En combinant avec l'équation de conservation de la charge $\diver \vect{j} + \frac{\partial \rho}{\partial t} = 0$, on en déduit que $\frac{\partial \rho}{\partial t} = 0$. La densité de charge $\rho$ est donc constante dans le temps en tout point, ou nulle.
    \textbf{Extension du domaine de validité des expressions des champs magnétiques obtenues en régime stationnaire} :
    Dans l'ARQS magnétique, l'équation de Maxwell-Ampère simplifiée $\rot \vect{B} = \mu_0 \vect{j}$ est identique à celle du régime stationnaire (magnétostatique). Ainsi, les expressions des champs magnétiques obtenues en régime stationnaire (loi de Biot et Savart, théorème d'Ampère) restent valables à condition de considérer les intensités $I(t)$ et les densités de courant $\vect{j}(t)$ comme des fonctions du temps, tant que l'ARQS est respectée (i.e., les variations sont "lentes").
\end{itemize}

\subsubsection{Induction, courants de Foucault}
\begin{itemize}[label={}]
    \item \textbf{Loi de Faraday-Lenz}
    \textbf{Équation locale} : $\rot \vect{E} = - \frac{\partial \vect{B}}{\partial t}$
    \textbf{Équation intégrale} : La circulation du champ électrique le long d'un contour fermé $\mathcal{C}$ est reliée à la dérivée temporelle du flux magnétique $\Phi$ à travers toute surface $\mathcal{S}$ s'appuyant sur $\mathcal{C}$ :
    $\oint_{\mathcal{C}} \vect{E} \cdot \dd{\vect{l}} = - \frac{\dd{\Phi}}{\dd{t}} = - \frac{\dd{}}{\dd{t}} \iint_{\mathcal{S}} \vect{B} \cdot \dd{\vect{S}} = e_{\text{ind}}$
    où $e_{\text{ind}}$ est la force électromotrice induite.

    \item \textbf{Géométrie des courants de Foucault}
    Dans un conducteur cylindrique soumis à un champ magnétique uniforme et oscillant $\vect{B}(t) = B_0 \cos(\omega t) \vect{u}_z$ parallèle à son axe :
    Le champ $\vect{B}$ variable dans le temps induit un champ électrique $\vect{E}$ qui, par symétrie cylindrique, est tangentiel et forme des boucles concentriques dans les plans orthogonaux à l'axe du cylindre. Ces boucles de champ électrique engendrent des \textbf{courants de Foucault} (courants induits) qui circulent dans des plans perpendiculaires à l'axe du cylindre, sous forme d'anneaux ou de spires concentriques.

    \item \textbf{Puissance dissipée par effet Joule et rôle du feuilletage}
    La puissance dissipée par effet Joule par les courants de Foucault est donnée par :
    $\mathcal{P}_J = \iiint_V \rho_e \vect{j}^2 \dd{V} = \iiint_V \frac{\vect{j}^2}{\sigma} \dd{V}$
    où $\sigma$ est la conductivité électrique du matériau.
    En négligeant le champ propre (le champ magnétique créé par les courants de Foucault eux-mêmes) devant le champ inducteur, le champ électrique induit $E$ est proportionnel à la dérivée du flux magnétique et, dans un conducteur, le courant $j$ est $j = \sigma E$. La puissance dissipée est proportionnelle au carré de la fréquence et au carré du rayon des boucles de courant.
    Le \textbf{feuilletage} consiste à découper le conducteur en fines lamelles isolées les unes des autres et orientées parallèlement aux lignes de champ du flux magnétique. Cela force les courants de Foucault à circuler dans des boucles de plus petit rayon, ce qui augmente la résistance de ces boucles et diminue leur intensité. La puissance dissipée par effet Joule, proportionnelle à $R I^2$, est considérablement réduite, ce qui limite l'échauffement du matériau.

    \item \textbf{Énergie magnétique et densité volumique d'énergie magnétique}
    \begin{itemize}
        \item \textbf{Énergie magnétique d'une bobine seule} :
        $W_m = \frac{1}{2} L I^2$ où $L$ est l'inductance propre et $I$ l'intensité du courant.
        \item \textbf{Énergie magnétique de deux bobines couplées} :
        $W_m = \frac{1}{2} L_1 I_1^2 + \frac{1}{2} L_2 I_2^2 + M I_1 I_2$
        où $L_1, L_2$ sont les inductances propres, $I_1, I_2$ les intensités des courants, et $M$ le coefficient d'inductance mutuelle.
        \item \textbf{Densité volumique d'énergie magnétique} :
        Dans le vide (ou milieu linéaire, homogène, isotrope) : $w_m = \frac{1}{2 \mu_0} \norm{\vect{B}}^2$ ou $w_m = \frac{1}{2} \vect{B} \cdot \vect{H}$.
    \end{itemize}

    \item \textbf{Inégalité $\boldsymbol{M^2 < L_1 L_2}$}
    L'énergie magnétique est une grandeur toujours positive. Pour deux bobines couplées, l'énergie magnétique s'écrit :
    $W_m = \frac{1}{2} L_1 I_1^2 + \frac{1}{2} L_2 I_2^2 + M I_1 I_2 \ge 0$
    Cette expression est une forme quadratique. Pour qu'elle soit toujours positive, le discriminant réduit du polynôme en $I_1$ (ou $I_2$) doit être négatif.
    Considérons $W_m$ comme un polynôme du second degré en $I_1$ :
    $W_m = \frac{1}{2} L_1 I_1^2 + (M I_2) I_1 + \frac{1}{2} L_2 I_2^2$
    Pour que ce polynôme soit toujours positif, son discriminant doit être négatif ou nul (si $L_1 > 0$, ce qui est le cas) :
    $\Delta = (M I_2)^2 - 4 \left( \frac{1}{2} L_1 \right) \left( \frac{1}{2} L_2 I_2^2 \right) \le 0$
    $M^2 I_2^2 - L_1 L_2 I_2^2 \le 0$
    $I_2^2 (M^2 - L_1 L_2) \le 0$
    Pour que cette condition soit vérifiée pour toute valeur de $I_2$, on doit avoir $M^2 - L_1 L_2 \le 0$, donc $M^2 \le L_1 L_2$.
    Le cas $M^2 = L_1 L_2$ correspond à un \textbf{couplage parfait}. En général, il y a toujours des fuites de flux, le couplage n'est jamais parfait et on a $M^2 < L_1 L_2$. Le coefficient de couplage $k = \frac{M}{\sqrt{L_1 L_2}}$ vérifie donc $0 \le k < 1$.

    \item \textbf{Forces de Laplace}
    La force de Laplace élémentaire s'exerçant sur un élément de courant $\dd{\vect{l}}$ parcouru par une intensité $I$ et placé dans un champ magnétique $\vect{B}$ est :
    $\dd{\vect{F}} = I \dd{\vect{l}} \wedge \vect{B}$
    Pour une distribution volumique de courants de densité $\vect{j}$, la force de Laplace élémentaire s'exerçant sur un volume $\dd{V}$ est :
    $\dd{\vect{F}} = \vect{j} \wedge \vect{B} \dd{V}$
\end{itemize}

\subsection{Questions de cours sur la diffusion thermique}

L'\textbf{équation de diffusion thermique} est établie à partir d'un \textbf{bilan d'énergie} et de la \textbf{loi de Fourier}.
Le bilan d'énergie dans un volume $\dd{V}$ donne :
$\rho c_p \frac{\partial T}{\partial t} \dd{V} = - \diver \vect{j}_Q \dd{V} + p_v \dd{V}$
où $\rho$ est la masse volumique, $c_p$ la capacité thermique massique à pression constante, $\vect{j}_Q$ le vecteur densité de courant thermique, et $p_v$ la puissance volumique des sources de chaleur.
La \textbf{loi de Fourier} relie le flux de chaleur au gradient de température :
$\vect{j}_Q = - \lambda \grad T$
où $\lambda$ est la conductivité thermique.
En substituant la loi de Fourier dans le bilan d'énergie, on obtient l'équation de diffusion thermique :
$\rho c_p \frac{\partial T}{\partial t} = \diver (\lambda \grad T) + p_v$
Si $\lambda$ est constante et uniforme :
$\rho c_p \frac{\partial T}{\partial t} = \lambda \Delta T + p_v$
On introduit la \textbf{diffusivité thermique} $\kappa = \frac{\lambda}{\rho c_p}$ :
$\frac{\partial T}{\partial t} = \kappa \Delta T + \frac{p_v}{\rho c_p}$

\begin{enumerate}
    \item \textbf{Dans le cas d'une diffusion unidirectionnelle en coordonnées cartésiennes (selon $\boldsymbol{x}$)}
    L'opérateur laplacien $\Delta$ en coordonnées cartésiennes est $\Delta T = \frac{\partial^2 T}{\partial x^2} + \frac{\partial^2 T}{\partial y^2} + \frac{\partial^2 T}{\partial z^2}$.
    Pour une diffusion unidirectionnelle selon $x$, la température $T$ ne dépend que de $x$ et $t$.
    Donc, $\Delta T = \frac{\partial^2 T}{\partial x^2}$.
    L'équation de diffusion thermique devient :
    $\frac{\partial T}{\partial t} = \kappa \frac{\partial^2 T}{\partial x^2} + \frac{p_v}{\rho c_p}$

    \item \textbf{Dans le cas d'une diffusion radiale en coordonnées cylindriques}
    L'opérateur laplacien $\Delta$ en coordonnées cylindriques est $\Delta T = \frac{1}{r} \frac{\partial}{\partial r} \left( r \frac{\partial T}{\partial r} \right) + \frac{1}{r^2} \frac{\partial^2 T}{\partial \theta^2} + \frac{\partial^2 T}{\partial z^2}$.
    Pour une diffusion radiale, la température $T$ ne dépend que de $r$ et $t$.
    Donc, $\Delta T = \frac{1}{r} \frac{\partial}{\partial r} \left( r \frac{\partial T}{\partial r} \right)$.
    L'équation de diffusion thermique devient :
    $\frac{\partial T}{\partial t} = \kappa \frac{1}{r} \frac{\partial}{\partial r} \left( r \frac{\partial T}{\partial r} \right) + \frac{p_v}{\rho c_p}$

    \item \textbf{Dans le cas d'une diffusion radiale en coordonnées sphériques}
    L'opérateur laplacien $\Delta$ en coordonnées sphériques est $\Delta T = \frac{1}{r^2} \frac{\partial}{\partial r} \left( r^2 \frac{\partial T}{\partial r} \right) + \frac{1}{r^2 \sin \theta} \frac{\partial}{\partial \theta} \left( \sin \theta \frac{\partial T}{\partial \theta} \right) + \frac{1}{r^2 \sin^2 \theta} \frac{\partial^2 T}{\partial \phi^2}$.
    Pour une diffusion radiale, la température $T$ ne dépend que de $r$ et $t$.
    Donc, $\Delta T = \frac{1}{r^2} \frac{\partial}{\partial r} \left( r^2 \frac{\partial T}{\partial r} \right)$.
    L'équation de diffusion thermique devient :
    $\frac{\partial T}{\partial t} = \kappa \frac{1}{r^2} \frac{\partial}{\partial r} \left( r^2 \frac{\partial T}{\partial r} \right) + \frac{p_v}{\rho c_p}$
\end{enumerate}

\section{Notions clés à retenir}
\begin{itemize}
    \item \textbf{Principe de Le Châtelier et loi de Van't Hoff} : Comprendre comment la température et la pression affectent l'équilibre chimique et le rendement.
    \item \textbf{Courants de deplacement et ARQS magnétique} : Savoir définir les courants de deplacement, justifier et appliquer l'ARQS magnétique pour simplifier les équations de Maxwell.
    \item \textbf{Loi de Faraday-Lenz} : Maîtriser les formes locale et intégrale et leur application aux phénomènes d'induction.
    \item \textbf{Courants de Foucault et feuilletage} : Comprendre leur origine, leur géométrie et le rôle pratique du feuilletage pour minimiser les pertes Joule.
    \item \textbf{Énergie magnétique et couplage} : Connaître les expressions de l'énergie pour une ou deux bobines, la densité volumique et l'inégalité de couplage $M^2 < L_1 L_2$.
    \item \textbf{Équation de diffusion thermique} : Savoir l'établir à partir des principes fondamentaux (bilan énergétique et loi de Fourier) et l'exprimer dans différents systèmes de coordonnées (cartésiennes, cylindriques, sphériques) pour des situations à symétrie.
\end{itemize}

\section{Erreurs fréquentes}
\begin{itemize}
    \item \textbf{Confusion entre $K^{\circ}$ et $Q_r$} : Ne pas mélanger la constante d'équilibre ($K^{\circ}$, dépend de T) et le quotient réactionnel ($Q_r$, dépend des concentrations/pressions à un instant donné).
    \item \textbf{Oubli du signe dans la loi de Faraday} : Le signe moins dans $\rot \vect{E} = - \frac{\partial \vect{B}}{\partial t}$ et $e_{\text{ind}} = - \frac{\dd{\Phi}}{\dd{t}}$ est essentiel (loi de Lenz).
    \item \textbf{Négligence des courants de deplacement sans justification} : Les courants de deplacement ne sont négligeables que dans l'ARQS magnétique, dont les conditions doivent être précisées.
    \item \textbf{Erreurs sur les opérateurs différentiels} : En coordonnées cylindriques ou sphériques, le laplacien pour une dépendance radiale a des formes spécifiques ($ \frac{1}{r} \frac{\partial}{\partial r} (r \frac{\partial T}{\partial r})$ et $ \frac{1}{r^2} \frac{\partial}{\partial r} (r^2 \frac{\partial T}{\partial r})$).
    \item \textbf{Mauvaise interprétation de l'influence de la pression ou d'un inerte} : Distinguer l'effet de l'ajout d'un inerte à volume constant (aucun) de celui à pression constante (déplacement de l'équilibre).
    \item \textbf{Inégalité de couplage $\boldsymbol{M^2 < L_1 L_2}$} : Oublier la démonstration via la positivité de l'énergie magnétique ou le cas d'égalité pour le couplage parfait.
    \item \textbf{Accents dans les indices mathématiques} : Utiliser $\text{...}$ pour les indices ou les exposants pour éviter les erreurs de compilation (par exemple, $I_{\text{enlace}}$ et non $I_{\text{enlace}}$).
\end{itemize}

\end{document}