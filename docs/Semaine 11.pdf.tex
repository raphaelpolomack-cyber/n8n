\documentclass[12pt,a4paper]{article}
\usepackage[utf8]{inputenc}
\usepackage[french]{babel}
\usepackage{amsmath,amssymb}
\usepackage{mathtools}
\usepackage{siunitx}
\usepackage{esint}
\usepackage{enumitem}
\usepackage{geometry}
\geometry{a4paper, left=2.5cm, right=2.5cm, top=2.5cm, bottom=2.5cm}

% Definitions mathematiques
\DeclareMathOperator{\rot}{rot}
\DeclareMathOperator{\diver}{div}
\newcommand{\dd}[1]{\mathrm{d}#1}
\newcommand{\norm}[1]{\left\|#1\right\|}
\newcommand{\vect}[1]{\vec{#1}}

\title{Correction de colle}
\author{Correction type}
\date{\today}


% Definition robuste de grad
\providecommand{\grad}{\nabla}
\begin{document}
\maketitle

\section{Réponses aux questions de cours}

\subsection*{Chimie : Optimisation d'un procédé chimique}

\begin{itemize}
    \item \textbf{Critère d'évolution :}
    Un système chimique évolue spontanément dans le sens où l'enthalpie libre de réaction $\Delta_r G$ est négative lorsque la transformation est réalisée à température et pression constantes, ou plus généralement, si $\Delta_r G \cdot \dd{\xi} < 0$. Le critère $\frac{\dd{Q_r}}{Q_r} \cdot \dd{\xi} < 0$ est équivalent et signifie que le système évolue pour se rapprocher de l'équilibre ($Q_r = K^\circ$).

    \item \textbf{Modification de la valeur de $K^\circ$ :}
    La constante d'équilibre $K^\circ$ ne dépend que de la température. Pour une réaction exothermique ($\Delta_r H^\circ < 0$), une augmentation de $T$ diminue $K^\circ$. Pour une réaction endothermique ($\Delta_r H^\circ > 0$), une augmentation de $T$ augmente $K^\circ$. Cette dépendance est décrite par la \textbf{loi de Van't Hoff} :
    $\frac{\dd{\ln K^\circ}}{\dd{T}} = \frac{\Delta_r H^\circ}{R T^2}$.

    \item \textbf{Modification de la valeur du quotient réactionnel $Q_r$ :}
    Le quotient réactionnel peut être modifié par :
    \begin{itemize}
        \item L'\textbf{introduction ou la consommation d'un constituant actif} (réactif ou produit) : Le système évolue dans le sens qui tend à s'opposer à cette modification (principe de Le Châtelier).
        \item La \textbf{modification de la pression totale} pour les réactions avec variation du nombre de moles gazeuses ($\Delta n_g \neq 0$) : Une augmentation de pression déplace l'équilibre vers le sens qui minimise le nombre de moles gazeuses.
        \item L'\textbf{introduction d'un constituant inerte} : À volume constant, elle n'a pas d'influence sur l'équilibre. À pression constante, elle augmente le volume et diminue les pressions partielles des réactifs gazeux, agissant comme une diminution de pression.
    \end{itemize}
    Une étude au cas par cas doit être menée pour prédire le sens d'évolution.
\end{itemize}

\subsection*{Électromagnétisme V : Les régimes variables}

\begin{itemize}
    \item \textbf{Compatibilité des équations de Maxwell avec la conservation de la charge :}
    L'équation locale de conservation de la charge est $\diver \vect{J} + \frac{\partial \rho}{\partial t} = 0$.
    Prenons la divergence de l'équation de Maxwell-Ampère : $\rot \vect{B} = \mu_0 \vect{J} + \mu_0 \varepsilon_0 \frac{\partial \vect{E}}{\partial t}$.
    $\diver (\rot \vect{B}) = \mu_0 \diver \vect{J} + \mu_0 \varepsilon_0 \frac{\partial (\diver \vect{E})}{\partial t}$.
    Comme $\diver (\rot \vect{B}) = 0$, et d'après l'équation de Maxwell-Gauss $\diver \vect{E} = \frac{\rho}{\varepsilon_0}$, on obtient :
    $0 = \mu_0 \diver \vect{J} + \mu_0 \varepsilon_0 \frac{\partial (\rho/\varepsilon_0)}{\partial t} = \mu_0 \left( \diver \vect{J} + \frac{\partial \rho}{\partial t} \right)$.
    Ceci implique $\diver \vect{J} + \frac{\partial \rho}{\partial t} = 0$, démontrant la compatibilité.

    \item \textbf{Simplification des équations de Maxwell et de l'équation de conservation de la charge dans l'ARQS magnétique :}
    L'\textbf{Approximation des Régimes Quasi Stationnaires (ARQS)} magnétique est valide si le temps caractéristique de variation des champs est très grand devant le temps de propagation des ondes électromagnétiques sur la taille caractéristique du système. Dans ce cadre, on admet que les \textbf{courants de deplacement} sont négligeables devant les \textbf{courants de conduction} : $\norm{\varepsilon_0 \frac{\partial \vect{E}}{\partial t}} \ll \norm{\vect{J}}$.
    Les équations de Maxwell deviennent alors :
    \begin{itemize}
        \item $\diver \vect{E} = \frac{\rho}{\varepsilon_0}$
        \item $\diver \vect{B} = 0$
        \item $\rot \vect{E} = - \frac{\partial \vect{B}}{\partial t}$ (Maxwell-Faraday)
        \item $\rot \vect{B} = \mu_0 \vect{J}$ (Ampère en ARQS magnétique)
    \end{itemize}
    L'équation de conservation de la charge reste $\diver \vect{J} + \frac{\partial \rho}{\partial t} = 0$.

    \item \textbf{Extension du domaine de validité des expressions des champs magnétiques obtenues en régime stationnaire :}
    Dans l'\textbf{ARQS magnétique}, l'équation de Maxwell-Ampère simplifiée $\rot \vect{B} = \mu_0 \vect{J}$ est la même que celle de la magnétostatique. Par conséquent, les expressions des champs magnétiques obtenues en régime stationnaire pour une distribution de courants donnée sont directement transposables aux régimes variables, à condition que l'ARQS magnétique soit valide. Le champ magnétique est alors à tout instant celui qu'il serait en régime stationnaire avec les mêmes courants.

    \item \textbf{Relation entre la circulation du champ électrique à la dérivée temporelle du flux magnétique :}
    Cette relation est donnée par l'\textbf{équation de Maxwell-Faraday} :
    \begin{itemize}
        \item \textbf{Forme locale :} $\rot \vect{E} = - \frac{\partial \vect{B}}{\partial t}$
        \item \textbf{Forme intégrale (Loi de Faraday) :} Pour un circuit fermé $\mathcal{C}$ délimitant une surface $\mathcal{S}$ orientée :
            $\oint_{\mathcal{C}} \vect{E} \cdot \dd{\vect{l}} = - \frac{\dd{}}{\dd{t}} \iint_{\mathcal{S}} \vect{B} \cdot \dd{\vect{S}} = - \frac{\dd{\Phi}}{\dd{t}} = e_{\text{ind}}$
            où $\Phi$ est le flux magnétique à travers $\mathcal{S}$ et $e_{\text{ind}}$ est la force électromotrice induite.
    \end{itemize}

    \item \textbf{Géométrie des courants de Foucault dans le cas d’un conducteur cylindrique soumis à un champ magnétique parallèle à son axe, uniforme et oscillant :}
    Considérons un cylindre conducteur de rayon $R$ et d'axe $Oz$, soumis à un champ magnétique $\vect{B}(t) = B_0 \cos(\omega t) \vect{e_z}$ uniforme et parallèle à son axe.
    Le champ magnétique étant variable, il induit un champ électrique selon la loi de Faraday. Par les symétries du problème, le champ électrique induit $\vect{E}$ sera tangentiel et circulaire dans les plans $z=\text{constante}$. Les \textbf{courants de Foucault} $\vect{J} = \sigma \vect{E}$ seront donc des boucles circulaires (des "tourbillons") dans les plans perpendiculaires à l'axe du cylindre, centrées sur l'axe $Oz$. Leur intensité sera proportionnelle à la distance $r$ à l'axe (pour $r<R$) et leur sens alternera avec le temps.

    \item \textbf{Puissance dissipée par effet Joule en négligeant le champ propre et rôle du feuilletage :}
    La \textbf{puissance volumique} dissipée par effet Joule est $p_J = \vect{J} \cdot \vect{E} = \frac{J^2}{\sigma} = \sigma E^2$. La puissance totale dissipée est $P_J = \iiint_V p_J \dd{V}$.
    Dans le cas des courants de Foucault, si l'on néglige le champ propre généré par les courants eux-mêmes devant le champ inducteur externe, la puissance dissipée est proportionnelle au carré de la fréquence et au carré de l'amplitude du champ magnétique.
    Le \textbf{feuilletage} consiste à découper la pièce conductrice en fines lamelles isolées les unes des autres (par exemple, par une fine couche de vernis isolant) et orientées parallèlement aux lignes de champ du champ inducteur. Ce découpage augmente fortement la résistance des chemins de courant des boucles de Foucault, réduisant ainsi leur intensité et donc la puissance dissipée par effet Joule. C'est une technique essentielle pour réduire les pertes dans les transformateurs ou les moteurs électriques.

    \item \textbf{Énergie magnétique d'une bobine seule ou de deux bobines couplées en fonction des coefficients d'inductance et des intensités :}
    \begin{itemize}
        \item \textbf{Bobine seule :} L'énergie magnétique stockée dans une bobine d'inductance propre $L$ parcourue par un courant $I$ est : $W_m = \frac{1}{2} L I^2$.
        \item \textbf{Deux bobines couplées :} Pour deux bobines d'inductances propres $L_1, L_2$ et d'inductance mutuelle $M$, parcourues respectivement par des courants $I_1, I_2$, l'énergie magnétique stockée est : $W_m = \frac{1}{2} L_1 I_1^2 + \frac{1}{2} L_2 I_2^2 + M I_1 I_2$.
    \end{itemize}

    \item \textbf{Expression de la densité volumique d'énergie magnétique :}
    La \textbf{densité volumique d'énergie magnétique} $w_m$ est l'énergie magnétique par unité de volume. Dans le vide ou un milieu linéaire, homogène et isotrope sans matière magnétique, elle est donnée par : $w_m = \frac{1}{2\mu_0} B^2$.

    \item \textbf{Établir, dans le cas de deux bobines couplées, l'inégalité $M^2 < L_1 L_2$ :}
    L'énergie magnétique stockée dans deux bobines couplées est $W_m = \frac{1}{2} L_1 I_1^2 + \frac{1}{2} L_2 I_2^2 + M I_1 I_2$.
    Cette énergie doit être une fonction quadratique positivement définie, c'est-à-dire que $W_m \ge 0$ pour toutes les valeurs de $I_1$ et $I_2$ (et $W_m=0$ seulement si $I_1=I_2=0$).
    On peut la voir comme un trinôme du second degré en $I_1$ (pour un $I_2$ donné et non nul) :
    $W_m = \frac{1}{2} L_1 I_1^2 + (M I_2) I_1 + \frac{1}{2} L_2 I_2^2$.
    Pour que ce trinôme soit toujours positif, son discriminant doit être négatif (en supposant $L_1 > 0$) :
    $\Delta = (M I_2)^2 - 4 \left(\frac{1}{2} L_1\right) \left(\frac{1}{2} L_2 I_2^2\right) = M^2 I_2^2 - L_1 L_2 I_2^2 = (M^2 - L_1 L_2) I_2^2$.
    Il faut $\Delta \le 0$ pour que $W_m \ge 0$. Puisque $I_2^2 \ge 0$, on doit avoir $M^2 - L_1 L_2 \le 0$, ce qui implique $M^2 \le L_1 L_2$.
    L'égalité $M^2 = L_1 L_2$ correspond au \textbf{couplage parfait}, qui n'est pas réalisable en pratique (toujours des fuites de flux), d'où l'inégalité stricte $M^2 < L_1 L_2$ en général.
\end{itemize}

\subsection*{Questions de cours sur la diffusion thermique : Établir l’équation de diffusion thermique vérifiée par la température avec un terme source.}

L'équation de diffusion thermique est issue du premier principe de la thermodynamique (conservation de l'énergie) appliqué à un volume de contrôle $\dd{V}$, combiné à la loi de Fourier et à la définition de la capacité thermique.

Considérons un volume $\dd{V}$ de matière. La variation de son énergie interne est due :
\begin{itemize}
    \item Au \textbf{flux de chaleur} $\vect{j_Q}$ traversant sa surface. La variation d'énergie due au flux est $-\diver \vect{j_Q} \dd{V}$.
    \item À la \textbf{production de chaleur} par un terme source volumique $P_v$ (énergie par unité de volume et de temps) au sein du volume ($\dd{Q}_{source} = P_v \dd{V}$).
\end{itemize}
L'énergie interne volumique $u$ est liée à la température $T$ par $u = \rho c_p T$ (où $\rho$ est la masse volumique et $c_p$ la capacité thermique massique à pression constante).
Le premier principe s'écrit : $\rho c_p \frac{\partial T}{\partial t} \dd{V} = -\diver \vect{j_Q} \dd{V} + P_v \dd{V}$.
La \textbf{loi de Fourier} pour la conduction thermique est $\vect{j_Q} = - \lambda \vect{\nabla} T$, où $\lambda$ est la conductivité thermique.
En substituant la loi de Fourier, on obtient l'équation générale de diffusion thermique :
$\rho c_p \frac{\partial T}{\partial t} = \diver (\lambda \vect{\nabla} T) + P_v$.
Si la conductivité thermique $\lambda$ est constante, l'équation devient :
$\rho c_p \frac{\partial T}{\partial t} = \lambda \Delta T + P_v$.

\begin{enumerate}
    \item \textbf{Dans le cas d’une diffusion unidirectionnelle en coordonnées cartésiennes : }
    Si la température ne dépend que de $x$ et $t$, $T(x,t)$, le laplacien $\Delta T = \frac{\partial^2 T}{\partial x^2}$.
    L'équation devient : $\rho c_p \frac{\partial T}{\partial t} = \frac{\partial}{\partial x} \left( \lambda \frac{\partial T}{\partial x} \right) + P_v$.
    Si $\lambda$ est constante : $\rho c_p \frac{\partial T}{\partial t} = \lambda \frac{\partial^2 T}{\partial x^2} + P_v$.

    \item \textbf{Dans le cas d’une diffusion radiale en coordonnées cylindriques : }
    Si la température ne dépend que de $r$ et $t$, $T(r,t)$, le laplacien en coordonnées cylindriques est $\Delta T = \frac{1}{r} \frac{\partial}{\partial r} \left( r \frac{\partial T}{\partial r} \right)$.
    L'équation devient : $\rho c_p \frac{\partial T}{\partial t} = \frac{1}{r} \frac{\partial}{\partial r} \left( r \lambda \frac{\partial T}{\partial r} \right) + P_v$.
    Si $\lambda$ est constante : $\rho c_p \frac{\partial T}{\partial t} = \lambda \left( \frac{1}{r} \frac{\partial T}{\partial r} + \frac{\partial^2 T}{\partial r^2} \right) + P_v$.

    \item \textbf{Dans le cas d’une diffusion radiale en coordonnées sphériques : }
    Si la température ne dépend que de $r$ et $t$, $T(r,t)$, le laplacien en coordonnées sphériques est $\Delta T = \frac{1}{r^2} \frac{\partial}{\partial r} \left( r^2 \frac{\partial T}{\partial r} \right)$.
    L'équation devient : $\rho c_p \frac{\partial T}{\partial t} = \frac{1}{r^2} \frac{\partial}{\partial r} \left( r^2 \lambda \frac{\partial T}{\partial r} \right) + P_v$.
    Si $\lambda$ est constante : $\rho c_p \frac{\partial T}{\partial t} = \lambda \left( \frac{2}{r} \frac{\partial T}{\partial r} + \frac{\partial^2 T}{\partial r^2} \right) + P_v$.
\end{enumerate}

\section{Notions clés à retenir}
\begin{itemize}
    \item \textbf{Critère d'évolution et équilibre chimique} : La spontanéité d'une réaction est régie par l'enthalpie libre de réaction $\Delta_r G$. À l'équilibre, $Q_r = K^\circ$ et $\Delta_r G = 0$.
    \item \textbf{Influence des paramètres sur l'équilibre} : Distinction entre l'action sur $K^\circ$ (uniquement la température, loi de Van't Hoff) et sur $Q_r$ (concentrations, pressions, principe de Le Châtelier).
    \item \textbf{Le rôle du courant de deplacement} : Terme crucial dans les équations de Maxwell en régime variable, assurant la conservation de la charge et la propagation des ondes électromagnétiques. Son omission caractérise l'ARQS magnétique.
    \item \textbf{Loi de Faraday-Lenz} : Exprime l'induction électromagnétique, la création d'un champ électrique et d'une f.e.m. induite par la variation d'un flux magnétique. Le signe moins indique le sens de Lenz (opposition à la cause).
    \item \textbf{Courants de Foucault} : Courants induits dans des conducteurs massifs soumis à des champs magnétiques variables. Ils sont sources de pertes par effet Joule et peuvent être limités par le feuilletage.
    \item \textbf{Énergie magnétique et inductances} : L'énergie stockée dans les bobines est liée aux inductances propres ($L$) et mutuelles ($M$). L'inégalité $M^2 < L_1 L_2$ traduit le fait que le couplage parfait n'est pas réalisable.
    \item \textbf{Équation de diffusion thermique} : Modélise la propagation de la chaleur dans un milieu. Elle découle de la conservation de l'énergie et de la loi de Fourier, et sa forme dépend du système de coordonnées et des symétries.
\end{itemize}

\section{Erreurs fréquentes}
\begin{itemize}
    \item \textbf{Confondre $K^\circ$ et $Q_r$} : $K^\circ$ est une constante d'équilibre dépendant uniquement de $T$, $Q_r$ est le quotient réactionnel calculé à tout instant et qui tend vers $K^\circ$ à l'équilibre.
    \item \textbf{Oublier le courant de deplacement} : En régime variable, ne pas inclure $\mu_0 \varepsilon_0 \frac{\partial \vect{E}}{\partial t}$ dans l'équation de Maxwell-Ampère est une erreur, sauf si l'ARQS est explicitement justifiée.
    \item \textbf{Erreur de signe dans la loi de Faraday} : Oublier le signe moins dans $\rot \vect{E} = - \frac{\partial \vect{B}}{\partial t}$ ou $\oint \vect{E} \cdot \dd{\vect{l}} = - \frac{\dd{\Phi}}{\dd{t}}$. Ce signe est fondamental pour le principe de Lenz.
    \item \textbf{Formules de laplacien incorrectes} : Utiliser le mauvais opérateur laplacien ou en oublier des termes lors du passage en coordonnées cylindriques ou sphériques. Toujours vérifier la forme avec ou sans $\lambda$ variable.
    \item \textbf{Oublier le terme source} : Dans les problèmes de diffusion thermique, ne pas inclure le terme $P_v$ s'il y a des productions de chaleur volumiques.
    \item \textbf{Syntaxe LaTeX incorrecte} : Utilisation de `\textbf` sans accolade ouvrante ou fermante, ou utilisation d'accents dans les indices mathématiques (par exemple, `$I_{\text{enlace}}$` au lieu de `$I_{\text{enlace}}$`).
\end{itemize}

\end{document}