\documentclass[12pt,a4paper]{article}
\usepackage[utf8]{inputenc}
\usepackage[french]{babel}
\usepackage{amsmath,amssymb}
\usepackage{mathtools}
\usepackage{siunitx}
\usepackage{esint}
\usepackage{enumitem}
\usepackage{geometry}
\geometry{a4paper, left=2.5cm, right=2.5cm, top=2.5cm, bottom=2.5cm}

% Definitions mathematiques
\DeclareMathOperator{\rot}{rot}
\DeclareMathOperator{\diver}{div}
\newcommand{\dd}[1]{\mathrm{d}#1}
\newcommand{\norm}[1]{\left\|#1\right\|}
\newcommand{\vect}[1]{\vec{#1}}

\title{Correction de colle}
\author{Correction type}
\date{\today}


% Definition robuste de grad
\providecommand{\grad}{\nabla}
\begin{document}
\maketitle

\section{Réponses aux questions de cours}

\subsection{Optimisation d'un procédé chimique}

L'optimisation d'un procédé chimique réversible consiste à déplacer l'équilibre chimique dans le sens de la formation des produits afin de maximiser le rendement.

\begin{itemize}[label=$\bullet$]
    \item \textbf{Modification de la valeur de $\boldsymbol{K^\circ}$} :
    La constante d'équilibre standard $K^\circ$ est une fonction uniquement de la température $T$. Sa dépendance est donnée par la \textbf{loi de Van't Hoff} :
    \[ \frac{\dd{\ln K^\circ}}{\dd{T}} = \frac{\Delta_r H^\circ}{R T^2} \]
    où $\Delta_r H^\circ$ est l'enthalpie standard de réaction et $R$ la constante des gaz parfaits.
    \begin{itemize}
        \item Pour une réaction endothermique ($\Delta_r H^\circ > 0$), augmenter la température augmente $K^\circ$ et favorise la formation des produits.
        \item Pour une réaction exothermique ($\Delta_r H^\circ < 0$), diminuer la température augmente $K^\circ$.
    \end{itemize}
    
    \item \textbf{Modification de la valeur du quotient réactionnel $\boldsymbol{Q_r}$} :
    Le quotient réactionnel $Q_r$ peut être modifié en agissant sur les concentrations ou les pressions partielles des constituants.
    \begin{itemize}
        \item \textbf{Introduction d'un constituant actif} (réactif ou produit) :
        L'ajout d'un réactif ou le retrait d'un produit (par exemple par distillation) diminue $Q_r$ par rapport à $K^\circ$, ce qui déplace l'équilibre dans le sens direct (formation des produits). Inversement, l'ajout d'un produit ou le retrait d'un réactif déplace l'équilibre dans le sens inverse.
        \item \textbf{Influence de la pression} (pour les réactions en phase gazeuse) :
        Une augmentation de la pression totale (par compression, c'est-à-dire diminution du volume) déplace l'équilibre dans le sens qui diminue le nombre de moles de gaz. Inversement, une diminution de pression favorise le sens qui augmente le nombre de moles de gaz. Cette est une application du \textbf{principe de Le Châtelier}. La pression n'affecte pas $K^\circ$, mais elle modifie les pressions partielles et donc $Q_r$.
        \item \textbf{Introduction d'un constituant inerte} :
        L'ajout d'un gaz inerte à volume constant ne modifie pas les pressions partielles des gaz réactifs et produits, donc $Q_r$ et l'équilibre ne sont pas affectés. Si l'ajout se fait à pression constante (ce qui implique une augmentation du volume total), alors les pressions partielles de tous les gaz diminuent, et l'équilibre se déplace dans le sens qui augmente le nombre total de moles de gaz (effet similaire à une diminution de pression).
    \end{itemize}
    
    \item \textbf{Critère d'évolution utilisée} :
    Le critère d'évolution spontanée d'un système est donné par la variation de l'enthalpie libre de réaction :
    \[ \Delta_r G \cdot \dd{\xi} < 0 \]
    où $\Delta_r G$ est l'enthalpie libre de réaction et $\dd{\xi}$ est l'avancement de réaction. Cette condition est celle qui assure la diminution de l'enthalpie libre du système, caractéristique d'une évolution spontanée vers l'équilibre. La formulation alternative $\frac{\dd{Q_r}}{Q_r} \cdot \dd{\xi} < 0$ n'est pas le critère standard pour l'évolution spontanée et peut prêter à confusion.

\end{itemize}

\subsection{Électromagnétisme V : Les régimes variables}

\subsubsection{Compatibilité des équations de Maxwell avec la conservation de la charge}
Pour établir la compatibilité, nous partons de l'équation de Maxwell-Ampère et de l'équation de Maxwell-Gauss.
L'équation de Maxwell-Ampère est :
\[ \rot \vect{H} = \vect{j} + \frac{\partial \vect{D}}{\partial t} \]
Prenons la divergence de cette équation :
\[ \diver (\rot \vect{H}) = \diver \vect{j} + \diver \left(\frac{\partial \vect{D}}{\partial t}\right) \]
Nous savons que la divergence d'un rotationnel est toujours nulle : $\diver (\rot \vect{H}) = 0$. De plus, les opérateurs divergence et dérivée temporelle commutent : $\diver \left(\frac{\partial \vect{D}}{\partial t}\right) = \frac{\partial}{\partial t} (\diver \vect{D})$.
L'équation précédente se simplifie alors en :
\[ 0 = \diver \vect{j} + \frac{\partial}{\partial t} (\diver \vect{D}) \]
Utilisons maintenant l'équation de Maxwell-Gauss : $\diver \vect{D} = \rho$. En substituant cette expression, nous obtenons :
\[ \diver \vect{j} + \frac{\partial \rho}{\partial t} = 0 \]
C'est l'\textbf{équation de conservation de la charge} (ou équation de continuité locale), qui exprime que la variation temporelle de la densité de charge en un point est compensée par un flux de courant (de conduction et de convection) sortant de ce point. Les équations de Maxwell sont donc intrinsèquement compatibles avec le principe de conservation de la charge électrique.

\subsubsection{Simplification des équations de Maxwell et de l'équation de conservation de la charge dans l'ARQS magnétique}
L'\textbf{Approximation des Régimes Quasi-Stationnaires (ARQS) magnétique} est une simplification valable lorsque le courant de déplacement $\frac{\partial \vect{D}}{\partial t}$ est négligeable devant le courant de conduction $\vect{j}$ ($\norm{\vect{j}} \gg \norm{\frac{\partial \vect{D}}{\partial t}}$). Ceci se produit lorsque les phénomènes de propagation des ondes électromagnétiques sont suffisamment rapides pour être considérés comme instantanés à l'échelle du système étudié.
Dans cette approximation, les équations de Maxwell se simplifient comme suit :
\begin{itemize}
    \item \textbf{Maxwell-Ampère} :
    \[ \rot \vect{H} = \vect{j} \]
    Le terme de courant de déplacement est négligé.
    \item \textbf{Maxwell-Gauss} :
    \[ \diver \vect{D} = \rho \]
    Cette équation ne subit pas de simplification directe.
    \item \textbf{Maxwell-Faraday} :
    \[ \rot \vect{E} = - \frac{\partial \vect{B}}{\partial t} \]
    Cette équation ne subit pas de simplification, le champ électrique peut être non conservatif en présence d'un champ magnétique variable.
    \item \textbf{Maxwell-Thomson} :
    \[ \diver \vect{B} = 0 \]
    Cette équation ne subit pas de simplification.
\end{itemize}
L'\textbf{équation de conservation de la charge} ne se simplifie pas directement dans l'ARQS magnétique et conserve sa forme générale :
\[ \diver \vect{j} + \frac{\partial \rho}{\partial t} = 0 \]
En effet, même si le courant de déplacement est négligeable, la densité de charge $\rho$ peut varier au cours du temps, ce qui génère des courants de conduction.

\subsubsection{Extension du domaine de validité des expressions des champs magnétiques obtenues en régime stationnaire}
En régime stationnaire (magnétostatique), les champs ne varient pas dans le temps ($\frac{\partial}{\partial t} = 0$). L'équation de Maxwell-Ampère se réduit à $\rot \vect{H} = \vect{j}$.
Dans l'ARQS magnétique, l'équation de Maxwell-Ampère est également $\rot \vect{H} = \vect{j}$.
Par conséquent, toutes les méthodes et expressions établies en magnétostatique (comme le théorème d'Ampère pour le calcul du champ magnétique ou la loi de Biot et Savart) restent valables dans l'ARQS magnétique. Cela signifie que les champs magnétiques sont déterminés localement par les courants de conduction qui les produisent, sans retard de propagation significatif, même si ces courants varient dans le temps.

\subsubsection{Relation entre la circulation du champ électrique et la dérivée temporelle du flux magnétique}
Cette relation est l'expression de la \textbf{loi de Faraday} (ou équation de Maxwell-Faraday) sous ses formes locale et intégrale.
\begin{itemize}
    \item \textbf{Forme locale} :
    \[ \rot \vect{E} = - \frac{\partial \vect{B}}{\partial t} \]
    \item \textbf{Forme intégrale} :
    Pour passer à la forme intégrale, nous appliquons le \textbf{théorème de Stokes} à la forme locale sur une surface $\mathcal{S}$ ouverte orientée, bordée par un contour fermé $\mathcal{C}$ :
    \[ \iint_{\mathcal{S}} (\rot \vect{E}) \cdot \dd{\vect{S}} = \oint_{\mathcal{C}} \vect{E} \cdot \dd{\vect{l}} \]
    En substituant l'expression de $\rot \vect{E}$ :
    \[ \oint_{\mathcal{C}} \vect{E} \cdot \dd{\vect{l}} = \iint_{\mathcal{S}} \left(- \frac{\partial \vect{B}}{\partial t}\right) \cdot \dd{\vect{S}} \]
    Si le contour $\mathcal{C}$ (et donc la surface $\mathcal{S}$) est fixe par rapport au référentiel, nous pouvons intervertir la dérivation temporelle et l'intégration spatiale :
    \[ \oint_{\mathcal{C}} \vect{E} \cdot \dd{\vect{l}} = - \frac{\dd{}}{\dd{t}} \left( \iint_{\mathcal{S}} \vect{B} \cdot \dd{\vect{S}} \right) \]
    On définit le flux magnétique $\Phi = \iint_{\mathcal{S}} \vect{B} \cdot \dd{\vect{S}}$. La relation devient donc :
    \[ \oint_{\mathcal{C}} \vect{E} \cdot \dd{\vect{l}} = - \frac{\dd{\Phi}}{\dd{t}} = e_{\text{ind}} \]
    Cette relation exprime que la force électromotrice induite ($e_{\text{ind}}$) dans un circuit fermé est égale à l'opposé de la dérivée temporelle du flux magnétique qui le traverse.

\end{itemize}

\subsubsection{Géométrie des courants de Foucault dans le cas d'un conducteur cylindrique}
Considérons un cylindre conducteur d'axe $Oz$, soumis à un champ magnétique uniforme, parallèle à son axe, et oscillant : $\vect{B}(t) = B_0 \cos(\omega t) \vect{u_z}$.
Un tel champ variable induit une force électromotrice (f.e.m.) et donc un champ électrique $\vect{E}$ dans le conducteur, d'après la loi de Faraday. Pour une boucle circulaire de rayon $r$ centrée sur l'axe $Oz$ dans un plan perpendiculaire à $Oz$, le flux magnétique est $\Phi(r,t) = \pi r^2 B(t) = \pi r^2 B_0 \cos(\omega t)$.
La f.e.m. induite est $e_{\text{ind}} = - \frac{\dd{\Phi}}{\dd{t}} = \pi r^2 B_0 \omega \sin(\omega t)$.
Par symétrie, le champ électrique induit $\vect{E}$ est azimuthal et de module $E_\theta$. Sa circulation le long de la boucle est $e_{\text{ind}} = E_\theta (2\pi r)$. D'où $E_\theta(r,t) = \frac{1}{2} r B_0 \omega \sin(\omega t)$.
Les \textbf{courants de Foucault} sont des courants induits dans la masse du conducteur. Leur densité $\vect{j} = \sigma \vect{E}$ (où $\sigma$ est la conductivité électrique) est donc également azimutale.
La géométrie des courants de Foucault est celle de \textbf{spires circulaires concentriques}, contenues dans des plans perpendiculaires à l'axe du cylindre, dont l'amplitude augmente linéairement avec la distance $r$ à l'axe du cylindre.

\subsubsection{Puissance dissipée par effet Joule et rôle du feuilletage}
La puissance volumique dissipée par effet Joule est $p_J = \vect{j} \cdot \vect{E}$.
En négligeant le champ magnétique propre créé par les courants de Foucault eux-mêmes, le champ électrique $\vect{E}$ est celui induit par le champ extérieur. Dans un conducteur ohmique, $\vect{j} = \sigma \vect{E}$.
La puissance volumique dissipée devient :
\[ p_J = \sigma E^2 \]
Pour le cas du cylindre décrit précédemment, $E = E_\theta(r,t)$, donc :
\[ p_J(r,t) = \sigma \left( \frac{1}{2} r B_0 \omega \sin(\omega t) \right)^2 = \frac{1}{4} \sigma r^2 B_0^2 \omega^2 \sin^2(\omega t) \]
La puissance totale dissipée $P_J$ est obtenue en intégrant $p_J$ sur le volume du conducteur. Elle est proportionnelle à $\sigma B_0^2 \omega^2$ et, pour un cylindre de rayon $R$, elle dépend de $R^4$ (après intégration en $r$ et sur la hauteur).

Le \textbf{rôle du feuilletage} : Pour limiter les pertes d'énergie par effet Joule dues aux courants de Foucault (qui peuvent entraîner un échauffement excessif et une perte d'efficacité), les conducteurs sont souvent \textbf{feuilletés}. Cela consiste à les fabriquer en empilant de fines lames de matériau conducteur, isolées les unes des autres par une fine couche isolante. Ces lames sont orientées parallèlement à la direction du champ magnétique induisant les courants. En réduisant la dimension caractéristique des boucles de courant possibles (le rayon $R$ effectif de chaque lame), la puissance dissipée, qui est proportionnelle à $R^4$, est considérablement diminuée. Cette technique est essentielle dans la construction des transformateurs, des moteurs électriques et d'autres dispositifs électromagnétiques.

\subsubsection{Énergie magnétique d'une bobine seule ou de deux bobines couplées}
\begin{itemize}
    \item \textbf{Énergie magnétique d'une bobine seule} (d'inductance propre $L$) parcourue par un courant $I$ :
    \[ W_m = \frac{1}{2} L I^2 \]
    \item \textbf{Énergie magnétique de deux bobines couplées} (d'inductances propres $L_1$, $L_2$ et d'inductance mutuelle $M$) parcourues par les courants $I_1$ et $I_2$ :
    \[ W_m = \frac{1}{2} L_1 I_1^2 + \frac{1}{2} L_2 I_2^2 + M I_1 I_2 \]
\end{itemize}

\subsubsection{Densité volumique d'énergie magnétique}
La densité volumique d'énergie magnétique $w_m$ en un point de l'espace est donnée par :
\[ w_m = \frac{1}{2} \vect{B} \cdot \vect{H} \]
Dans un milieu linéaire, homogène et isotrope de perméabilité magnétique $\mu$ (où $\vect{B} = \mu \vect{H}$), cette expression peut s'écrire :
\[ w_m = \frac{B^2}{2\mu} = \frac{\mu H^2}{2} \]
Pour le vide ou l'air, où $\mu = \mu_0$ (perméabilité magnétique du vide), on a $w_m = \frac{B^2}{2\mu_0}$.

\subsubsection{Inégalité $\boldsymbol{M^2 < L_1 L_2}$ pour deux bobines couplées}
L'énergie magnétique $W_m$ stockée dans un système est une grandeur physique qui doit toujours être positive ou nulle, quelles que soient les intensités $I_1$ et $I_2$ qui le traversent.
\[ W_m = \frac{1}{2} L_1 I_1^2 + \frac{1}{2} L_2 I_2^2 + M I_1 I_2 \ge 0 \]
Il s'agit d'une forme quadratique. Pour qu'elle soit toujours positive ou nulle, son discriminant doit être négatif ou nul. Considérons la fonction $f(I_1) = \frac{1}{2} L_1 I_1^2 + (M I_2) I_1 + \frac{1}{2} L_2 I_2^2$. Pour que $f(I_1) \ge 0$ pour tout $I_1$ (avec $L_1>0$), son discriminant réduit doit être négatif ou nul :
$(M I_2)^2 - 4 \left(\frac{1}{2} L_1\right) \left(\frac{1}{2} L_2 I_2^2\right) \le 0$
$M^2 I_2^2 - L_1 L_2 I_2^2 \le 0$
$I_2^2 (M^2 - L_1 L_2) \le 0$
Comme $I_2^2 \ge 0$, il faut nécessairement que $M^2 - L_1 L_2 \le 0$.
D'où l'inégalité fondamentale pour l'inductance mutuelle :
\[ M^2 \le L_1 L_2 \]
L'inégalité stricte $M^2 < L_1 L_2$ est souvent utilisée en pratique car un couplage parfait (où toutes les lignes de champ d'une bobine traversent l'autre) est idéal et n'est généralement pas atteint, ce qui signifie que le coefficient de couplage $k = \frac{M}{\sqrt{L_1 L_2}}$ est toujours inférieur à 1 ($k \le 1$).

\subsection{Questions de cours sur la diffusion thermique}

L'\textbf{équation de diffusion thermique} (ou équation de la chaleur) décrit l'évolution de la température dans un matériau en fonction du temps et de la position, sous l'effet de gradients de température et de sources de chaleur. Elle est établie à partir du premier principe de la thermodynamique local et de la loi de Fourier.

\textbf{Établissement de l'équation générale avec un terme source} :
1.  \textbf{Premier principe de la thermodynamique sous forme locale} : Il exprime que la variation de l'énergie interne par unité de volume est due à l'apport de chaleur par conduction et à la puissance volumique des sources de chaleur ($P_v$).
    \[ \rho c_p \frac{\partial T}{\partial t} = - \diver \vect{j_Q} + P_v \]
    où $\rho$ est la masse volumique, $c_p$ la capacité thermique massique (à pression constante), $T$ la température, $\vect{j_Q}$ le vecteur densité de courant thermique, et $P_v$ la puissance volumique des sources de chaleur (par exemple, par effet Joule).
2.  \textbf{Loi de Fourier} : Elle décrit que le flux de chaleur est proportionnel à l'opposé du gradient de température.
    \[ \vect{j_Q} = - \lambda \vect{\nabla} T \]
    où $\lambda$ est la conductivité thermique du matériau. Le signe négatif indique que la chaleur s'écoule des régions de haute température vers les régions de basse température.
3.  \textbf{Combinaison des deux lois} : En substituant l'expression de la loi de Fourier dans l'équation du premier principe :
    \[ \rho c_p \frac{\partial T}{\partial t} = - \diver (-\lambda \vect{\nabla} T) + P_v \]
    En supposant que la conductivité thermique $\lambda$ est constante dans l'espace (matériau homogène) :
    \[ \rho c_p \frac{\partial T}{\partial t} = \lambda \diver (\vect{\nabla} T) + P_v \]
    L'opérateur divergence du gradient est le laplacien ($\Delta = \diver \vect{\nabla}$). Ainsi, l'équation devient :
    \[ \rho c_p \frac{\partial T}{\partial t} = \lambda \Delta T + P_v \]
    En divisant par $\rho c_p$, on obtient la forme canonique de l'équation de diffusion thermique :
    \[ \frac{\partial T}{\partial t} = D_{\text{th}} \Delta T + \frac{P_v}{\rho c_p} \]
    où $D_{\text{th}} = \frac{\lambda}{\rho c_p}$ est le \textbf{coefficient de diffusion thermique}.

Maintenant, appliquons cette équation aux différentes géométries :

\subsubsection{1) Dans le cas d'une diffusion unidirectionnelle en coordonnées cartésiennes}
Si la température $T$ ne dépend que d'une seule coordonnée spatiale, par exemple $x$, alors l'opérateur Laplacien se réduit à la dérivée seconde par rapport à cette coordonnée :
\[ \Delta T = \frac{\partial^2 T}{\partial x^2} \]
L'équation de diffusion thermique devient :
\[ \frac{\partial T}{\partial t} = D_{\text{th}} \frac{\partial^2 T}{\partial x^2} + \frac{P_v}{\rho c_p} \]

\subsubsection{2) Dans le cas d'une diffusion radiale en coordonnées cylindriques}
Si la température $T$ ne dépend que de la coordonnée radiale $r$ (cas de la symétrie cylindrique), l'opérateur Laplacien s'écrit :
\[ \Delta T = \frac{1}{r} \frac{\partial}{\partial r} \left( r \frac{\partial T}{\partial r} \right) \]
L'équation de diffusion thermique devient :
\[ \frac{\partial T}{\partial t} = D_{\text{th}} \frac{1}{r} \frac{\partial}{\partial r} \left( r \frac{\partial T}{\partial r} \right) + \frac{P_v}{\rho c_p} \]

\subsubsection{3) Dans le cas d'une diffusion radiale en coordonnées sphériques}
Si la température $T$ ne dépend que de la coordonnée radiale $r$ (cas de la symétrie sphérique), l'opérateur Laplacien s'écrit :
\[ \Delta T = \frac{1}{r^2} \frac{\partial}{\partial r} \left( r^2 \frac{\partial T}{\partial r} \right) \]
L'équation de diffusion thermique devient :
\[ \frac{\partial T}{\partial t} = D_{\text{th}} \frac{1}{r^2} \frac{\partial}{\partial r} \left( r^2 \frac{\partial T}{\partial r} \right) + \frac{P_v}{\rho c_p} \]

\section{Notions clés à retenir}
\begin{itemize}
    \item \textbf{Équilibre chimique} : Maîtriser l'impact de la température (loi de Van't Hoff) et de la pression/concentrations (principe de Le Châtelier) sur la position de l'équilibre chimique.
    \item \textbf{Critère d'évolution spontanée} : Connaître et appliquer le critère $\Delta_r G \cdot \dd{\xi} < 0$.
    \item \textbf{Conservation de la charge} : Savoir dériver l'équation de continuité à partir des équations de Maxwell (Gauss et Ampère).
    \item \textbf{ARQS magnétique} : Comprendre l'hypothèse de la négligeabilité du courant de déplacement et ses implications pour les équations de Maxwell et la validité des résultats de magnétostatique.
    \item \textbf{Loi de Faraday} : Connaître ses formes locale ($\rot \vect{E} = - \frac{\partial \vect{B}}{\partial t}$) et intégrale ($e_{\text{ind}} = - \frac{\dd{\Phi}}{\dd{t}}$), et la relation entre elles (théorème de Stokes).
    \item \textbf{Courants de Foucault} : Décrire leur origine (induction), leur géométrie dans un cas simple et l'intérêt du feuilletage pour réduire les pertes par effet Joule.
    \item \textbf{Énergie magnétique} : Maîtriser les expressions de l'énergie stockée dans une bobine seule ($\frac{1}{2} L I^2$), dans deux bobines couplées ($\frac{1}{2} L_1 I_1^2 + \frac{1}{2} L_2 I_2^2 + M I_1 I_2$) et la densité volumique d'énergie ($\frac{1}{2} \vect{B} \cdot \vect{H}$).
    \item \textbf{Inégalité d'inductance mutuelle} : Connaître et savoir justifier l'inégalité $M^2 \le L_1 L_2$ (due à la positivité de l'énergie magnétique).
    \item \textbf{Équation de la chaleur} : Savoir l'établir à partir du premier principe de la thermodynamique et de la loi de Fourier, et pouvoir l'écrire dans les différentes systèmes de coordonnées (cartésiennes, cylindriques, sphériques) en diffusion radiale.
\end{itemize}

\section{Erreurs fréquentes}
\begin{itemize}
    \item \textbf{Critère d'évolution chimique} : Confusion sur le signe ou l'expression du critère d'évolution spontanée (utiliser systématiquement $\Delta_r G \cdot \dd{\xi} < 0$).
    \item \textbf{Accents dans les indices mathématiques} : Utiliser des accents (é, è, à...) dans les indices des variables en mode mathématique (ex : $I_{\text{enlace}}$ et non $I_{\text{enlace}}$).
    \item \textbf{Syntaxe des commandes \textbf} : Oublier l'accolade ouvrante après `\textbf` (ex : `\textbf{texte}` et non `\textbf texte`).
    \item \textbf{ARQS magnétique vs. électrique} : Confondre les hypothèses et les simplifications associées aux deux types d'ARQS (par exemple, croire que $\rot \vect{E} = \vect{0}$ en ARQS magnétique).
    \item \textbf{Courants de déplacement} : Oublier le terme de courant de déplacement dans l'équation de Maxwell-Ampère générale ou mal l'interpréter.
    \item \textbf{Sens/géométrie des courants de Foucault} : Difficulté à décrire la géométrie des courants de Foucault ou à expliquer leur origine à partir de la loi de Faraday.
    \item \textbf{Application du Laplacien} : Erreurs dans l'expression de l'opérateur Laplacien $\Delta$ dans les différentes coordonnées (cartésiennes, cylindriques, sphériques) lors de la résolution de l'équation de la chaleur.
    \item \textbf{Omission du terme source} : Oublier le terme $\frac{P_v}{\rho c_p}$ dans l'équation de diffusion thermique lorsque des sources de chaleur sont présentes.
    \item \textbf{Inégalité $\boldsymbol{M^2 < L_1 L_2}$} : Ne pas savoir justifier cette inégalité par la condition de positivité de l'énergie magnétique.
\end{itemize}

\end{document}