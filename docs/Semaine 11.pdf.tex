\documentclass[12pt,a4paper]{article}
\usepackage[utf8]{inputenc}
\usepackage[french]{babel}
\usepackage{amsmath,amssymb}
\usepackage{mathtools}
\usepackage{siunitx}
\usepackage{esint}
\usepackage{enumitem}
\usepackage{geometry}
\geometry{a4paper, left=2.5cm, right=2.5cm, top=2.5cm, bottom=2.5cm}

% Definitions mathematiques
\DeclareMathOperator{\rot}{rot}
\DeclareMathOperator{\diver}{div}
\newcommand{\dd}[1]{\mathrm{d}#1}
\newcommand{\norm}[1]{\left\|#1\right\|}
\newcommand{\vect}[1]{\vec{#1}}

\title{Correction de colle}
\author{Correction type}
\date{\today}


% Definition robuste de grad
\providecommand{\grad}{\nabla}
\begin{document}
\maketitle

\section{Réponses aux questions de cours}

\subsection{Optimisation d'un procédé chimique}

L'optimisation d'un procédé chimique visant à maximiser le rendement d'une réaction réversible implique de déplacer l'équilibre chimique dans le sens désiré. L'état d'équilibre est caractérisé par $\Delta_r G = 0$, ce qui implique que le quotient réactionnel $Q_r$ est égal à la constante d'équilibre $K^\circ(T)$.

\begin{enumerate}[label=\textbf{\arabic*)}]
    \item \textbf{Modification de la valeur de $K^\circ$} :
    La constante d'équilibre $K^\circ$ ne dépend que de la température. Pour modifier sa valeur, il faut donc changer la température du système.
    La loi de Van't Hoff décrit cette dépendance : $\frac{\dd{\ln K^\circ}}{\dd{T}} = \frac{\Delta_r H^\circ}{R T^2}$.
    \begin{itemize}
        \item Pour une réaction endothermique ($\Delta_r H^\circ > 0$), augmenter la température augmente $K^\circ$, favorisant la formation des produits.
        \item Pour une réaction exothermique ($\Delta_r H^\circ < 0$), diminuer la température augmente $K^\circ$, favorisant la formation des produits.
    \end{itemize}

    \item \textbf{Modification de la valeur du quotient réactionnel $Q_r$} :
    Le quotient réactionnel $Q_r$ peut être modifié par les actions suivantes, en accord avec le principe de Le Châtelier :
    \begin{itemize}
        \item \textbf{Introduction d'un constituant actif} :
        L'introduction d'un réactif ou l'élimination d'un produit déplace l'équilibre dans le sens direct (formation des produits). Inversement, l'introduction d'un produit ou l'élimination d'un réactif déplace l'équilibre dans le sens inverse.
        \item \textbf{Influence de la pression} :
        L'influence de la pression est significative pour les réactions impliquant des gaz et où le nombre de moles de gaz varie ($\Delta n_{\text{gaz}} \neq 0$). Une augmentation de la pression totale (par diminution du volume) déplace l'équilibre dans le sens qui diminue le nombre total de moles de gaz. Une diminution de la pression déplace l'équilibre dans le sens qui augmente le nombre total de moles de gaz. Si $\Delta n_{\text{gaz}} = 0$, la pression n'a pas d'influence sur l'équilibre.
        \item \textbf{Introduction d'un constituant inerte} :
        \begin{itemize}
            \item À \textbf{volume constant} : L'ajout d'un gaz inerte augmente la pression totale mais ne modifie pas les pressions partielles des réactifs et des produits. Le quotient réactionnel $Q_r$ reste inchangé, et l'équilibre n'est pas déplacé.
            \item À \textbf{pression constante} : L'ajout d'un gaz inerte augmente le volume total. Les pressions partielles des réactifs et des produits diminuent. Cela a le même effet qu'une diminution de la pression totale, déplaçant l'équilibre dans le sens qui augmente le nombre de moles de gaz.
        \end{itemize}
    \end{itemize}

    \item \textbf{Critère d’évolution utilisé} :
    L'évolution spontanée d'un système se fait dans le sens où l'enthalpie libre de réaction $\Delta_r G$ est négative pour une transformation infinitésimale $\dd{\xi}$ : $\Delta_r G \cdot \dd{\xi} < 0$.
    À l'équilibre, $\Delta_r G = 0$.
    On peut relier $\Delta_r G$ à $Q_r$ et $K^\circ$ par $\Delta_r G = RT \ln\left(\frac{Q_r}{K^\circ}\right)$.
    Le critère $\frac{\dd{Q_r}}{Q_r} \cdot \dd{\xi} < 0$ signifie que le système évolue pour que $Q_r$ se rapproche de $K^\circ$. Si $Q_r > K^\circ$, $Q_r$ diminue ($\dd{Q_r} < 0$), ce qui correspond à $\dd{\xi} < 0$ (sens inverse). Si $Q_r < K^\circ$, $Q_r$ augmente ($\dd{Q_r} > 0$), ce qui correspond à $\dd{\xi} > 0$ (sens direct). Il faut être vigilant sur le signe de $\dd{\xi}$ qui dépend du sens de la réaction écrite. Le critère $\Delta_r G \cdot \dd{\xi} < 0$ est plus général et moins ambigu.
\end{enumerate}

\subsection{Électromagnétisme V : Les régimes variables}

\begin{enumerate}[label=\textbf{\arabic*)}]
    \item \textbf{Courants de deplacement et ARQS magnétique} :
    \begin{itemize}
        \item Les \textbf{courants de deplacement} sont le terme $\epsilon_0 \frac{\partial \vect{E}}{\partial t}$ dans l'équation de Maxwell-Ampère. Ils représentent la contribution d'un champ électrique variable dans le temps à la création d'un champ magnétique.
        \item L'\textbf{Approximation des Régimes Quasi Stationnaires (ARQS) magnétique} est valide lorsque les courants de deplacement sont négligeables devant les courants de conduction (ou de convection). Mathématiquement, cela signifie que $\left\| \epsilon_0 \frac{\partial \vect{E}}{\partial t} \right\| \ll \left\| \vect{J} \right\|$. Elle est généralement justifiée pour des dimensions du système $L$ petites devant la longueur d'onde des ondes électromagnétiques associées à la fréquence des régimes variables ($L \ll c/f = \lambda$).
    \end{itemize}

    \item \textbf{Induction, courants de Foucault} :
    \begin{itemize}
        \item L'\textbf{induction électromagnétique} est le phénomène par lequel une variation de flux magnétique à travers un circuit crée une force électromotrice (f.e.m.) induite, et donc un courant induit si le circuit est fermé. Ce phénomène est décrit par la loi de Faraday.
        \item Les \textbf{courants de Foucault} (ou courants de eddy) sont des courants induits qui apparaissent dans un matériau conducteur massif soumis à une variation de champ magnétique. Ils se présentent sous forme de boucles fermées dans le conducteur, tendant à s'opposer à la variation du flux magnétique qui les a créés (loi de Lenz).
    \end{itemize}

    \item \textbf{Énergie magnétique et densité volumique d'énergie magnétique} :
    \begin{itemize}
        \item L'\textbf{énergie magnétique} stockée dans une bobine seule d'inductance $L$ parcourue par un courant $I$ est $W_m = \frac{1}{2} L I^2$.
        \item Pour deux bobines couplées d'inductances propres $L_1$, $L_2$ et mutuelle $M$, parcourues par des courants $I_1$ et $I_2$ respectivement, l'énergie magnétique est $W_m = \frac{1}{2} L_1 I_1^2 + \frac{1}{2} L_2 I_2^2 + M I_1 I_2$.
        \item La \textbf{densité volumique d'énergie magnétique} est $w_m = \frac{\vect{B}^2}{2\mu_0}$ (dans le vide ou un milieu non magnétique).
    \end{itemize}

    \item \textbf{Couplage partiel, couplage parfait} :
    Le couplage entre deux bobines est caractérisé par leur inductance mutuelle $M$ et le coefficient de couplage $k = \frac{M}{\sqrt{L_1 L_2}}$.
    \begin{itemize}
        \item Le \textbf{couplage partiel} correspond à $0 < |k| < 1$. Une partie du flux créé par une bobine traverse l'autre.
        \item Le \textbf{couplage parfait} correspond à $|k| = 1$. Tout le flux créé par une bobine traverse l'autre. Cela est une idéalisation difficile à atteindre en pratique.
    \end{itemize}

    \item \textbf{Compatibilité des équations de Maxwell avec la conservation de la charge} :
    L'équation locale de conservation de la charge est $\diver \vect{J} + \frac{\partial \rho}{\partial t} = 0$.
    Partons de l'équation de Maxwell-Ampère : $\rot \vect{B} = \mu_0 \vect{J} + \mu_0 \epsilon_0 \frac{\partial \vect{E}}{\partial t}$.
    Appliquons la divergence à cette équation :
    $\diver(\rot \vect{B}) = \mu_0 \diver \vect{J} + \mu_0 \epsilon_0 \diver\left(\frac{\partial \vect{E}}{\partial t}\right)$.
    On sait que la divergence d'un rotationnel est toujours nulle : $\diver(\rot \vect{B}) = 0$.
    Donc, $0 = \mu_0 \diver \vect{J} + \mu_0 \epsilon_0 \frac{\partial}{\partial t}(\diver \vect{E})$.
    D'après l'équation de Maxwell-Gauss : $\diver \vect{E} = \frac{\rho}{\epsilon_0}$.
    En substituant, on obtient : $0 = \mu_0 \diver \vect{J} + \mu_0 \epsilon_0 \frac{\partial}{\partial t}\left(\frac{\rho}{\epsilon_0}\right)$.
    $0 = \mu_0 \diver \vect{J} + \mu_0 \frac{\partial \rho}{\partial t}$.
    En divisant par $\mu_0$ (qui est non nul), on retrouve bien l'équation de conservation de la charge : $\diver \vect{J} + \frac{\partial \rho}{\partial t} = 0$.

    \item \textbf{Simplification des équations de Maxwell et de la conservation de la charge dans l'ARQS magnétique (courants de deplacement négligeables)} :
    Les équations de Maxwell générales sont :
    \begin{itemize}
        \item Maxwell-Gauss : $\diver \vect{E} = \frac{\rho}{\epsilon_0}$
        \item Maxwell-Flux : $\diver \vect{B} = 0$
        \item Maxwell-Faraday : $\rot \vect{E} = - \frac{\partial \vect{B}}{\partial t}$
        \item Maxwell-Ampère : $\rot \vect{B} = \mu_0 \vect{J} + \mu_0 \epsilon_0 \frac{\partial \vect{E}}{\partial t}$
    \end{itemize}
    L'équation de conservation de la charge est : $\diver \vect{J} + \frac{\partial \rho}{\partial t} = 0$.
    Dans l'\textbf{ARQS magnétique} où les courants de deplacement sont négligeables ($\left\| \mu_0 \epsilon_0 \frac{\partial \vect{E}}{\partial t} \right\| \ll \left\| \mu_0 \vect{J} \right\|$), l'équation de Maxwell-Ampère se simplifie en :
    $\rot \vect{B} = \mu_0 \vect{J}$.
    Les autres équations de Maxwell restent inchangées, et l'équation de conservation de la charge également.

    \item \textbf{Extension du domaine de validité des expressions des champs magnétiques obtenues en régime stationnaire} :
    Dans l'ARQS magnétique, le champ magnétique $\vect{B}$ peut être calculé à partir des courants $\vect{J}$ (qui peuvent être variables dans le temps) en utilisant les mêmes lois que pour le régime stationnaire (lois de Biot et Savart ou théorème d'Ampère). L'hypothèse est que les variations temporelles sont suffisamment lentes pour que la propagation des perturbations puisse être considérée comme instantanée à l'échelle du système. Autrement dit, les champs magnétiques à un instant $t$ sont déterminés par les courants au même instant $t$.

    \item \textbf{Relation entre la circulation du champ électrique et la dérivée temporelle du flux magnétique} :
    Cette relation est donnée par la loi de Faraday pour l'induction électromagnétique :
    \begin{itemize}
        \item \textbf{Forme locale} : $\rot \vect{E} = - \frac{\partial \vect{B}}{\partial t}$
        \item \textbf{Forme intégrale} : En intégrant sur une surface ouverte $\mathcal{S}$ s'appuyant sur un contour fermé $\mathcal{C}$, et en utilisant le théorème de Stokes, on obtient :
        $\oint_{\mathcal{C}} \vect{E} \cdot \dd{\vect{l}} = \iint_{\mathcal{S}} (\rot \vect{E}) \cdot \dd{\vect{S}} = \iint_{\mathcal{S}} \left(- \frac{\partial \vect{B}}{\partial t}\right) \cdot \dd{\vect{S}}$
        Si le contour $\mathcal{C}$ est fixe : $\oint_{\mathcal{C}} \vect{E} \cdot \dd{\vect{l}} = - \frac{\dd{}}{\dd{t}} \iint_{\mathcal{S}} \vect{B} \cdot \dd{\vect{S}} = - \frac{\dd{\Phi}}{\dd{t}} = e_{\text{ind}}$.
        Ici, $\Phi$ est le flux magnétique à travers la surface $\mathcal{S}$, et $e_{\text{ind}}$ est la force électromotrice induite.
    \end{itemize}

    \item \textbf{Géométrie des courants de Foucault dans un conducteur cylindrique} :
    Considérons un conducteur cylindrique d'axe Oz, soumis à un champ magnétique uniforme et oscillant parallèle à son axe : $\vect{B}(t) = B_0 \cos(\omega t) \vect{u_z}$.
    Le flux magnétique à travers une surface circulaire de rayon $r$ et d'axe Oz (perpendiculaire à $\vect{u_z}$) est $\Phi(r,t) = B(t) \cdot \pi r^2 = B_0 \cos(\omega t) \pi r^2$.
    La f.e.m. induite le long d'un cercle de rayon $r$ est $e_{\text{ind}}(r,t) = - \frac{\dd{\Phi}}{\dd{t}} = - B_0 \pi r^2 (-\omega \sin(\omega t)) = B_0 \pi r^2 \omega \sin(\omega t)$.
    Cette f.e.m. induit des courants. La loi d'Ohm locale $\vect{J} = \sigma \vect{E}$ et le fait que $\oint \vect{E} \cdot \dd{\vect{l}} = e_{\text{ind}}$ impliquent que le champ électrique $\vect{E}$ est tangentiel et de module $E = \frac{e_{\text{ind}}}{2\pi r} = \frac{B_0 r \omega}{2} \sin(\omega t)$.
    Les \textbf{courants de Foucault} seront donc des boucles circulaires concentriques à l'axe du cylindre, situées dans des plans perpendiculaires à l'axe Oz. Leur intensité varie avec le rayon $r$ et le temps $t$.

    \item \textbf{Puissance dissipée par effet Joule et rôle du feuilletage} :
    La \textbf{puissance dissipée par effet Joule} dans un conducteur de conductivité $\sigma$ est $P_J = \iiint_V \frac{\vect{J}^2}{\sigma} \dd{V}$. Si le champ propre des courants de Foucault est négligé, le champ $\vect{E}$ est déterminé par l'induction du champ extérieur $\vect{B}$, et $\vect{J} = \sigma \vect{E}$.
    Le \textbf{feuilletage} (ou lamination) consiste à remplacer une pièce conductrice massive par un empilement de fines tôles isolées les unes des autres.
    Le rôle du feuilletage est de \textbf{réduire la puissance dissipée par effet Joule}. En effet, en découpant le conducteur en fines lamelles isolées, on restreint la taille des boucles de courants de Foucault possibles. La résistance électrique de chaque boucle est augmentée, et la f.e.m. induite dans chaque boucle est proportionnelle à sa surface. Par conséquent, l'amplitude des courants de Foucault est fortement réduite, ce qui diminue proportionnellement la dissipation d'énergie par effet Joule.

    \item \textbf{Inégalité $M^2 < L_1 L_2$ pour deux bobines couplées} :
    L'énergie magnétique totale stockée dans deux bobines couplées est $W_m = \frac{1}{2} L_1 I_1^2 + \frac{1}{2} L_2 I_2^2 + M I_1 I_2$.
    Cette énergie doit être positive ou nulle pour toute valeur des courants $I_1$ et $I_2$.
    On peut écrire $W_m = \frac{1}{2} L_1 \left(I_1 + \frac{M}{L_1} I_2\right)^2 + \frac{1}{2} \left(L_2 - \frac{M^2}{L_1}\right) I_2^2$.
    Pour que $W_m \ge 0$ pour toutes les valeurs de $I_1$ et $I_2$, le coefficient devant $I_2^2$ doit être positif ou nul.
    Donc $L_2 - \frac{M^2}{L_1} \ge 0$, ce qui implique $L_1 L_2 - M^2 \ge 0$, soit $M^2 \le L_1 L_2$.
    En général, l'inégalité est stricte $M^2 < L_1 L_2$ car le cas d'égalité $M^2 = L_1 L_2$ correspond à un couplage parfait ($k=1$), ce qui est une idéalisation qui n'entraîne pas une annulation de l'énergie pour des courants non nuls dans un cas réel (où il y a toujours des pertes).

    \item \textbf{Forces de Laplace} :
    La force de Laplace élémentaire $\dd{\vect{F}}$ exercée par un champ magnétique $\vect{B}$ sur un élément de courant $\dd{\vect{l}}$ parcouru par un courant $I$ est :
    $\dd{\vect{F}} = I \dd{\vect{l}} \times \vect{B}$.
    Pour un conducteur filiforme de longueur finie, la force totale est $\vect{F} = \int I \dd{\vect{l}} \times \vect{B}$.
    Pour un volume de courant de densité volumique $\vect{J}$, la force de Laplace élémentaire sur un volume $\dd{V}$ est :
    $\dd{\vect{F}} = \vect{J} \times \vect{B} \dd{V}$.
\end{enumerate}

\subsection{Questions de cours sur la diffusion thermique}

L'équation de diffusion thermique est issue du bilan d'énergie.
Considérons une masse volumique $\rho$, une capacité thermique massique $c$, une conductivité thermique $\lambda$, et une source de puissance volumique $p_v$.
Le bilan d'énergie sur un volume $\dd{V}$ s'écrit : $\rho c \frac{\partial T}{\partial t} \dd{V} = - \diver(\vect{j}_{\text{th}}) \dd{V} + p_v \dd{V}$.
Où $\vect{j}_{\text{th}}$ est le vecteur densité de flux de chaleur, donné par la loi de Fourier : $\vect{j}_{\text{th}} = - \lambda \vect{\nabla} T$.
En substituant et en divisant par $\dd{V}$ : $\rho c \frac{\partial T}{\partial t} = \diver(\lambda \vect{\nabla} T) + p_v$.
Si la conductivité thermique $\lambda$ est constante et uniforme : $\rho c \frac{\partial T}{\partial t} = \lambda \Delta T + p_v$.
On définit la diffusivité thermique $D_{\text{th}} = \frac{\lambda}{\rho c}$. L'équation devient :
$\frac{\partial T}{\partial t} = D_{\text{th}} \Delta T + \frac{p_v}{\rho c}$.

\begin{enumerate}[label=\textbf{\arabic*)}]
    \item \textbf{Diffusion unidirectionnelle en coordonnées cartésiennes} :
    La température ne dépend que d'une seule coordonnée spatiale, par exemple $x$. Le Laplacien $\Delta T = \frac{\partial^2 T}{\partial x^2}$.
    L'équation de diffusion thermique est :
    $\frac{\partial T}{\partial t} = D_{\text{th}} \frac{\partial^2 T}{\partial x^2} + \frac{p_v}{\rho c}$.

    \item \textbf{Diffusion radiale en coordonnées cylindriques} :
    La température ne dépend que de la coordonnée radiale $r$. Le Laplacien $\Delta T = \frac{1}{r} \frac{\partial}{\partial r}\left(r \frac{\partial T}{\partial r}\right)$.
    L'équation de diffusion thermique est :
    $\frac{\partial T}{\partial t} = D_{\text{th}} \frac{1}{r} \frac{\partial}{\partial r}\left(r \frac{\partial T}{\partial r}\right) + \frac{p_v}{\rho c}$.

    \item \textbf{Diffusion radiale en coordonnées sphériques} :
    La température ne dépend que de la coordonnée radiale $r$. Le Laplacien $\Delta T = \frac{1}{r^2} \frac{\partial}{\partial r}\left(r^2 \frac{\partial T}{\partial r}\right)$.
    L'équation de diffusion thermique est :
    $\frac{\partial T}{\partial t} = D_{\text{th}} \frac{1}{r^2} \frac{\partial}{\partial r}\left(r^2 \frac{\partial T}{\partial r}\right) + \frac{p_v}{\rho c}$.
\end{enumerate}

\section{Notions clés à retenir}
\begin{itemize}
    \item \textbf{Équilibre chimique} : La position de l'équilibre est régie par la constante $K^\circ(T)$ et le principe de Le Châtelier. Les actions possibles incluent la variation de température, de pression, et l'ajout/retrait de constituants actifs ou inertes (à volume/pression constante).
    \item \textbf{ARQS magnétique} : Approximation cruciale en électromagnétisme pour les régimes variables lents, permettant de négliger les courants de deplacement et d'appliquer les lois du régime stationnaire pour $\vect{B}$ à chaque instant.
    \item \textbf{Loi de Faraday et induction} : Fondamentale pour comprendre la création de f.e.m. et de courants induits par variation de flux magnétique, sous ses formes locale ($\rot \vect{E} = - \frac{\partial \vect{B}}{\partial t}$) et intégrale ($e_{\text{ind}} = - \frac{\dd{\Phi}}{\dd{t}}$).
    \item \textbf{Courants de Foucault et feuilletage} : Comprendre leur origine, leur géométrie et le principe de réduction des pertes Joule par feuilletage (augmentation des résistances des boucles induites).
    \item \textbf{Énergie magnétique} : Savoir exprimer l'énergie stockée dans des bobines simples ou couplées, et la densité volumique d'énergie. La condition d'énergie positive conduit à l'inégalité $M^2 \le L_1 L_2$.
    \item \textbf{Équation de diffusion thermique} : Maîtriser sa dérivation à partir du bilan d'énergie et de la loi de Fourier, et savoir l'appliquer dans différentes coordonnées (cartésiennes, cylindriques, sphériques) pour des problèmes de diffusion radiale ou unidirectionnelle.
\end{itemize}

\section{Erreurs fréquentes}
\begin{itemize}
    \item \textbf{Confusion $K^\circ$ et $Q_r$} : $K^\circ$ est la constante d'équilibre (dépend de T uniquement), $Q_r$ est le quotient réactionnel (dépend des concentrations/pressions partielles instantanées). L'équilibre est atteint lorsque $Q_r = K^\circ$.
    \item \textbf{Application incorrecte de Le Châtelier} : Notamment pour l'ajout de gaz inerte : l'effet diffère selon que le volume ou la pression totale est maintenu constant.
    \item \textbf{Négligence des courants de deplacement sans justification} : L'ARQS magnétique n'est pas toujours valide, il faut en connaître les conditions d'application.
    \item \textbf{Oubli du signe moins dans la loi de Faraday} : Le signe de Lenz est crucial pour exprimer que l'effet s'oppose à la cause.
    \item \textbf{Erreurs dans les Laplaciens} : Les expressions du Laplacien en coordonnées cylindriques et sphériques sont source d'erreurs (facteurs $1/r$, $1/r^2$).
    \item \textbf{Accents dans les indices mathématiques} : Utiliser $\text{...}$ ou enlever les accents dans les indices pour éviter les erreurs de compilation (ex: $I_{\text{enlace}}$, $I_{\text{deplacement}}$).
    \item \textbf{Syntaxe \texttt{\textbackslash textbf}} : Toujours veiller à la bonne fermeture de l'accolade ouvrante (\texttt{\textbackslash textbf\{...\}}).
\end{itemize}

\end{document}