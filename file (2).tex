\documentclass[11pt,a4paper]{article}
\usepackage[utf8]{inputenc}
\usepackage[T1]{fontenc}
\usepackage[french]{babel}
\usepackage{amsmath}
\usepackage{amssymb}
\usepackage{physics} % For vector notations (div, rot, grad, vec, pdv, dd)
\usepackage{esvect}  % For better looking vectors if needed (optional, physics provides \vec)
\usepackage{enumerate} % For custom list numbering
\usepackage{siunitx} % For units, good practice
\usepackage{geometry} % For page margins
\geometry{a4paper, margin=1in}

% Custom command for differential d, to make it upright in math mode
\renewcommand{\dd}[1]{\ensuremath{\mathrm{d}#1}}

\begin{document}

\section*{Programme de Colle : Physique - Chimie}
\subsection*{Semaine du 17 au 22 novembre}

\subsection*{Remarques Préliminaires}
Le programme couvre des aspects fondamentaux de l'électromagnétisme (électrostatique et magnétostatique) ainsi que des notions d'équilibres chimiques. Une attention particulière doit être portée à la maîtrise des démonstrations des théorèmes intégraux à partir des équations de Maxwell locales, ainsi qu'à l'application rigoureuse des théorèmes de Gauss et d'Ampère, notamment concernant les symétries et l'orientation des surfaces et contours.

\section{Réponses aux questions de cours}

\subsection*{1) Démontrer le théorème de Gauss et le théorème d'Ampère à l'aide des équations de Maxwell ainsi que des théorèmes de Green-Ostrogradski et la Stokes-Ampère.}

\subsubsection*{Démonstration du théorème de Gauss (électrostatique)}
Le théorème de Gauss est une formulation intégrale de l'équation de Maxwell-Gauss. Il relie le flux du champ électrique à travers une surface fermée à la charge électrique totale contenue dans le volume délimité par cette surface.

\begin{enumerate}[(a)]
    \item \textbf{Point de départ : Équation de Maxwell-Gauss (forme locale)}
    En tout point de l'espace, la divergence du champ électrique $\vec{E}$ est proportionnelle à la densité volumique de charge $\rho$ en ce point :
    \begin{equation*}
        \div \vec{E} = \frac{\rho}{\epsilon_0}
    \end{equation*}

    \item \textbf{Intégration sur un volume fermé $V$}
    Intégrons cette équation sur un volume $V$ délimité par une surface fermée $\Sigma$ :
    \begin{equation*}
        \iiint_V \div \vec{E} \dd V = \iiint_V \frac{\rho}{\epsilon_0} \dd V
    \end{equation*}

    \item \textbf{Application du théorème de Green-Ostrogradski (théorème de la divergence)}
    Le théorème de Green-Ostrogradski stipule que l'intégrale de volume de la divergence d'un champ vectoriel est égale au flux de ce champ à travers la surface fermée qui délimite ce volume :
    \begin{equation*}
        \iiint_V \div \vec{E} \dd V = \oint_{\Sigma} \vec{E} \cdot \dd \vec{S}
    \end{equation*}

    \item \textbf{Définition de la charge intérieure}
    La charge électrique totale $Q_{\text{int}}$ contenue dans le volume $V$ est donnée par :
    \begin{equation*}
        Q_{\text{int}} = \iiint_V \rho \dd V
    \end{equation*}

    \item \textbf{Conclusion : Théorème de Gauss}
    En combinant ces résultats, on obtient le théorème de Gauss :
    \begin{equation*}
        \oint_{\Sigma} \vec{E} \cdot \dd \vec{S} = \frac{Q_{\text{int}}}{\epsilon_0}
    \end{equation*}
\end{enumerate}

\subsubsection*{Démonstration du théorème d'Ampère (magnétostatique)}
Le théorème d'Ampère (en régime stationnaire) est une formulation intégrale de l'équation de Maxwell-Ampère. Il relie la circulation du champ magnétique le long d'un contour fermé au courant électrique total enlacé par ce contour.

\begin{enumerate}[(a)]
    \item \textbf{Point de départ : Équation de Maxwell-Ampère (forme locale, régime stationnaire)}
    En régime stationnaire (c'est-à-dire quand les champs électriques et magnétiques sont indépendants du temps, donc $\pdv{\vec{E}}{t} = \vec{0}$), l'équation de Maxwell-Ampère se simplifie à :
    \begin{equation*}
        \curl \vec{B} = \mu_0 \vec{j}
    \end{equation*}
    où $\vec{j}$ est la densité volumique de courant.

    \item \textbf{Intégration sur une surface $S$}
    Prenons une surface $S$ qui s'appuie sur un contour fermé $\Gamma$. Intégrons l'équation précédente sur cette surface :
    \begin{equation*}
        \iint_S \curl \vec{B} \cdot \dd \vec{S} = \iint_S \mu_0 \vec{j} \cdot \dd \vec{S}
    \end{equation*}

    \item \textbf{Application du théorème de Stokes-Ampère (théorème du rotationnel)}
    Le théorème de Stokes-Ampère stipule que l'intégrale de surface du rotationnel d'un champ vectoriel est égale à la circulation de ce champ le long du contour fermé qui délimite cette surface :
    \begin{equation*}
        \iint_S \curl \vec{B} \cdot \dd \vec{S} = \oint_{\Gamma} \vec{B} \cdot \dd \vec{l}
    \end{equation*}
    L'orientation de $\dd \vec{S}$ par rapport à celle de $\dd \vec{l}$ doit respecter la règle du tire-bouchon.

    \item \textbf{Définition du courant enlacé}
    Le courant total $I_{\text{enlacé}}$ traversant la surface $S$ (et donc enlacé par le contour $\Gamma$) est donné par :
    \begin{equation*}
        I_{\text{enlacé}} = \iint_S \vec{j} \cdot \dd \vec{S}
    \end{equation*}

    \item \textbf{Conclusion : Théorème d'Ampère}
    En combinant ces résultats, on obtient le théorème d'Ampère :
    \begin{equation*}
        \oint_{\Gamma} \vec{B} \cdot \dd \vec{l} = \mu_0 I_{\text{enlacé}}
    \end{equation*}
    Il est crucial d'appliquer la règle du tire-bouchon pour l'orientation de $\Gamma$ et $S$ : si on parcourt $\Gamma$ dans un certain sens, le courant $I_{\text{enlacé}}$ est positif s'il traverse $S$ dans le sens du pouce de la main droite.
\end{enumerate}

\subsection*{Application du théorème d'Ampère sur les géométries ci-dessous :}
Pour chaque application, la démarche est la suivante :
\begin{enumerate}
    \item \textbf{Analyse des symétries} : Déduire la direction et la dépendance du champ $\vec{B}$.
    \item \textbf{Choix du contour d'Ampère $\Gamma$} : Un contour fermé judicieux pour simplifier le calcul de la circulation.
    \item \textbf{Calcul de la circulation} : $\oint_{\Gamma} \vec{B} \cdot \dd \vec{l}$.
    \item \textbf{Calcul du courant enlacé} : $I_{\text{enlacé}}$.
    \item \textbf{Application du théorème} : $\oint_{\Gamma} \vec{B} \cdot \dd \vec{l} = \mu_0 I_{\text{enlacé}}$.
    \item \textbf{Orientation} : Toujours préciser l'orientation du contour et de la surface associée avec la règle du tire-bouchon.
\end{enumerate}

\subsubsection*{2) Fil infini parcouru par un courant I.}
Considérons un fil rectiligne infini, orienté selon l'axe $Oz$, parcouru par un courant $I$ dans le sens $Oz$.

\begin{enumerate}[(a)]
    \item \textbf{Symétries} :
    La distribution de courant présente une symétrie cylindrique autour de l'axe $Oz$ et une invariance par translation le long de $Oz$. Par conséquent, le champ magnétique $\vec{B}$ ne peut dépendre que de la distance $r$ à l'axe et est purement azimutal : $\vec{B}(M) = B(r) \vec{u}_{\theta}$.

    \item \textbf{Contour d'Ampère} :
    Nous choisissons un cercle de rayon $r$, centré sur le fil, situé dans un plan perpendiculaire au fil ($z=\text{constante}$). Le contour est parcouru dans le sens positif (sens trigonométrique si on regarde depuis $z>0$).

    \item \textbf{Circulation} :
    Le long de ce contour, $\vec{B}$ est tangent au contour et de module constant $B(r)$. $\dd \vec{l} = r \dd\theta \vec{u}_{\theta}$.
    \begin{equation*}
        \oint_{\Gamma} \vec{B} \cdot \dd \vec{l} = \oint_{\Gamma} B(r) \vec{u}_{\theta} \cdot (r \dd\theta \vec{u}_{\theta}) = B(r) \int_0^{2\pi} r \dd\theta = B(r) (2\pi r)
    \end{equation*}

    \item \textbf{Courant enlacé} :
    La surface délimitée par ce contour est un disque. Le courant $I$ traverse entièrement ce disque.
    Selon la règle du tire-bouchon, si le contour est parcouru dans le sens positif, le courant $I$ est enlacé positivement s'il est dirigé vers les $z$ positifs. Donc $I_{\text{enlacé}} = I$.

    \item \textbf{Application du théorème} :
    \begin{equation*}
        B(r) (2\pi r) = \mu_0 I
    \end{equation*}
    D'où le champ magnétique :
    \begin{equation*}
        \vec{B}(r) = \frac{\mu_0 I}{2\pi r} \vec{u}_{\theta}
    \end{equation*}
\end{enumerate}

\subsubsection*{3) Cylindre plein infini de rayon R parcouru par un courant I.}
Considérons un cylindre plein infini de rayon $R$, orienté selon l'axe $Oz$, parcouru par un courant total $I$ uniformément réparti sur sa section. La densité de courant volumique est donc $j = \frac{I}{\pi R^2}$ pour $r \le R$, et nulle ailleurs.

\begin{enumerate}[(a)]
    \item \textbf{Symétries} :
    Mêmes symétries que le fil infini : symétrie cylindrique et invariance par translation. Le champ magnétique est donc $\vec{B}(M) = B(r) \vec{u}_{\theta}$.

    \item \textbf{Contour d'Ampère} :
    Nous choisissons un cercle de rayon $r$, centré sur l'axe du cylindre, dans un plan perpendiculaire à l'axe.

    \item \textbf{Circulation} :
    Comme pour le fil infini, la circulation vaut $B(r) (2\pi r)$.

    \item \textbf{Courant enlacé} :
    \begin{itemize}
        \item \textbf{Cas 1 : $r < R$ (intérieur du cylindre)}
        Le courant enlacé est le courant qui traverse le disque de rayon $r$.
        Puisque le courant est uniformément réparti, la densité surfacique de courant est constante.
        $I_{\text{enlacé}} = J \times \text{aire du disque de rayon } r = \frac{I}{\pi R^2} \times (\pi r^2) = I \frac{r^2}{R^2}$.
        \item \textbf{Cas 2 : $r \ge R$ (extérieur du cylindre)}
        Le courant enlacé est le courant total $I$ qui traverse la section du cylindre.
        $I_{\text{enlacé}} = I$.
    \end{itemize}

    \item \textbf{Application du théorème} :
    \begin{itemize}
        \item \textbf{Cas 1 : $r < R$}
        $B(r) (2\pi r) = \mu_0 I \frac{r^2}{R^2}$
        \begin{equation*}
            \vec{B}(r) = \frac{\mu_0 I r}{2\pi R^2} \vec{u}_{\theta}
        \end{equation*}
        Le champ croît linéairement avec $r$ à l'intérieur.
        \item \textbf{Cas 2 : $r \ge R$}
        $B(r) (2\pi r) = \mu_0 I$
        \begin{equation*}
            \vec{B}(r) = \frac{\mu_0 I}{2\pi r} \vec{u}_{\theta}
        \end{equation*}
        Le champ décroît comme $1/r$ à l'extérieur, comme pour un fil infini.
    \end{itemize}
\end{enumerate}

\subsubsection*{4) Solénoïde infini parcouru par un courant I en admettant que le champ magnétique extérieur est nul.}
Considérons un solénoïde idéal infini d'axe $Oz$, constitué de $n$ spires par unité de longueur, parcourues par un courant $I$.

\begin{enumerate}[(a)]
    \item \textbf{Symétries} :
    Le système présente une symétrie de révolution autour de l'axe $Oz$ et une invariance par translation le long de $Oz$.
    Par superposition des champs créés par chaque spire (assimilée à un cercle), on déduit que le champ $\vec{B}$ est axial ($\vec{B} = B_z \vec{u}_z$) et ne dépend que de $r$ (distance à l'axe).
    L'hypothèse du solénoïde infini implique que le champ est uniforme à l'intérieur et nul à l'extérieur (ce dernier point est explicitement donné).

    \item \textbf{Contour d'Ampère} :
    Nous choisissons un rectangle ABCD dans un plan contenant l'axe $Oz$.
    \begin{itemize}
        \item Côté AB, de longueur $L$, parallèle à l'axe $Oz$, à l'intérieur du solénoïde.
        \item Côté CD, de longueur $L$, parallèle à l'axe $Oz$, à l'extérieur du solénoïde.
        \item Côtés BC et DA, perpendiculaires à l'axe $Oz$.
    \end{itemize}
    Le contour est parcouru dans le sens A $\rightarrow$ B $\rightarrow$ C $\rightarrow$ D $\rightarrow$ A.

    \item \textbf{Circulation} :
    $\oint_{\Gamma} \vec{B} \cdot \dd \vec{l} = \int_A^B \vec{B} \cdot \dd \vec{l} + \int_B^C \vec{B} \cdot \dd \vec{l} + \int_C^D \vec{B} \cdot \dd \vec{l} + \int_D^A \vec{B} \cdot \dd \vec{l}$
    \begin{itemize}
        \item Sur AB : $\vec{B}$ est parallèle à $\dd \vec{l}$ et uniforme $B_z$. Donc $\int_A^B \vec{B} \cdot \dd \vec{l} = B_z L$.
        \item Sur BC et DA : $\vec{B}$ est axial, $\dd \vec{l}$ est radial. Donc $\vec{B} \cdot \dd \vec{l} = 0$. Les intégrales sont nulles.
        \item Sur CD : $\vec{B}$ est nul à l'extérieur. Donc $\int_C^D \vec{B} \cdot \dd \vec{l} = 0$.
    \end{itemize}
    La circulation totale est donc $B_z L$.

    \item \textbf{Courant enlacé} :
    Le contour ABCD enlace toutes les spires qui traversent la surface définie par le rectangle. Sur une longueur $L$, il y a $nL$ spires. Si le courant $I$ circule dans le sens qui, par la règle du tire-bouchon, donne un champ $B_z$ positif, alors $I_{\text{enlacé}} = nLI$.

    \item \textbf{Application du théorème} :
    $B_z L = \mu_0 nLI$
    \begin{equation*}
        \vec{B} = \mu_0 n I \vec{u}_z \quad \text{(à l'intérieur du solénoïde)}
    \end{equation*}
    Et $\vec{B} = \vec{0}$ à l'extérieur. Le champ est uniforme à l'intérieur.
\end{enumerate}

\subsubsection*{5) Bobine torique parcourue par un courant I.}
Une bobine torique est constituée de $N$ spires enroulées uniformément autour d'un tore (un anneau). Soit $r$ la distance à l'axe de symétrie du tore.

\begin{enumerate}[(a)]
    \item \textbf{Symétries} :
    Le système présente une symétrie de révolution autour de l'axe du tore. Le champ $\vec{B}$ est donc azimutal ($\vec{B} = B(r) \vec{u}_{\theta}$) et ne dépend que de la distance $r$ à l'axe du tore.

    \item \textbf{Contour d'Ampère} :
    Nous choisissons un cercle de rayon $r$, centré sur l'axe du tore, dans le plan médian du tore.

    \item \textbf{Circulation} :
    Le long de ce contour, $\vec{B}$ est tangent au contour et de module constant $B(r)$.
    $\oint_{\Gamma} \vec{B} \cdot \dd \vec{l} = B(r) (2\pi r)$.

    \item \textbf{Courant enlacé} :
    \begin{itemize}
        \item \textbf{Cas 1 : Le contour de rayon $r$ est à l'intérieur du tore (c'est-à-dire que le cercle est contenu dans le matériau où sont enroulées les spires).}
        Toutes les $N$ spires traversent la surface délimitée par le contour.
        Selon la règle du tire-bouchon, si le contour est parcouru dans le sens positif, le courant $I$ est enlacé positivement si les spires sont enroulées de manière à ce que le courant total $N \times I$ traverse la surface dans le sens du pouce.
        $I_{\text{enlacé}} = N I$.
        \item \textbf{Cas 2 : Le contour de rayon $r$ est à l'extérieur du tore (ou à l'intérieur du "trou" du tore).}
        Dans ce cas, aucune spire ne traverse la surface délimitée par le contour (ou bien toutes les spires la traversent une fois dans un sens et une fois dans l'autre, annulant le courant net).
        $I_{\text{enlacé}} = 0$.
    \end{itemize}

    \item \textbf{Application du théorème} :
    \begin{itemize}
        \item \textbf{Cas 1 : À l'intérieur du tore (dans le matériau de la bobine)}
        $B(r) (2\pi r) = \mu_0 N I$
        \begin{equation*}
            \vec{B}(r) = \frac{\mu_0 N I}{2\pi r} \vec{u}_{\theta}
        \end{equation*}
        Le champ n'est pas uniforme mais dépend de $1/r$. Pour un tore "mince" (rayon moyen grand par rapport à la section), on peut l'approximer comme uniforme.
        \item \textbf{Cas 2 : À l'extérieur du tore (et dans le trou central)}
        $B(r) (2\pi r) = \mu_0 \times 0$
        \begin{equation*}
            \vec{B}(r) = \vec{0}
        \end{equation*}
    \end{itemize}
\end{enumerate}

\section{Notions clés à retenir}

\subsection*{Électrostatique}
\begin{itemize}
    \item \textbf{Symétries et invariances} des sources de $\vec{E}$ et leurs conséquences sur la forme de $\vec{E}$.
    \item \textbf{Champ électrostatique conservatif} : Circulation nulle sur un contour fermé ($\oint \vec{E} \cdot \dd \vec{l} = 0$). Implique l'existence d'un potentiel $V$ tel que $\vec{E} = -\grad V$.
    \item \textbf{Relation champ-potentiel} : $\vec{E} = -\grad V$.
    \item \textbf{Surfaces équipotentielles} : Surfaces sur lesquelles $V = \text{constante}$. $\vec{E}$ est toujours orthogonal à ces surfaces.
    \item \textbf{Énergie potentielle électrique} d'une charge $q$ dans un potentiel $V$: $E_p = qV$.
    \item \textbf{Travail de la force électrostatique} : $W_{AB}(\vec{F}_e) = q(V_A - V_B)$.
    \item \textbf{Théorème de Gauss} : $\oint_{\Sigma} \vec{E} \cdot \dd \vec{S} = \frac{Q_{\text{int}}}{\epsilon_0}$.
    \item \textbf{Équations de Maxwell en électrostatique} :
    \begin{itemize}
        \item Maxwell-Gauss : $\div \vec{E} = \frac{\rho}{\epsilon_0}$
        \item Maxwell-Faraday (stationnaire) : $\curl \vec{E} = \vec{0}$
    \end{itemize}
    \item \textbf{Équation de Poisson} : $\Delta V = -\frac{\rho}{\epsilon_0}$. (Si $\rho=0$, alors $\Delta V = 0$, équation de Laplace).
    \item \textbf{Théorème de Green-Ostrogradski} : $\oint_{\Sigma} \vec{E} \cdot \dd \vec{S} = \iiint_V \div \vec{E} \dd V$.
    \item \textbf{Condensateur} :
    \begin{itemize}
        \item Capacité d'un condensateur plan : $C = \frac{\epsilon_0 S}{e}$.
        \item Densité volumique d'énergie électrique : $w_e = \frac{1}{2} \epsilon_0 E^2$.
    \end{itemize}
\end{itemize}

\subsection*{Équilibres chimiques}
\begin{itemize}
    \item \textbf{Grandeurs de réaction} : Enthalpie de réaction ($\Delta_r H$), entropie de réaction ($\Delta_r S$), enthalpie libre de réaction ($\Delta_r G$). Connaître leurs définitions et leurs relations avec les grandeurs standards ($\Delta_r H^0$, $\Delta_r S^0$, $\Delta_r G^0$).
    \item \textbf{Relation entre $\Delta_r G$ et $Q_r$ (quotient de réaction)} :
    $\Delta_r G = \Delta_r G^0 + RT \ln(Q_r) = RT \ln \left( \frac{Q_r}{K^0(T)} \right)$.
    \item \textbf{Relation entre $\Delta_r G^0$ et $K^0(T)$ (constante d'équilibre)} :
    $\Delta_r G^0 = -RT \ln K^0(T)$.
    \item \textbf{Relation de Van't Hoff} :
    $\frac{\dd \ln K^0}{\dd T} = \frac{\Delta_r H^0}{RT^2}$.
    \item \textbf{Critère d'évolution spontanée} : $\Delta_r G \dd\xi < 0$. À l'équilibre, $\Delta_r G = 0$.
    \item \textbf{Approximation d'Ellingham} : $\Delta_r H^0$ et $\Delta_r S^0$ sont indépendants de la température.
\end{itemize}

\subsection*{Magnétostatique}
\begin{itemize}
    \item \textbf{Symétries et invariances} des distributions de courants sources de $\vec{B}$ et leurs conséquences sur la forme de $\vec{B}$.
    \item \textbf{Théorème d'Ampère} : $\oint_{\Gamma} \vec{B} \cdot \dd \vec{l} = \mu_0 I_{\text{enlacé}}$.
    \item \textbf{Calculs de champs magnétiques classiques} : Fil infini, cylindre plein, solénoïde infini, bobine torique.
    \item \textbf{Équations de Maxwell en magnétostatique} :
    \begin{itemize}
        \item Maxwell-Thomson : $\div \vec{B} = 0$
        \item Maxwell-Ampère (stationnaire) : $\curl \vec{B} = \mu_0 \vec{j}$
    \end{itemize}
    \item \textbf{Théorème de Stokes-Ampère} : $\oint_{\Gamma} \vec{B} \cdot \dd \vec{l} = \iint_S \curl \vec{B} \cdot \dd \vec{S}$.
    \item \textbf{Densité volumique d'énergie magnétique} : $w_m = \frac{B^2}{2\mu_0}$.
    \item \textbf{Densité volumique d'énergie électromagnétique} : $w_{em} = \frac{1}{2} \epsilon_0 E^2 + \frac{B^2}{2\mu_0}$.
\end{itemize}

\section{Erreurs fréquentes}

\begin{itemize}
    \item \textbf{Oubli des symétries} : Ne pas utiliser correctement les symétries de la distribution de charges ou de courants pour simplifier l'expression des champs $\vec{E}$ ou $\vec{B}$. C'est la première étape cruciale dans l'application des théorèmes de Gauss et d'Ampère.
    \item \textbf{Mauvaise application des théorèmes intégraux} :
    \begin{itemize}
        \item \textbf{Gauss} : Erreur dans le calcul de la charge intérieure $Q_{\text{int}}$, notamment pour les distributions volumiques ou les charges ponctuelles.
        \item \textbf{Ampère} : Erreur dans le calcul du courant enlacé $I_{\text{enlacé}}$, particulièrement pour les cylindres (distinction $r<R$ et $r>R$) ou les solénoïdes/bobines toriques.
    \end{itemize}
    \item \textbf{Erreur d'orientation} : \textbf{L'erreur la plus critique pour le théorème d'Ampère}. Oublier d'orienter le contour $\Gamma$ et la surface $S$ enlacée à l'aide de la règle du tire-bouchon (ou du tournevis). Cela peut conduire à un signe erroné du champ magnétique. Pour le théorème de Gauss, s'assurer que $\dd \vec{S}$ est bien un vecteur normal sortant de la surface fermée.
    \item \textbf{Confusion entre les régimes} : Appliquer l'équation de Maxwell-Ampère complète (`$\curl \vec{B} = \mu_0 (\vec{j} + \epsilon_0 \pdv{\vec{E}}{t})$`) en magnétostatique, alors qu'en régime stationnaire, le terme de courant de déplacement ($\epsilon_0 \pdv{\vec{E}}{t}$) est nul.
    \item \textbf{Confusion en électrostatique} : Oublier que $\curl \vec{E} = \vec{0}$ en électrostatique, ce qui justifie l'existence d'un potentiel scalaire.
    \item \textbf{Notations vectorielles} : Manque de rigueur dans les notations vectorielles (absence de flèches ou de gras, confusion scalaire/vecteur).
    \item \textbf{Chimie} : Confusion entre les grandeurs standard ($\Delta_r G^0$, $K^0$) et les grandeurs réelles ($\Delta_r G$, $Q_r$). Erreur dans l'utilisation de la relation de Van't Hoff ou le critère d'évolution.
\end{itemize}

\end{document}
