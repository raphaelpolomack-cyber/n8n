\textbackslash documentclass{[}12pt, a4paper{]}\{article\}
\textbackslash usepackage{[}utf8{]}\{inputenc\}
\textbackslash usepackage{[}T1{]}\{fontenc\}
\textbackslash usepackage{[}french{]}\{babel\}
\textbackslash usepackage\{amsmath, amssymb, amsfonts\}
\textbackslash usepackage\{physics\} \% Provides \textbackslash grad,
\textbackslash div, \textbackslash rot, \textbackslash laplacian,
\textbackslash vec \textbackslash usepackage\{geometry\}
\textbackslash usepackage\{siunitx\} \% For units

\textbackslash geometry\{a4paper, margin=1in\}

\textbackslash begin\{document\}

\textbackslash begin\{center\} \textbackslash vspace*\{0.5cm\}
\textbackslash textbf\{\textbackslash Large Correction de Colle :
Électromagnétisme et Équilibres Chimiques\}
\textbackslash vspace*\{0.5cm\} \textbackslash end\{center\}

\textbackslash section\{Réponses aux questions de cours\}

\textbackslash subsection\{Question 1: Démonstration des théorèmes de
Gauss et d\textquotesingle Ampère\}

\textbackslash subsubsection\{Démonstration du Théorème de Gauss\} Le
théorème de Gauss relie le flux du champ électrostatique à travers une
surface fermée aux charges internes. \textbackslash begin\{enumerate\}
\textbackslash item \textbackslash textbf\{Forme locale de
l\textquotesingle équation de Maxwell-Gauss :\} En électrostatique,
l\textquotesingle équation de Maxwell-Gauss s\textquotesingle écrit sous
sa forme locale (ou différentielle) : \$\textbackslash div
\textbackslash vec\{E\} =
\textbackslash frac\{\textbackslash rho\}\{\textbackslash epsilon\_0\}\$
où \$\textbackslash vec\{E\}\$ est le champ électrique,
\$\textbackslash rho\$ est la densité volumique de charge et
\$\textbackslash epsilon\_0\$ est la permittivité du vide.

\textbackslash item \textbackslash textbf\{Application du Théorème de
Green-Ostrogradski (Théorème de la Divergence) :\} Le théorème de
Green-Ostrogradski établit une relation entre
l\textquotesingle intégrale de volume de la divergence
d\textquotesingle un champ vectoriel et le flux de ce champ à travers la
surface fermée qui délimite ce volume :
\$\textbackslash oiint\_\textbackslash Sigma \textbackslash vec\{E\}
\textbackslash cdot d\textbackslash vec\{S\} = \textbackslash iiint\_V
\textbackslash div \textbackslash vec\{E\} \textbackslash,
d\textbackslash tau\$ où \$\textbackslash Sigma\$ est une surface fermée
et \$V\$ est le volume qu\textquotesingle elle enveloppe.

\textbackslash item \textbackslash textbf\{Substitution et conclusion
:\} En substituant l\textquotesingle expression de \$\textbackslash div
\textbackslash vec\{E\}\$ de l\textquotesingle équation de Maxwell-Gauss
dans le théorème de Green-Ostrogradski, on obtient :
\$\textbackslash oiint\_\textbackslash Sigma \textbackslash vec\{E\}
\textbackslash cdot d\textbackslash vec\{S\} = \textbackslash iiint\_V
\textbackslash left(\textbackslash frac\{\textbackslash rho\}\{\textbackslash epsilon\_0\}\textbackslash right)
\textbackslash, d\textbackslash tau\$
\$\textbackslash oiint\_\textbackslash Sigma \textbackslash vec\{E\}
\textbackslash cdot d\textbackslash vec\{S\} =
\textbackslash frac\{1\}\{\textbackslash epsilon\_0\}
\textbackslash iiint\_V \textbackslash rho \textbackslash,
d\textbackslash tau\$ L\textquotesingle intégrale
\$\textbackslash iiint\_V \textbackslash rho \textbackslash,
d\textbackslash tau\$ représente la charge totale \$Q\_\{int\}\$
contenue à l\textquotesingle intérieur du volume \$V\$ délimité par la
surface \$\textbackslash Sigma\$. Ainsi, le théorème de Gauss sous sa
forme intégrale est démontré :
\$\textbackslash oiint\_\textbackslash Sigma \textbackslash vec\{E\}
\textbackslash cdot d\textbackslash vec\{S\} =
\textbackslash frac\{Q\_\{int\}\}\{\textbackslash epsilon\_0\}\$
\textbackslash end\{enumerate\}

\textbackslash subsubsection\{Démonstration du Théorème
d\textquotesingle Ampère\} Le théorème d\textquotesingle Ampère relie la
circulation du champ magnétique le long d\textquotesingle un contour
fermé aux courants enlacés par ce contour.
\textbackslash begin\{enumerate\} \textbackslash item
\textbackslash textbf\{Forme locale de l\textquotesingle équation de
Maxwell-Ampère (en régime stationnaire) :\} En magnétostatique (régime
stationnaire, où \$\textbackslash frac\{\textbackslash partial
\textbackslash vec\{E\}\}\{\textbackslash partial t\} =
\textbackslash vec\{0\}\$), l\textquotesingle équation de Maxwell-Ampère
se simplifie et s\textquotesingle écrit sous sa forme locale :
\$\textbackslash rot \textbackslash vec\{B\} = \textbackslash mu\_0
\textbackslash vec\{j\}\$ où \$\textbackslash vec\{B\}\$ est le champ
magnétique, \$\textbackslash vec\{j\}\$ est la densité volumique de
courant et \$\textbackslash mu\_0\$ est la perméabilité magnétique du
vide.

\textbackslash item \textbackslash textbf\{Application du Théorème de
Stokes-Ampère (Théorème de Stokes) :\} Le théorème de Stokes-Ampère (ou
théorème du rotationnel) établit une relation entre la circulation
d\textquotesingle un champ vectoriel le long d\textquotesingle un
contour fermé et le flux du rotationnel de ce champ à travers toute
surface s\textquotesingle appuyant sur ce contour :
\$\textbackslash oint\_\textbackslash Gamma \textbackslash vec\{B\}
\textbackslash cdot d\textbackslash vec\{l\} = \textbackslash iint\_S
\textbackslash rot \textbackslash vec\{B\} \textbackslash cdot
d\textbackslash vec\{S\}\$ où \$\textbackslash Gamma\$ est un contour
fermé et \$S\$ est une surface s\textquotesingle appuyant sur
\$\textbackslash Gamma\$. L\textquotesingle orientation de
\$\textbackslash Gamma\$ et de \$S\$ est liée par la règle du
tire-bouchon.

\textbackslash item \textbackslash textbf\{Substitution et conclusion
:\} En substituant l\textquotesingle expression de \$\textbackslash rot
\textbackslash vec\{B\}\$ de l\textquotesingle équation de
Maxwell-Ampère dans le théorème de Stokes-Ampère, on obtient :
\$\textbackslash oint\_\textbackslash Gamma \textbackslash vec\{B\}
\textbackslash cdot d\textbackslash vec\{l\} = \textbackslash iint\_S
(\textbackslash mu\_0 \textbackslash vec\{j\}) \textbackslash cdot
d\textbackslash vec\{S\}\$ \$\textbackslash oint\_\textbackslash Gamma
\textbackslash vec\{B\} \textbackslash cdot d\textbackslash vec\{l\} =
\textbackslash mu\_0 \textbackslash iint\_S \textbackslash vec\{j\}
\textbackslash cdot d\textbackslash vec\{S\}\$
L\textquotesingle intégrale \$\textbackslash iint\_S
\textbackslash vec\{j\} \textbackslash cdot d\textbackslash vec\{S\}\$
représente le courant \$I\_\{enlacé\}\$ traversant la surface \$S\$.
Ainsi, le théorème d\textquotesingle Ampère sous sa forme intégrale est
démontré : \$\textbackslash oint\_\textbackslash Gamma
\textbackslash vec\{B\} \textbackslash cdot d\textbackslash vec\{l\} =
\textbackslash mu\_0 I\_\{enlacé\}\$ Il est crucial de bien orienter le
contour \$\textbackslash Gamma\$ et la surface \$S\$ à
l\textquotesingle aide de la règle du tire-bouchon. Si le contour est
parcouru dans un sens donné, le vecteur surface
\$d\textbackslash vec\{S\}\$ est orienté dans le sens
d\textquotesingle avancement d\textquotesingle un tire-bouchon
qu\textquotesingle on tournerait dans le même sens que le contour.
\textbackslash end\{enumerate\}

\textbackslash subsection\{Questions 2 à 5: Applications du théorème
d\textquotesingle Ampère\} Pour chaque application, la méthode consiste
à : \textbackslash begin\{enumerate\} \textbackslash item Étudier les
symétries de la distribution de courant pour déduire la direction et la
dépendance du champ \$\textbackslash vec\{B\}\$. \textbackslash item
Choisir un contour d\textquotesingle Ampère \$\textbackslash Gamma\$
astucieux (souvent un cercle ou un rectangle) où
\$\textbackslash vec\{B\} \textbackslash cdot d\textbackslash vec\{l\}\$
est facile à calculer. \textbackslash item Calculer la circulation
\$\textbackslash oint\_\textbackslash Gamma \textbackslash vec\{B\}
\textbackslash cdot d\textbackslash vec\{l\}\$. \textbackslash item
Calculer le courant \$I\_\{enlacé\}\$ traversant la surface \$S\$
délimitée par \$\textbackslash Gamma\$, en respectant la règle du
tire-bouchon. \textbackslash item Égaler les deux expressions via le
théorème d\textquotesingle Ampère :
\$\textbackslash oint\_\textbackslash Gamma \textbackslash vec\{B\}
\textbackslash cdot d\textbackslash vec\{l\} = \textbackslash mu\_0
I\_\{enlacé\}\$. \textbackslash end\{enumerate\}

\textbackslash subsubsection\{Question 2: Champ magnétique
d\textquotesingle un fil infini parcouru par un courant \$I\$\}
\textbackslash begin\{enumerate\} \textbackslash item
\textbackslash textbf\{Symétries :\} Le fil est infini le long de
l\textquotesingle axe \$Oz\$. La distribution de courant est invariante
par rotation autour de \$Oz\$ et par translation le long de \$Oz\$. Par
conséquent, le champ \$\textbackslash vec\{B\}\$ ne dépend que de la
distance \$r\$ à l\textquotesingle axe
(\$r=\textbackslash sqrt\{x\^{}2+y\^{}2\}\$) et est azimuthal :
\$\textbackslash vec\{B\}(M) = B(r)
\textbackslash vec\{u\}\_\textbackslash theta\$.

\textbackslash item \textbackslash textbf\{Contour
d\textquotesingle Ampère :\} On choisit un cercle
\$\textbackslash Gamma\$ de rayon \$r\$ centré sur le fil, situé dans un
plan perpendiculaire au fil. On oriente \$\textbackslash Gamma\$ dans le
sens de \$\textbackslash vec\{u\}\_\textbackslash theta\$.

\textbackslash item \textbackslash textbf\{Circulation de
\$\textbackslash vec\{B\}\$ :\} Le long de \$\textbackslash Gamma\$,
\$\textbackslash vec\{B\}\$ est tangentiel et de module constant
\$B(r)\$. \$d\textbackslash vec\{l\} = r \textbackslash,
d\textbackslash theta \textbackslash,
\textbackslash vec\{u\}\_\textbackslash theta\$.
\$\textbackslash oint\_\textbackslash Gamma \textbackslash vec\{B\}
\textbackslash cdot d\textbackslash vec\{l\} =
\textbackslash int\_0\^{}\{2\textbackslash pi\} B(r)
\textbackslash vec\{u\}\_\textbackslash theta \textbackslash cdot (r
\textbackslash, d\textbackslash theta \textbackslash,
\textbackslash vec\{u\}\_\textbackslash theta) = B(r) r
\textbackslash int\_0\^{}\{2\textbackslash pi\} d\textbackslash theta =
B(r) (2\textbackslash pi r)\$.

\textbackslash item \textbackslash textbf\{Courant enlacé
\$I\_\{enlacé\}\$ :\} La surface \$S\$ s\textquotesingle appuyant sur
\$\textbackslash Gamma\$ est le disque de rayon \$r\$. Si
\$r\textgreater0\$, le courant \$I\$ traverse ce disque une fois. En
appliquant la règle du tire-bouchon, si \$I\$ est dirigé vers les
\$z\textgreater0\$ et \$\textbackslash Gamma\$ est parcouru dans le sens
trigonométrique, alors \$I\_\{enlacé\} = I\$.

\textbackslash item \textbackslash textbf\{Résultat :\} \$B(r)
(2\textbackslash pi r) = \textbackslash mu\_0 I \textbackslash implies
B(r) = \textbackslash frac\{\textbackslash mu\_0 I\}\{2\textbackslash pi
r\}\$. Finalement : \$\textbackslash vec\{B\}(M) =
\textbackslash frac\{\textbackslash mu\_0 I\}\{2\textbackslash pi r\}
\textbackslash vec\{u\}\_\textbackslash theta\$.
\textbackslash end\{enumerate\}

\textbackslash subsubsection\{Question 3: Champ magnétique
d\textquotesingle un cylindre plein infini de rayon \$R\$\} Le cylindre
est parcouru par un courant total \$I\$ uniformément réparti sur sa
section. On suppose l\textquotesingle axe du cylindre confondu avec
\$Oz\$.

\textbackslash begin\{enumerate\} \textbackslash item
\textbackslash textbf\{Symétries :\} Identiques au fil infini. Le champ
est de la forme \$\textbackslash vec\{B\}(M) = B(r)
\textbackslash vec\{u\}\_\textbackslash theta\$.

\textbackslash item \textbackslash textbf\{Contour
d\textquotesingle Ampère :\} Un cercle \$\textbackslash Gamma\$ de rayon
\$r\$ centré sur l\textquotesingle axe du cylindre. La circulation reste
\$B(r) (2\textbackslash pi r)\$.

\textbackslash item \textbackslash textbf\{Courant enlacé
\$I\_\{enlacé\}\$ :\} \textbackslash begin\{itemize\}
\textbackslash item \textbackslash textbf\{Cas \$r \textless{} R\$
(intérieur du cylindre) :\} Le courant est réparti uniformément. La
densité surfacique de courant est \$j = I/(\textbackslash pi R\^{}2)\$.
Le courant enlacé par \$\textbackslash Gamma\$ est proportionnel à
l\textquotesingle aire du disque de rayon \$r\$ : \$I\_\{enlacé\} = j
(\textbackslash pi r\^{}2) = \textbackslash frac\{I\}\{\textbackslash pi
R\^{}2\} (\textbackslash pi r\^{}2) = I
\textbackslash frac\{r\^{}2\}\{R\^{}2\}\$. \textbackslash item
\textbackslash textbf\{Cas \$r \textbackslash ge R\$ (extérieur du
cylindre) :\} Tout le courant \$I\$ traverse la surface délimitée par
\$\textbackslash Gamma\$. \$I\_\{enlacé\} = I\$.
\textbackslash end\{itemize\}

\textbackslash item \textbackslash textbf\{Résultat :\}
\textbackslash begin\{itemize\} \textbackslash item
\textbackslash textbf\{Pour \$r \textless{} R\$ :\} \$B(r)
(2\textbackslash pi r) = \textbackslash mu\_0 I
\textbackslash frac\{r\^{}2\}\{R\^{}2\} \textbackslash implies B(r) =
\textbackslash frac\{\textbackslash mu\_0 I r\}\{2\textbackslash pi
R\^{}2\}\$. \$\textbackslash vec\{B\}(M) =
\textbackslash frac\{\textbackslash mu\_0 I r\}\{2\textbackslash pi
R\^{}2\} \textbackslash vec\{u\}\_\textbackslash theta\$.
\textbackslash item \textbackslash textbf\{Pour \$r \textbackslash ge
R\$ :\} \$B(r) (2\textbackslash pi r) = \textbackslash mu\_0 I
\textbackslash implies B(r) = \textbackslash frac\{\textbackslash mu\_0
I\}\{2\textbackslash pi r\}\$. \$\textbackslash vec\{B\}(M) =
\textbackslash frac\{\textbackslash mu\_0 I\}\{2\textbackslash pi r\}
\textbackslash vec\{u\}\_\textbackslash theta\$.
\textbackslash end\{itemize\} \textbackslash end\{enumerate\}

\textbackslash subsubsection\{Question 4: Champ magnétique
d\textquotesingle un solénoïde infini\} Un solénoïde infini est
caractérisé par \$n\$ spires par unité de longueur, parcourues par un
courant \$I\$. Son axe est l\textquotesingle axe \$Oz\$.
L\textquotesingle énoncé admet que le champ magnétique extérieur est
nul.

\textbackslash begin\{enumerate\} \textbackslash item
\textbackslash textbf\{Symétries :\} Invariance par translation le long
de \$Oz\$ et par rotation autour de \$Oz\$. La seule direction possible
pour le champ est l\textquotesingle axe \$Oz\$. Donc
\$\textbackslash vec\{B\}(M) = B\_z(r) \textbackslash vec\{u\}\_z\$.
L\textquotesingle hypothèse de solénoïde infini permet de considérer
\$B\_z\$ indépendant de \$z\$. De plus, la symétrie par rapport au plan
médiateur d\textquotesingle une spire implique que \$B\_z\$ ne change
pas de signe si on inverse \$z\$. Enfin, le champ est uniforme à
l\textquotesingle intérieur et nul à l\textquotesingle extérieur. Donc
\$\textbackslash vec\{B\}\_\{int\} = B \textbackslash vec\{u\}\_z\$ et
\$\textbackslash vec\{B\}\_\{ext\} = \textbackslash vec\{0\}\$.

\textbackslash item \textbackslash textbf\{Contour
d\textquotesingle Ampère :\} On choisit un contour rectangulaire
\$ABCD\$ de longueur \$L\$ et largeur \$h\$.
\textbackslash begin\{itemize\} \textbackslash item \$AB\$ est parallèle
à l\textquotesingle axe \$Oz\$, situé à l\textquotesingle intérieur du
solénoïde (longueur \$L\$). \textbackslash item \$CD\$ est parallèle à
l\textquotesingle axe \$Oz\$, situé à l\textquotesingle extérieur du
solénoïde (longueur \$L\$). \textbackslash item \$BC\$ et \$DA\$ sont
perpendiculaires à l\textquotesingle axe \$Oz\$ (longueur \$h\$).
\textbackslash end\{itemize\} On parcourt le contour dans le sens \$A
\textbackslash to B \textbackslash to C \textbackslash to D
\textbackslash to A\$.

\textbackslash item \textbackslash textbf\{Circulation de
\$\textbackslash vec\{B\}\$ :\}
\$\textbackslash oint\_\textbackslash Gamma \textbackslash vec\{B\}
\textbackslash cdot d\textbackslash vec\{l\} =
\textbackslash int\_\{AB\} \textbackslash vec\{B\} \textbackslash cdot
d\textbackslash vec\{l\} + \textbackslash int\_\{BC\}
\textbackslash vec\{B\} \textbackslash cdot d\textbackslash vec\{l\} +
\textbackslash int\_\{CD\} \textbackslash vec\{B\} \textbackslash cdot
d\textbackslash vec\{l\} + \textbackslash int\_\{DA\}
\textbackslash vec\{B\} \textbackslash cdot d\textbackslash vec\{l\}\$
\textbackslash begin\{itemize\} \textbackslash item
\$\textbackslash int\_\{AB\} \textbackslash vec\{B\} \textbackslash cdot
d\textbackslash vec\{l\} = B L\$ (car \$\textbackslash vec\{B\} = B
\textbackslash vec\{u\}\_z\$ et \$d\textbackslash vec\{l\} =
\textbackslash vec\{u\}\_z dz\$ le long de \$AB\$). \textbackslash item
\$\textbackslash int\_\{CD\} \textbackslash vec\{B\} \textbackslash cdot
d\textbackslash vec\{l\} = 0\$ (car \$\textbackslash vec\{B\} =
\textbackslash vec\{0\}\$ à l\textquotesingle extérieur).
\textbackslash item \$\textbackslash int\_\{BC\} \textbackslash vec\{B\}
\textbackslash cdot d\textbackslash vec\{l\} = 0\$ et
\$\textbackslash int\_\{DA\} \textbackslash vec\{B\} \textbackslash cdot
d\textbackslash vec\{l\} = 0\$ (car \$\textbackslash vec\{B\}\$ est
parallèle à \$Oz\$ et \$d\textbackslash vec\{l\}\$ est perpendiculaire).
\textbackslash end\{itemize\} Donc
\$\textbackslash oint\_\textbackslash Gamma \textbackslash vec\{B\}
\textbackslash cdot d\textbackslash vec\{l\} = B L\$.

\textbackslash item \textbackslash textbf\{Courant enlacé
\$I\_\{enlacé\}\$ :\} La surface \$S\$ délimitée par le rectangle
\$ABCD\$ est traversée par les spires du solénoïde. Pour une longueur
\$L\$, il y a \$nL\$ spires. Chaque spire apporte un courant \$I\$. En
respectant la règle du tire-bouchon (si \$I\$ est tel que
\$\textbackslash vec\{B\}\$ est dans le sens de \$AB\$), \$I\_\{enlacé\}
= nLI\$.

\textbackslash item \textbackslash textbf\{Résultat :\} \$B L =
\textbackslash mu\_0 nLI \textbackslash implies B = \textbackslash mu\_0
nI\$. Finalement : \$\textbackslash vec\{B\} = \textbackslash mu\_0 nI
\textbackslash vec\{u\}\_z\$ (à l\textquotesingle intérieur du
solénoïde), et \$\textbackslash vec\{B\} = \textbackslash vec\{0\}\$ (à
l\textquotesingle extérieur). \textbackslash end\{enumerate\}

\textbackslash subsubsection\{Question 5: Champ magnétique
d\textquotesingle une bobine torique\} Une bobine torique est constituée
de \$N\$ spires enroulées uniformément sur un tore (un anneau). Soit
\$I\$ le courant dans chaque spire. On note \$R\_1\$ et \$R\_2\$ les
rayons intérieur et extérieur du tore, et \$r\$ la distance à
l\textquotesingle axe de symétrie du tore.

\textbackslash begin\{enumerate\} \textbackslash item
\textbackslash textbf\{Symétries :\} La distribution de courant est
invariante par rotation autour de l\textquotesingle axe central du tore.
Le champ \$\textbackslash vec\{B\}\$ est donc azimuthal et ne dépend que
de \$r\$ : \$\textbackslash vec\{B\}(M) = B(r)
\textbackslash vec\{u\}\_\textbackslash theta\$.

\textbackslash item \textbackslash textbf\{Contour
d\textquotesingle Ampère :\} On choisit un cercle
\$\textbackslash Gamma\$ de rayon \$r\$ centré sur l\textquotesingle axe
du tore. On oriente \$\textbackslash Gamma\$ dans le sens de
\$\textbackslash vec\{u\}\_\textbackslash theta\$.

\textbackslash item \textbackslash textbf\{Circulation de
\$\textbackslash vec\{B\}\$ :\} Le long de \$\textbackslash Gamma\$,
\$\textbackslash vec\{B\}\$ est tangentiel et de module constant
\$B(r)\$. \$d\textbackslash vec\{l\} = r \textbackslash,
d\textbackslash theta \textbackslash,
\textbackslash vec\{u\}\_\textbackslash theta\$.
\$\textbackslash oint\_\textbackslash Gamma \textbackslash vec\{B\}
\textbackslash cdot d\textbackslash vec\{l\} = B(r) (2\textbackslash pi
r)\$.

\textbackslash item \textbackslash textbf\{Courant enlacé
\$I\_\{enlacé\}\$ :\} \textbackslash begin\{itemize\}
\textbackslash item \textbackslash textbf\{Cas \$r \textless{} R\_1\$ ou
\$r \textgreater{} R\_2\$ (extérieur de l\textquotesingle enroulement)
:\} Le contour \$\textbackslash Gamma\$ n\textquotesingle enlace aucun
courant (ou les courants dans un sens sont compensés par les courants
dans l\textquotesingle autre sens, si l\textquotesingle on considère le
fil comme une boucle fermée sur elle-même). \$I\_\{enlacé\} = 0\$.
\textbackslash item \textbackslash textbf\{Cas \$R\_1 \textless{} r
\textless{} R\_2\$ (intérieur de l\textquotesingle enroulement) :\} Le
contour \$\textbackslash Gamma\$ enlace les \$N\$ spires, chacune
parcourue par le courant \$I\$. En respectant la règle du tire-bouchon,
\$I\_\{enlacé\} = NI\$. \textbackslash end\{itemize\}

\textbackslash item \textbackslash textbf\{Résultat :\}
\textbackslash begin\{itemize\} \textbackslash item
\textbackslash textbf\{Pour \$r \textless{} R\_1\$ ou \$r \textgreater{}
R\_2\$ :\} \$B(r) (2\textbackslash pi r) = \textbackslash mu\_0
\textbackslash cdot 0 \textbackslash implies B(r) = 0\$.
\$\textbackslash vec\{B\}(M) = \textbackslash vec\{0\}\$.
\textbackslash item \textbackslash textbf\{Pour \$R\_1 \textless{} r
\textless{} R\_2\$ :\} \$B(r) (2\textbackslash pi r) =
\textbackslash mu\_0 NI \textbackslash implies B(r) =
\textbackslash frac\{\textbackslash mu\_0 NI\}\{2\textbackslash pi
r\}\$. \$\textbackslash vec\{B\}(M) =
\textbackslash frac\{\textbackslash mu\_0 NI\}\{2\textbackslash pi r\}
\textbackslash vec\{u\}\_\textbackslash theta\$.
\textbackslash end\{itemize\} \textbackslash end\{enumerate\}

\textbackslash section\{Notions clés à retenir\}

\textbackslash subsection\{Électromagnétisme\}
\textbackslash begin\{itemize\} \textbackslash item
\textbackslash textbf\{Champs Électrostatiques :\}
\textbackslash begin\{itemize\} \textbackslash item Théorème de Gauss
(forme intégrale) : \$\textbackslash oiint\_\textbackslash Sigma
\textbackslash vec\{E\} \textbackslash cdot d\textbackslash vec\{S\} =
\textbackslash frac\{Q\_\{int\}\}\{\textbackslash epsilon\_0\}\$.
\textbackslash item Équation de Maxwell-Gauss (forme locale) :
\$\textbackslash div \textbackslash vec\{E\} =
\textbackslash frac\{\textbackslash rho\}\{\textbackslash epsilon\_0\}\$.
\textbackslash item Relation champ-potentiel : \$\textbackslash vec\{E\}
= -\textbackslash grad V\$. \textbackslash item Potentiel
électrostatique et surfaces équipotentielles. \textbackslash item
Énergie potentielle électrique : \$W\_p = qV\$. \textbackslash item
Équation de Poisson : \$\textbackslash Delta V =
-\textbackslash frac\{\textbackslash rho\}\{\textbackslash epsilon\_0\}\$.
\textbackslash item Densité volumique d\textquotesingle énergie
électrique : \$w\_e = \textbackslash frac\{1\}\{2\}
\textbackslash epsilon\_0 E\^{}2\$. \textbackslash item Capacité
d\textquotesingle un condensateur plan : \$C =
\textbackslash frac\{\textbackslash epsilon\_0 S\}\{e\}\$.
\textbackslash end\{itemize\} \textbackslash item
\textbackslash textbf\{Champs Magnétostatiques (régime stationnaire) :\}
\textbackslash begin\{itemize\} \textbackslash item Théorème
d\textquotesingle Ampère (forme intégrale) :
\$\textbackslash oint\_\textbackslash Gamma \textbackslash vec\{B\}
\textbackslash cdot d\textbackslash vec\{l\} = \textbackslash mu\_0
I\_\{enlacé\}\$. \textbackslash item Équation de Maxwell-Ampère (forme
locale, régime stationnaire) : \$\textbackslash rot
\textbackslash vec\{B\} = \textbackslash mu\_0
\textbackslash vec\{j\}\$. \textbackslash item Équation de
Maxwell-Thomson (forme locale) : \$\textbackslash div
\textbackslash vec\{B\} = 0\$. \textbackslash item Densité volumique
d\textquotesingle énergie magnétique : \$w\_m =
\textbackslash frac\{B\^{}2\}\{2\textbackslash mu\_0\}\$.
\textbackslash item Densité volumique d\textquotesingle énergie
électromagnétique : \$w\_\{em\} = w\_e + w\_m\$.
\textbackslash end\{itemize\} \textbackslash item
\textbackslash textbf\{Outils Mathématiques :\}
\textbackslash begin\{itemize\} \textbackslash item Théorème de
Green-Ostrogradski (divergence) :
\$\textbackslash oiint\_\textbackslash Sigma \textbackslash vec\{F\}
\textbackslash cdot d\textbackslash vec\{S\} = \textbackslash iiint\_V
\textbackslash div \textbackslash vec\{F\} \textbackslash,
d\textbackslash tau\$. \textbackslash item Théorème de Stokes
(rotationnel) : \$\textbackslash oint\_\textbackslash Gamma
\textbackslash vec\{F\} \textbackslash cdot d\textbackslash vec\{l\} =
\textbackslash iint\_S \textbackslash rot \textbackslash vec\{F\}
\textbackslash cdot d\textbackslash vec\{S\}\$. \textbackslash item
Importance de l\textquotesingle étude des symétries des sources pour la
détermination des champs. \textbackslash item Règle du tire-bouchon (ou
du tournevis) pour l\textquotesingle orientation cohérente des contours
et des surfaces. \textbackslash end\{itemize\}
\textbackslash end\{itemize\}

\textbackslash subsection\{Équilibres Chimiques\}
\textbackslash begin\{itemize\} \textbackslash item Grandeurs de
réaction : \$\textbackslash Delta\_r H\$, \$\textbackslash Delta\_r S\$,
\$\textbackslash Delta\_r G\$. \textbackslash item Critère
d\textquotesingle évolution spontanée : \$\textbackslash Delta\_r G
\textbackslash cdot d\textbackslash xi \textless{} 0\$.
\textbackslash item Relation entre l\textquotesingle enthalpie libre de
réaction standard et la constante d\textquotesingle équilibre :
\$\textbackslash Delta\_r G\^{}0 = -RT \textbackslash ln K\^{}0(T)\$.
\textbackslash item Relation entre l\textquotesingle enthalpie libre de
réaction et le quotient de réaction : \$\textbackslash Delta\_r G =
\textbackslash Delta\_r G\^{}0 + RT \textbackslash ln Q\_r\$.
\textbackslash item Relation de Van\textquotesingle t Hoff :
\$\textbackslash frac\{d (\textbackslash ln K\^{}0)\}\{dT\} =
\textbackslash frac\{\textbackslash Delta\_r H\^{}0\}\{RT\^{}2\}\$.
\textbackslash item Approximation d\textquotesingle Ellingham :
\$\textbackslash Delta\_r H\^{}0\$ et \$\textbackslash Delta\_r S\^{}0\$
sont indépendants de la température. \textbackslash end\{itemize\}

\textbackslash section\{Erreurs fréquentes\}

\textbackslash begin\{itemize\} \textbackslash item
\textbackslash textbf\{Applications des Théorèmes de Gauss et
d\textquotesingle Ampère :\} \textbackslash begin\{itemize\}
\textbackslash item \textbackslash textbf\{Mauvaise interprétation des
symétries :\} Ne pas justifier correctement la direction du champ ou sa
dépendance aux coordonnées. Par exemple, supposer un champ radial pour
un fil infini. \textbackslash item \textbackslash textbf\{Mauvais choix
de surface de Gauss ou de contour d\textquotesingle Ampère :\} Choisir
une surface/un contour qui ne permet pas de simplifier le calcul de
l\textquotesingle intégrale (e.g., \$\textbackslash vec\{E\}
\textbackslash cdot d\textbackslash vec\{S\}\$ ou
\$\textbackslash vec\{B\} \textbackslash cdot d\textbackslash vec\{l\}\$
non constant ou non nul). \textbackslash item
\textbackslash textbf\{Oubli de la règle du tire-bouchon/tournevis :\}
Inversion du signe de \$Q\_\{int\}\$ ou \$I\_\{enlacé\}\$ à cause
d\textquotesingle une orientation incohérente entre le contour et la
surface associée. \textbackslash item \textbackslash textbf\{Confusion
entre \$Q\_\{int\}\$ et \$Q\_\{total\}\$ ou \$I\_\{enlacé\}\$ et
\$I\_\{total\}\$ :\} Utiliser la charge ou le courant total au lieu de
ceux contenus/enlacés par la surface/le contour choisi.
\textbackslash item \textbackslash textbf\{Absence de justification :\}
Ne pas expliquer pourquoi certaines parties de la circulation ou du flux
sont nulles (e.g., champ nul à l\textquotesingle extérieur
d\textquotesingle un solénoïde, ou champ perpendiculaire au contour).
\textbackslash end\{itemize\} \textbackslash item
\textbackslash textbf\{Formes locales et intégrales :\} Mélanger les
expressions, par exemple utiliser \$\textbackslash div
\textbackslash vec\{E\} = Q\_\{int\}/\textbackslash epsilon\_0\$ au lieu
de \$\textbackslash rho/\textbackslash epsilon\_0\$. \textbackslash item
\textbackslash textbf\{Signes :\} Erreur fréquente sur le signe dans la
relation \$\textbackslash vec\{E\} = -\textbackslash grad V\$.
\textbackslash item \textbackslash textbf\{Régime stationnaire :\}
Oublier que \$\textbackslash rot \textbackslash vec\{E\} =
\textbackslash vec\{0\}\$ et que \$\textbackslash rot
\textbackslash vec\{B\} = \textbackslash mu\_0 \textbackslash vec\{j\}\$
ne sont valables qu\textquotesingle en régime stationnaire (ou négliger
le courant de déplacement). \textbackslash item
\textbackslash textbf\{Constantes physiques :\} Oublier
\$\textbackslash mu\_0\$ ou \$\textbackslash epsilon\_0\$ dans les
formules. \textbackslash item \textbackslash textbf\{Chimie :\}
\textbackslash begin\{itemize\} \textbackslash item Confusion entre
\$Q\_r\$ (quotient de réaction) et \$K\^{}0\$ (constante
d\textquotesingle équilibre). \textbackslash item Erreurs de signe ou de
facteur \$RT\$ dans les expressions d\textquotesingle enthalpie libre de
réaction. \textbackslash item Application incorrecte de la loi de
Van\textquotesingle t Hoff. \textbackslash item Oubli des conditions de
validité des approximations (e.g., Ellingham).
\textbackslash end\{itemize\} \textbackslash item
\textbackslash textbf\{Notation vectorielle :\} Ne pas utiliser de
notation vectorielle pour les champs (flèches au-dessus des symboles).
\textbackslash end\{itemize\}

\textbackslash end\{document\}
